\documentclass{article}
\usepackage[utf8]{inputenc}
\usepackage{graphicx}
\usepackage[spanish]{babel}
\usepackage{amssymb,amsmath,geometry,xcolor}
\usepackage{etoolbox} %titulo
\makeatletter %titulo
\patchcmd{\@maketitle}{\vskip 2em}{\vspace*{-3cm}}{}{} %titulo
\makeatother %titulo
\usepackage{vmargin}
\setpapersize{A4}
\setmargins{2.5cm}             % margen izquierdo
{1.5cm}                        % margen superior
{16.5cm}                       % anchura del texto
{23.42cm}                      % altura del texto
{10pt}                         % altura de los encabezados
{1cm}                          % espacio entre el texto y los encabezados
{0pt}                          % altura del pie de página
{2cm}                          % espacio entre el texto y el pie de página
\title{Cálculo}
\author{Andoni Latorre Galarraga}
\date{}
\newcommand{\bb}[1]{\mathbb{#1}}
\newcommand{\p}[0]{\textbf{Proposición:}}
\newcommand{\dem}[0]{\textit{Dem:}}
\newcommand{\R}{\mathbb{R}}
\begin{document}

\maketitle

\stepcounter{section}
\stepcounter{section}
\stepcounter{section}
\stepcounter{section}

\section{Sucesiones y series de funciones}


De ahora en adelante $X$ representa un conjunto no vacío y $\{f_n\}$ es una sucesión de funciones tales que $f_n\::\:X\longrightarrow\R$.
\subsection{Convergencia puntual}

\textbf{Definición:} Se dice que $\{f_n\}$ converge puntualmente a una función $f\::\:X\longrightarrow\R$ si para todo $x\in \R$
$$
\lim_{n\to\infty}f_n(x)=f(x)
$$

\subsubsection{Ejemplo}

Si $X=[0,1]$ y $f_n(x)=x^n$.
$$
\lim_{n\to\infty}f_n(x)=
\left\{
    \begin{array}{ll}
        0 & 0\le x<1 \\
        1 & x=1
    \end{array}
\right.
$$

\subsubsection{Ejemplo}

Si $X=[0,1]$, y $f_n(x)=\frac{\sen(nx)}{n}$.
$$
\lim_{n\to\infty} f_n = 0
$$

\subsubsection{Ejemplo}

Si $X=[0,1]$ y $f_n= \left\{\begin{array}{ll}
    nx & 0\le x<\frac{1}{n} \\
    2-nx & \frac{1}{n} \le 
\end{array}\right.$
\end{document}