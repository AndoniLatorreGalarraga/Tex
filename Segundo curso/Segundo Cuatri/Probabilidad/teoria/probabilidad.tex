\documentclass{article}
\usepackage[utf8]{inputenc}
\usepackage{graphicx}
\usepackage[spanish]{babel}
\usepackage{amssymb,amsmath,geometry}
\usepackage{vmargin}
\setpapersize{A4}
\setmargins{2.5cm}       % margen izquierdo
{1.5cm}                        % margen superior
{16.5cm}                      % anchura del texto
{23.42cm}                    % altura del texto
{10pt}                           % altura de los encabezados
{1cm}                           % espacio entre el texto y los encabezados
{0pt}                             % altura del pie de página
{2cm}                           % espacio entre el texto y el pie de página
\title{Probabilidad}
\author{Andoni Latorre Galarraga}
\date{}
\begin{document}

\maketitle
\setlength{\parindent}{0cm}

\section{Tema 1}

\subsection{Definiciones básicas}

%ESPACIO MUESTRAL

\textbf{Definición:} El conjunto de todos los resultados posibles de un experimento o fenómeno se denomina espacio muestral asociado al experimento o fenómeno. Se denota $\Omega$.\\\\

%SUCESO

\textbf{Definición:} Un suceso es un subconjunto del espacio muestral, se denota con letras mayúsculas.\\\\

%PUNTO MUESTRAL

\textbf{Definición:} Un punto muestral $\omega$ es un elemento de $\Omega$.\\\\

%SUCESO SEGURO

\textbf{Definición:} $A=\Omega \subset \omega \Rightarrow A$ es suceso seguro.\\\\

%SUCESO IMPOSIBLE

\textbf{Definición:} $A=\emptyset \subset \omega \Rightarrow A$ es suceso imposible.

\subsection{Operaciones entre sucesos}

%COMPLEMENTARIO

\textbf{Definición:} $A^c$ es el complementario de $A$. $A^c$ ocurre si no ocurre A.\\\\

%UNION

\textbf{Definición:} $A\cup B$ es la unión de $A$ y $B$. $A\cup B$ ocurre si ocurre $A$ o ocurre $B$.\\\\

%INTERSECCIÓN

\textbf{Definición:} $A\cap B$ es la intersección de $A$ y $B$. $A\cap B$ ocurre si ocurre $A$ y ocurre $B$.\\\\

%DIFERENCIA

\textbf{Definición:} $A-B$ es la diferencia de $A$ y $B$. $A-B$ ocurre si ocurre $A$ y no ocurre $B$.
\end{document}