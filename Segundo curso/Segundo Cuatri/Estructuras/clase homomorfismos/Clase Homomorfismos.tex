\documentclass{article}
\usepackage[utf8]{inputenc}
\usepackage{graphicx}
\usepackage[spanish]{babel}
\usepackage{amssymb,amsmath,geometry}
\usepackage{etoolbox} %titulo
\makeatletter %titulo
\patchcmd{\@maketitle}{\vskip 2em}{\vspace*{-3cm}}{}{} %titulo
\makeatother %titulo
\usepackage{vmargin}
\setpapersize{A4}
\setmargins{2.5cm}       % margen izquierdo
{1.5cm}                        % margen superior
{16.5cm}                      % anchura del texto
{23.42cm}                    % altura del texto
{10pt}                           % altura de los encabezados
{1cm}                           % espacio entre el texto y los encabezados
{0pt}                             % altura del pie de página
{2cm}                           % espacio entre el texto y el pie de página
\title{Homomorfismos}
\author{Andoni Latorre Galarraga}
\date{}
\begin{document}
\setlength{\parindent}{0cm}
\maketitle


\textbf{Definición:}\\
Sean $(G_1,\cdot)$ y $(G_2,*)$ grupos. Decimos que $f\::\:G_1\longrightarrow G_2$ es un homomorfismo de de grupos, si verifica
$$
f(x\cdot y)=f(x)*f(y)\quad\forall x,y\in G_1
$$


\textbf{Ejemplo: }\\
Homomorfismo trivial $f(x)=e_2 \quad \forall x\in G_1$.


\textbf{Definición: }\\
Un homomorfismo inyectivo se llama monomorfismo.\\
Un homomorfismo suprayectivo se llama epimorfismo.\\
Un homomorfismo biyectivo se llama isomorfismo.\\
Si existe un isomorfismo entre $G_1$ y $G_2$ decimos que son isomorfos y escribimos $G_1\simeq G_2$.

\textbf{Proposición:}\\
Ser isomorfo es relación de equivalencia.\\
\textit{Dem:}\\
Propiedad reflexiva: $\begin{array}{cccc}
    1_G\::&G&\longrightarrow&G\\
        &g&\longmapsto&g
\end{array}$ es isomorfismo.\\
Propiedad simétrica: Si existe $\varphi$ isomorfismo $\begin{array}{cccc}
    \varphi\::&G_1&\longrightarrow&G_2\\
        &g&\longmapsto&\varphi(g)
\end{array}$ Por ser biyectiva existe $\tilde{\varphi}$ bien definida $\begin{array}{cccc}
    \tilde{\varphi}\::&G_2&\longrightarrow&G_1\\
        &\varphi(g)&\longmapsto&g
\end{array}$ Veamos que $\tilde{\varphi}$ es isomorfismo.
$$
\tilde{\varphi}(\underbrace{x}_{=\varphi(g)}*\underbrace{y}_{=\varphi(h)})=\tilde{\varphi}(\varphi(g)*\varphi(h))=\tilde{\varphi}(\varphi(g\cdot h))=g\cdot h = \tilde{\varphi}(\varphi(g))\cdot\tilde{\varphi}(\varphi(h))=\tilde{\varphi}(x)\cdot\tilde{\varphi}(y)
$$
Propiedad transitiva: Sabemos que la composición de funciones biyectivas es biyectiva. Veamos que la composición de homomorfismos es homomorfismo. Sean $(G_1, \overset{1}{*})$, $(G_2, \overset{2}{*})$ y $(G_3, \overset{3}{*})$ grupos. Si $f$ y $g$ son homomorfismos, entonces $g\circ f$ es homomorfismo.
$$
\begin{array}{cccccc}
    G_1 & \overset{g}{\longrightarrow} & G_2 & \overset{f}{\longrightarrow} & G_3\\
    g_1 &\longmapsto&g_2&\longmapsto&g_3
     & 
\end{array}
$$
$$
f(g(x\overset{1}{*}y))=f(g(x)\overset{2}{*}g(y))=f(g(x))\overset{3}{*}f(g(y))
$$

\textbf{Definición:}\\
Si $(G_1,\cdot)=(G_2,*)=G$ se llama endomorfismo, y si es biyectivo automorfismo.


\textbf{Elemplo:}\\
Automorfismo identidad $1_G(x)=x \quad \forall x\in G$.


\textbf{Proposición:}\\
Si  $f\::\:G_1\longrightarrow G_2$ es un homomorfismo de de grupos, entonces se verifican\\
$$
\begin{array}{ll}
\text{\textit{i)}} & \text{$f(e_1)=e_2$}\\
\text{\textit{ii)}} & \text{$f(a^{-1})=f(a)^{-1}$}\\
\text{\textit{iii)}} & \text{Si $o(a)=n<\infty$, entonces $f(a)^n=e_2$}\\
\text{\textit{iv}} & \text{Si $H_1\le G_1$, entonces $f(H_1)\le G_2$}\\
\text{\textit{v)}} & \text{Si $H_2\le G_2$, entonces $f^1(H_2)\le G_1$}\\
\end{array}
$$
\textit{Dem:}\\
\textit{i)} $f(e_1)=f(e_1\cdot e_1)=f(e_1)*f(e_1)$,  entonces $f(e_1)=f(e_1)*f(e_1) \Leftrightarrow f(e_1)^{-1} * f(e_1)= f(e_1)^{-1}*f(e_1)*f(e_1) \Leftrightarrow e_2=e_2*f(e_1)=f(e_1)$.\\ \textit{ii)} Veamos que $f(a)*f(a^-1)=e_2=f(a^{-1})*f(a)$.Aplicando \textit{i)} $f(a)*f(a^{-1})=f(a\cdot a^{-1})=f(e_1)=e_2=f(e_1)=f(a^{-1}\cdot a)=f(a^{-1})*f(a)$.\\
\textit{iii)} $f(a^n)=f(a)*\overset{n}{\cdots}*f(a)=f(a)^n$,entonces $e_2=f(e_1)=f(a^{o(a)})=f(a)^{o(a)}$\\
\textit{iv)}
$$
\begin{array}{rcl}
e_2\in f(H_1) && f(\underbrace{e_1}_{\in H_1})=e_2 \\
f(y)\in f(H_1)\Rightarrow f(y)^{-1}\in f(H_1) && f(y)^{-1}=f(\underbrace{y^{-1}}_{\in H_1})\in f(H_1)
\end{array}
$$
\textit{v)}
$$
\begin{array}{rcl}
e_1\in f^{-1}(H_2) && f^{-1}(\underbrace{e_2}_{\in H_2})\ni e_1 \\
y\in f^{-1}(H_2)\Rightarrow y^{-1}\in f^{-1}(H_1) && f(y)\in H_2 \Rightarrow f(y)^{-1}\in H_2\Rightarrow f(y^{-1})\in H_2 \Rightarrow y^{-1}\in f^{-1}(H_2)
\end{array}
$$

\textbf{Definición:}\\
Dado un homorfismo $f$ de $(G_1,\cdot)$ en $(G_2,*)$. Llamamos núcleo de $f$ a $Ker(f)=\{ g\in G_1 \mid f(g)=e_2 \}$.

\textbf{Propocición:}\\
Dado un homorfismo $f$ de $(G_1,\cdot)$ en $(G_2,*)$. $Ker(f)\trianglelefteq G_1$\\
\textit{Dem:}\\
Primero veamos que es subgrupo. $e_1\in Ker(f)$ ya que $f(e_1)=e_2$. Si $x\in Ker(f)$, entonces $x^{-1}\in Ker(f)$ ya que $e_2=f(e_1)=f(x^{-1} x)=f(x^{-1})*f(x)=f(x^{-1})*e_2=f(x^{-1})$. Ahora veamos que es subgrupo normal probando que $x\in Ker(f) \Rightarrow y^{-1}xy\in Ker(f)$. Observamos que $f(y^{-1}xy)=f(y^{-1})*f(x)*f(y)=f(y)^{-1}*e-2*f(y)=e_2$.

\textbf{Proposición:}\\
Dado un homorfismo $f$ de $(G_1,\cdot)$ en $(G_2,*)$. $f$ es inyectivo sii $Ker(f)=\{e_1\}$.\\
\textit{Dem:}\\
Si $f$ es inyectiva $f(a)=e_2=f(e_1)$, entonces $a=e_1$. Si $f(a)=f(b)$, entonces $e_2=f(a)*f(b)^{-1}=f(a)*f(b^{-1})=f(ab^{-1})\in Ker(f)=\{e_1\}$, ahora $ab^{-1}=e_1$ y $a=b$

\textbf{Primer teorema de isomorfía de grupos:}\\
Dado un homorfismo $f$ de $G_1,\cdot)$ en $(G_2,*)$. $G_1/Ker(f)\simeq f(G_1)$.\\
\textit{Dem:}\\
Consideramos la siguiente aplicación
$$
\begin{array}{cccc}
    \bar{f}\::&G_1/Ker(f)&\longrightarrow&f(G_1)\\
        &aKer(f)&\longmapsto&f(a)
\end{array}
$$
Veamos que es isomorfismo.Primero que es suprayectiva pues todo elemento de $f(G_1)$ es de la forma $f(x)$ con $x\in G_1$. Veamos que es inyectiva, si $aKer(f)\in Ker(\bar{f})$ entonces $f(a)=e_2$ por lo que $a\in Ker(f)$ y $aKer(f)=Ker(f)$. Veamos que es homomorfismo $\bar{f}(aKer(f)\cdot bKer(f))=\bar{f}((ab)Ker(f))=f(ab)=f(a)*f(b)=\bar{f}(aKer(f))*\bar{f}(bKer(f))$.

\textbf{Definición:}\\
Sea $G$ un grupo y $N\trianglelefteq G$. Definimos el epimorfismo canónico
$$
\begin{array}{cccc}
    \pi\::&G&\longrightarrow&G/N\\
        &g&\longmapsto&gN
\end{array}
$$
Es epimorfismo ya que todo elemento de $G/N$ es de la forma $xN$ y por lo tanto es imagen de $x$. Por otra parte, $\pi(xy)=(xy)N=(xN)(yN)=\pi(x)\pi(y)$.


\textbf{Proposición:} Si $f\::\:G\longrightarrow H$ es homomorfismo, entonces $Ker(f)\trianglelefteq G$ y $
\begin{array}{cccc}
    \pi\::&G&\longrightarrow&g/Ker(f)\\
        &a&\longmapsto&aKer(f)
\end{array}
$ es un epimorfismo con $Ker(\pi)=Ker(f)$.\\
\textit{Dem:}\\
$Ker(f)\trianglelefteq G$ ya está probado, y $x\in Ker(\pi)$ entonces $xKer(f)=Ker(f)$ y $x\in Ker(f)$. Ahora, si $x\in Ker(f)$, entonces $\pi(x)=xKer(f)=Ker(f)$ y $x \in Ker(\pi)$.

\textbf{Definición:}\\
Dado un homorfismo $f$ de $G_1,\cdot)$ en $(G_2,*)$.Se tiene la descomposición canonica del homomorfismo $$f=\underbrace{\iota}_\text{monomorfismo inclusión} \circ \underbrace{\bar{f}}_\text{isomorfismo} \circ \underbrace{\pi}_\text{epimorfismo canónico}$$
$$
\overset{f}{\overrightarrow{\begin{array}{cccccccccccc}
    G_1&\overset{\pi}{\longrightarrow}&G_1/Ker(f)&\overset{\bar{f}}{\longrightarrow}&f(G_1)&\overset{\iota}{\longrightarrow}&G_2\\
    a&\longmapsto&aKer(f)&\longmapsto&f(a)&\longmapsto&f(a)
\end{array}}}
$$

\textbf{Teorema:} Sea $N\trianglelefteq G$. Entonces todo subgrupo de $G/N$ es de la forma $H/N$ con $N\subseteq H \le G$.\\
\textit{Dem:}\\
Suponganmos que $K\le G/N$, Sea $H=\{ x\in G \mid xN\in K \}$. Veamos que $H$ es subgrupo. Tenemos que $e\in H$ ya que $e_1N=N$ que es el elemento neutro en $G/N$ y por lo tanto esta en $K$. Veamos que si $y\in H$, entonces $y^{-1}\in H$. Tenemos que $yN\in K$ entonces por ser $N$ subgrupo normal $K\ni(yN)^{-1}=N^{-1}y^{-1}=Ny^{-1}=y^{-1}N$ por lo que $y^{-1}\in H$. Para ver que $N\subseteq H$, si $n\in N$, entonces $nN=N\in K$ por ser el elemento neutro y $n\in H$.

\textbf{Teorema:} Sea $N\trianglelefteq G$. Entonces si $N\le H\le G$ entonces $H/N\le G/N$.\\
\textit{Dem:}\\
Veamos que es subgrupo. Si $x,y\in H$, $xN,yN\in H/N$ entonces $xy\in H$ y $(xy)N\in H/N$. Si $e\in H$, entonces $eN=N$. Si $x\in H$, $xN\in H/N$, entonces $x^{-1}\in H$ y $x^{-1}N\in H/N$, $(xN)(x^{-1}N)=(Nx)(x^{-1})=N$.

\textbf{Segundo teorema de isomorfía de grupos:}\\
Sea $G$ un grupo, $N\trianglelefteq G$ y $H\le G$, entonces
$$
\begin{array}{ll}
    \text{\textit{i)}} & \text{$N\trianglelefteq NH$} \\
    \text{\textit{ii)}} &  \text{$N\cap H \trianglelefteq H$}\\
    \text{\textit{iii)}} & \text{$NH/N\simeq H/(N\cap H)$}
\end{array}
$$
\textit{Dem:}\\
\textit{i)} Veamos que $(nh)^{-1}N(nh)=N$. $(nh)^{-1}N(nh)=h^{-1}n^{-1}Nnh=h^{-1}Nh=N$.\\
\textit{ii)} Veamos que si $h \in H$, entonces $h^{-1}(N\cap H)h=N\cap H$. $(h^{-1}Nh)\cap(h^{-1}Hh)=N\cap H$.\\
\textit{iii)} La aplicación $
\begin{array}{cccc}
    \bar{f}\::&NH/N&\longrightarrow&H/(N\cap H)\\
        &hnN&\longmapsto&h(N\cap H)
\end{array}
$ es un isomorfismo de grupos. Está bien definida ya que si $hN=h'N$ entonces $h^{-1}h'\in N,H$ y $h(N\cap H)=h'(N\cap H)$. Es homomorfismo ya que $f((hN)(h'N))=f((hh')N) = (hh')(N\cap H)=h(N\cap H)h'(N\cap H))=f(hN)f(h'N)$. Es inyectiva ya que si $x\in Ker(f)$, entonces $x\in (N\cap H)$ y $xN=N$, es decir, $Ker(f)=\{N\}$. Es suprayectiva ya que dado un $h\in H$ existe $hN$ tal que $f(hN)=h(N\cap H)$.

\textbf{Tercer teorema de isomorfia de grupos:}\\
Sea $G$ un grupo y $N\subseteq M$ dos subgrupos normales de $G$. Entonces se verifican
$$
\begin{array}{ll}
    \text{\textit{i)}} & M/N \trianglelefteq G/N \\
    \text{\textit{ii)}} & (G/N)/(M/N)\simeq G/M
\end{array}
$$
\textit{Dem:}\\
Sabemos que $M/N$ es subgrupo.
$$
(gN)^{-1}(mN)(gN)=(g^{-1}mg)N=mN
$$
Consideramos la aplicación $\begin{array}{cccc}
    f\::&G/N&\longrightarrow&G/M\\
        &xN&\longmapsto&xM
\end{array}$, esta bien definida ya que si $xN=yN$, entonces $x^{-1}y\in N\subseteq M$ y $xM=yM$. Es homomorfismo ya que $f((xN)(yN)=f((xy)N)=(xy)M=f(xN)f(yN)$. Es suprayectiva ya que para todo $xM\in G/M$ existe $xN\in G/N$ tal que $f(xN)=xM$. Calculamos $ker(f)$, si $N=f(xM)=xN$ se tiene que $x\in N$ por lo que
$$
Ker(f)=\{xM\mid x\in N\}=M/N
$$
Por el primer teorema de isomorfía, aplicado a $f$,
$$
(G/N)/Ker(f)=(G/N)/(M/N)\simeq f(G/N)=G/M
$$
$$
(G/N)/(M/N)\simeq G/M
$$
\end{document}
