\documentclass{article}
\usepackage[utf8]{inputenc}
\usepackage{graphicx}
\usepackage[spanish]{babel}
\usepackage{amssymb,amsmath,geometry}
\usepackage{etoolbox} %titulo
\makeatletter %titulo
\patchcmd{\@maketitle}{\vskip 2em}{\vspace*{-3cm}}{}{} %titulo
\makeatother %titulo
\usepackage{vmargin}
\setpapersize{A4}
\setmargins{2.5cm}       % margen izquierdo
{1.5cm}                        % margen superior
{16.5cm}                      % anchura del texto
{23.42cm}                    % altura del texto
{10pt}                           % altura de los encabezados
{1cm}                           % espacio entre el texto y los encabezados
{0pt}                             % altura del pie de página
{2cm}                           % espacio entre el texto y el pie de página
\title{Estructuras Algebraicas}
\author{Andoni Latorre Galarraga}
\date{}
\newcommand{\bb}[1]{\mathbb{#1}}
\newcommand{\fun}[5]{\begin{array}{lllll}
    #1: & #2 & \longrightarrow & #3  \\
        & #4 & \longmapsto     & #5 
\end{array}}
\newcommand{\p}[0]{\textbf{Proposición:}}
\newcommand{\dem}[0]{\textit{Dem:}}
\begin{document}

\maketitle

\section{Grupos}
\setlength{\parindent}{0cm}

%GRUPO

\textbf{Definición:} Un grupo $(G,*)$ son un conjunto no vacío $G$ y una operación $\begin{array}{lllll}
    *: & G\times G & \longrightarrow & G  \\
       & (a,b)     & \longmapsto     & a*b\in G 
\end{array}$ tales que\\
$$
\begin{array}{l}
    \text{i)} \quad (a*b)*c=a*(b*c) \quad \forall a,b,c\in G \\
    \text{ii)} \quad \exists e : \quad e*x=x=x*e \quad \forall x \in G\quad\text{se dice que }e\text{ es el elemento neutro}\\
    \text{iii)} \quad \forall a\in G \quad \exists a^\prime \in G:\quad a*a^\prime=e=a^\prime*a \quad \text{se dice que }a^\prime\text{ es el simétrrico de }a\text{.}
\end{array}
$$
Con $*=+$ al simétrico se le llama opuesto. Con $*=\cdot$ al simétrico se le llama inverso.\\\\

%INVERSO Y NEUTRO

\textbf{Proposición:} El elemento neutro y el simétrico son únicos.\\
\textit{Dem:} Supongamos que $e_1,e_2$ son elementos neutros. Por ser $e_1$ elemento neutro $e_2=e_2*e_1$ y por ser $e_2$ elemento neutro $e_2=e_2*e_1=e_1$, se tiene que el elemento neutro es único. Supongamos que $a^\prime,a^{\prime\prime}$ son simetricos de $a$. Por definición de simétrico $a*a'=e=a*a''$, operando con $a'$ por la izquierda $a'(a*a')=a'(a*a'')$. Por la propiedad asociativa, $(a'*a)*a'=(a'*a)*a''$. Por definición de simétrico $e*a'=e*a''$. Por definición de elemento neutro $a'=a''$.\\\\

%TABLA

\textbf{Definición:} Si $G$ es un grupo finito, $G=\{g_1,\cdots,g_n\}$, la tabla de grupo $G$ es:
$$
\begin{array}{|c|cccccc|}
    \cdot   & g_1    & g_2    & \cdots &   g_j        & \cdots & g_n \\\hline
     g_1    &        &        &        & \vdots       &        &     \\ 
     g_2    &        &        &        & \vdots       &        &     \\
     \vdots &        &        &        & \vdots       &        &     \\
     g_i    & \cdots & \cdots & \cdots & g_i\cdot g_j &        &     \\
     \vdots &        &        &        &              &        &     \\
     g_n    &        &        &        &              &        &     \\\hline
\end{array}
$$

%POTENCIA

\textbf{Definición:} En $(G,*)$
$$
\forall a \in G \quad
\left\{\begin{array}{ll}
    a^0=e\\
    a^n=\overbrace{a*\cdots*a}^n & \text{si }n\ge 1 \\
    a^n=\overbrace{a^{-1}*\cdots*a^{-1}}^n & \text{si }n< 0
\end{array}\right. \quad \forall n \in \mathbb{Z}
$$\\

%SUBGRUPO

\textbf{Definición:} Si $\emptyset \ne H\subsetneq G$ decimos que $H$ es subgrupo de $(G,*)$ si $(H,*_{|H\times H})$ es grupo. Se escribe $H \le G$.\\\\
%ORDEN
\textbf{Definición:} Si $\exists m\ge 0$ tal que $g^m=1$ llamamos orden de $g\in G$ al menor entero positivo $n$ tal que $g^n=e$ y escribimos $o(g)=n$.\\\\

%SUBGRUPO CICLICO

\textbf{Definición:} Si $a\in G$ llamamos subgrupo cíclico generado por $a$ a
$$
<a>=\{a^n\:|\:n\in\mathbb{Z}\}
$$
Veamos que $<a>$ es subgrupo. Cumple $i)$ ya que se cumple en en $G$. Para $ii)$ tenemos que $a^0=e$ y para $iii)$ tenemos que $a^{-n}*a^n=e=a^n*a^{-n}$.
$$
\overbrace{a^{-1}*\cdots*a^{-1}}^n*\overbrace{a*\cdots*a}^n=e=\overbrace{a*\cdots*a}^n*\overbrace{a^{-1}*\cdots*a^{-1}}^n
$$\\

%ORDEN=CICLICO

\textbf{Proposición:} $|<g>|=o(g)$\\
\textit{Dem:} Sabemos que $\forall n\in\mathbb{N}$ existen $q,r\in\mathbb{N}$ con $0\le r < o(g)$ tales que $n=q\dot o(g)+r$. Entonces
$$
<g>)=\{1,g,g^2,\cdots,g^{o(g)-1}, g^{o(g)}, g^{o(g)+1}, \cdots, g^{o(g)+(o(g)-1)},g^{2o(g)},g^{2o(g)+1},\cdots\}
$$
$$
=\{1,g,g^2,\cdots,g^{o(g)-1}, 1, 1\cdot g, \cdots, 1\cdot g^{o(g)-1}, 1^2,1^2 g,\cdots\}=
\{1,g,\cdots,g^{o(g)-1}\}
$$
Se tiene que $|<g>|=1+o(g)-1=o(g)$.\\\\

%SUBGRUPO GENERADO

\textbf{Definición:} Si $S\subset G$ llamamos subgrupo generado por el conjunto $S$ a
$$
<S>=\{s_{i_1}^{t_1}\cdots s_{i_n}^{t_n}\:|\: s_{i_j}\in S, t_j \in \{1,-1\}, n\in\mathbb{Z}\}
$$
Veamos que $<S>$ es subgrupo. Cumple $i)$ ya que se cumple en en $G$. Para $ii)$ tenemos que $s^1s^{-1}=e$ y para $iii)$ tenemos que $(s_{i_1}^{t_1}\cdots s_{i_n}^{t_n}) \cdot (s_{i_n}^{-t_n}\cdots s_{i_1}^{-t_1})=e=(s_{i_n}^{-t_n}\cdots s_{i_1}^{-t_1}) \cdot (s_{i_1}^{t_1}\cdots s_{i_n}^{t_n})$.\\\\
\textbf{Observación:} $<a>=<\{a\}>$\\\\

%COCLASE IZQUIERDA

\textbf{Definición:} Si $(G,\cdot)$ es grupo y $H\le G$ definimos la coclase por la izquierda $gH=\{gh\:|\:h\in H\}$.\\\\

%hH=H

\textbf{Lema:} $hH=H$. Como $h^{-1}\tilde{h}\in H$ se tiene que $h\underbrace{h^{-1}\tilde{h}}_{\in H}=\tilde{h}$\\\\

%COCLASE IZQUIERDA PARTICION

\textbf{Proposición:} $\mathcal{P}=\{gH=\{gh\:|\:h\in H\}\:|\:g\in G\}$ es una partición de $G$.\\
\textit{Dem:} Evidentemente $\displaystyle\bigcup_{g\in G} gH = G$ ya que $G\supset\displaystyle\bigcup_{g\in G} gH \supset\displaystyle\bigcup_{g\in G} g\underbrace{e}_{\in H}=\displaystyle\bigcup_{g\in G} g = G$. Ahora, veamos que si $g_1H\cap g_2H\ne\emptyset$, entonces $g_1H=g_2H$. Sabemos que existen $h_1,h_2 \in H$ tales que $g_1h_1=g_2h_2$, entonces $g_1=g_2h$ con $h=h_2h_1^{-1}\in H$. Ahora, $g_1H=(g_2h)H=g_2(hH)=g_2H$.\\\\

%CARDINAL COCLASE IZQUIERDA

\textbf{Proposición} $|gH|=|H|$\\
\textit{Dem:} Veamos que existe una biyección entre $gH$ y $H$.
$$
\begin{array}{lllll}
    \varphi & H & \longrightarrow & gH \\
         & h & \longmapsto     & gh
\end{array}
$$
Tenemos que es suprayectiva ya que $\varphi(H)=\{gh\:|\:h\in H\}=gH$. Es inyectiva ya que si $h_1=h_2$, entonces $gh_1=gh_2$.\\\\

%COCLASE DERECHA

\textbf{Definición:} Si $(G,\cdot)$ es grupo y $H\le G$ definimos la coclase por la derecha $Hg=\{hg\:|\:h\in H\}$.\\\\

%Hh=H

\textbf{Lema:} $Hh=H$. Como $\tilde{h}h^{-1}\in H$ se tiene que $\underbrace{\tilde{h}h^{-1}}_{\in H}h=\tilde{h}$\\\\

%COCLASE DERECHA PARTICION

\textbf{Proposición:} $\mathcal{P}=\{Hg=\{hg\:|\:h\in H\}\:|\:g\in G\}$ es una partición de $G$.\\
\textit{Dem:} Evidentemente $\displaystyle\bigcup_{g\in G} Hg = G$ ya que $G\supset\displaystyle\bigcup_{g\in G} Hg \supset\displaystyle\bigcup_{g\in G} \underbrace{e}_{\in H}g=\displaystyle\bigcup_{g\in G} g = G$. Ahora, veamos que si $Hg_1\cap Hg_2\ne\emptyset$, entonces $Hg_1=Hg_2$. Sabemos que existen $h_1,h_2 \in H$ tales que $h_1g_1=h_2g_2$, entonces $g_1=hg_2$ con $h=h_1^{-1}h_2\in H$. Ahora, $Hg_1=H(hg_2)=(Hh)g_2=Hg_2$.\\\\

%CARDINAL COCLASE DERECHA

\textbf{Proposición} $|Hg|=|H|$\\
\textit{Dem:} Veamos que existe una bijección entre $gH$ y $H$.
$$
\begin{array}{lllll}
    \varphi & H & \longrightarrow & Hg \\
         & h & \longmapsto     & hg
\end{array}
$$
Tenemos que es suprayectiva ya que $\varphi(H)=\{hg\:|\:h\in H\}=Hg$. Es inyectiva ya que si $h_1=h_2$, entonces $h_1g=h_2g$.\\\\

%INDICE

\textbf{Definición:} Como $G=\displaystyle\bigcup_{g\in I}^\cdot gH=\displaystyle\bigcup_{g\in \tilde{I}}^\cdot Hg$ y $|Hg|=|H|=|gH|$ se tiene que $|H|\mid|G|$. Tiene sentido definir $|G|=\underbrace{|I|}_{|G:H|}|H|=\underbrace{|\tilde{I}|}_{|G:H|}|H|$ y se dice que $|G:H|$ es el índice de $H$ en $G$.\\\\

%INDICE DIVIDE A G

\textbf{Obsevación:} $|G:H|\mid|G|,|H|\mid|G|\quad\forall H\le G$\\\\

%ORDEN DIVIDE A G

\textbf{Proposición:} $o(g)\mid|G|\quad\forall g\in G$\\
\textit{Dem:} Tenemos que $o(g)=|<g>|$ y como $<g>$ es subgrupo, por la observación anterior $|<g>|\mid|G|$, entonces $o(g)\mid|G|$.\\\\

%COCLASE IZQUIERDA-DERECHA

\textbf{Definición:} Para cada $x\in G$ si $H,K\le G$ definimos $HxK=\{hxk\mid h\in H, k\in K\}$.\\\\

%RELACION EQUIVALENCIA IZQUIERDA_DERECHA

\textbf{Proposición:} $x\sim y \Leftrightarrow \exists h\in H, \exists k\in K \mid x=hyk$ es una relación de equivalencia.\\
\textit{Dem:} Reflexividad, $e\in H, e\in K$ y $x=exe$. Simetría, si $\exists h\in H, \exists k\in K \mid x=hyk$, entonces $h^{-1}\in H, k^{-1}\in K$ y $y=h^{-1}xk^{-1}$. Transitividad, si $x=h_1yk_1, y=h_2zk_2$, entonces $x=\underbrace{h_1h_2}_{\in H}z\underbrace{k_2k_1}_{\in K}$.\\\\

%PARTICION IZQUIERDA DERECHA

\textbf{Corolario:} $\mathcal{P}=\{HxK\mid x \in G\}$ es una partición de $G$.\\\\

%CONJUGADO

\textbf{Definición:} Llamamos conjugado a $H^g=g^{-1}Hg$ con $g\in G$ y también $x^g=g^{-1}xg$ para $x,g\in G$.\\\\

%CONJUGADO ES SUBGRUPO

\textbf{Proposición:} Si $H$ es subgrupo de $G$, entonces $H^x$ es subgrupo de $G$.\\
\textit{Dem:} La propiedad asociativa se cumple ya que se cumple enn todo $G$. Veamos que el elemento neutro esta en $H^x$, $e\in H$ por ser $H$ subgrupo, entonces $e=x^{-1}x=x^{-1}ex\in H^x$. Veamos que existe el inverso, $(x^{-1}hx)^-1=x^{-1}\underbrace{h}_{\in H}^{-1}x\in H^x$\\\\

%SUBGRUPO INTERSECCION

\textbf{Proposición:} Si $H,K\le G$ entonces $H\cap K\le G$\\
\textit{Dem:} La propiedad asociativa se cumple ya que se cumple en todo $G$. El eleemto neutro está en la intersección ya que esta en ambos subgrupos. Veamos que existe inverso en la intersección, si $x\in H\cap K$, entonces $x\in H,x\in K$ y $x^{-1}\in H, x^{-1}\in K \Rightarrow x^{-1}\in H\cap K$.\\\\

%PARTICION Hxk

\textbf{Proposición:} $\mathcal{P}=\{Hxk\mid k\in K\}$ es una partición de $HxK$.\\\\
\textit{Dem:} Por la definicón de coclase a la derecha.
$$
\bigcup_{k\in K} Hxk = \bigcup_{k\in K} \{ hxk\mid h\in H\} = \{ hxk\mid h\in H,k\in K\} = HxK
$$
Ahora, veamos que si $Hxk_1\cap Hxk_2 \ne \emptyset$ entonces $Hxk_1=Hxk_2$. Sabemos que existen $h_1,h_2$ tales que $h_1xk_1=h_2xk_2$. Entonces, $xk_1=\underbrace{h_1^{-1}h_2}_{=h\in H}xk_2$ y $Hxk_1=Hhxk_2=Hxk_2$.También sabemos que $|Hxk|=|H|$. Veamos cuantos elementos tiene la partición. Si $k_1,k_2\in K$,
$$
Hxk_1=Hxk_2 \Leftrightarrow
Hx(k_1k_2^{-1})=Hx \Leftrightarrow
Hx(k_1k_2^{-1})x^{-1}=H
$$
$$
\Leftrightarrow
x(k_1k_2^{-1})x^{-1}\in H \Leftrightarrow
x^{-1}\underbrace{x(k_1k_2^{-1})x^{-1}}_{\in H}x\in x^{-1}Hx=H^x
$$
$$
\Leftrightarrow
k_1k_2^{-1}\in H^x
\Leftrightarrow
k_1k_2^{-1}\in H^x\cap K
\Leftrightarrow
(H^x\cap K)k_1=(H^x\cap K)k_2
$$
Por lo tanto $|\mathcal{P}|=|K:H^x\cap K|$. Entonces $|HxK|=|H||K:H^x\cap K|$.\\\\

%G Y SUBGRUPOS IZQUIERDA DERECHA

\textbf{Proposición:} $\displaystyle |G|=\sum_{i\in I} |H||K:H^x\cap K|$.\\
\textit{Dem:} $\mathcal{P}=\{HxK\mid x\in G\}=\{Hx_iK\mid i\in I\}$ entonces,
$$
|G| = \sum_{i\in I} |Hx_iK| = \sum_{i\in I} |H||K:H^{x_i}\cap K|
$$\\

%INDICE Y SUBGRUPOS IZQUIERDA DERECHA

\textbf{Corolario:} $\displaystyle |G:H|=\sum_{i\in I}|K:H^x\cap K|$.\\\\

%ST

\textbf{Definición:} Sean $S,T\ne\emptyset$, entonces $ST=\{st\mid s\in S,t\in T\}$\\\\\

%SS=S

\textbf{Obeservación:} $SS=S$\\\\

%PUNTO CUESTIONES :) (AHORA TENGO UN 10 EN TEORIA JAJAJA)

\textbf{Proposición:} Sean $H,K\le G$, entonces $HK\le G\Leftrightarrow HK=KH$\\
\textit{Dem:} $\Rightarrow$: $HK\ni hk = ((hk)^{-1})^{-1}=(\underbrace{k^{-1}h^{-1}}_{\in KH})^{-1}\in KH$, entonces $HK=KH$\\
$\Leftarrow$: Se cumple la asociativa ya que se cumpre en todo $G$. El elemento neutro esta en $HK$ ya que $e\in H,e\in K$ por ser $H,K$ subgrupos, entonces $HK\ni ee=e$. El inverso esta en $HK$, $(hk)^{-1}=k^{-1}h^{-1}\in KH=HK$\\\\

%SUBGRUPO NORMAL

\textbf{Definición:} Se dice que $N\le G$ es normal en $G$ si $gN=Ng\quad \forall g\in G$, equivalentemente $gNg^{-1}=N\quad\forall G \in G$. Se escribe $N\trianglelefteq G$ cuando $N$ es normal en $G$ y $N\not\trianglelefteq G$ cuando $N$ no es normal en $G$.\\\\

%NH SUBGRUPO

\textbf{Proposición:} Si $N\trianglelefteq G$ y $H\le G$, entonces $NH\le G$\\
\textit{Dem:} Sabemos que $NH\le G \Leftrightarrow NH=HN$. Por ser $N$ subgrupo normal, $NH=HN$ ya que $NH=\bigcup\{Nh\mid h\in H\}=\bigcup\{hN\mid h\in H\}=HN$.\\\\

%GRUPO COCIENTE

\textbf{Definición:} Sea $N\trianglelefteq G$ entonces definimos el grupo cociente $G/N=\{gN\mid g\in G\}=\{Ng\mid g\in G\}$.\\
Veamos que $(G/N,\cdot)$ es grupo. Para la propiedad asociativa, sean $[a],[b],[c]\in G/N$
$$
([a][b])[c]=\{(\alpha\beta)\gamma\mid \alpha\in[a],\beta\in[b],\gamma\in[c]\}=\{\alpha(\beta\gamma)\mid \alpha\in[a],\beta\in[b],\gamma\in[c]\}=[a]([b][c])
$$
Para el elemento neutro, $e_{G/N}=\underbrace{e_G}_{\in N}N=N$,
$$
N[a]=\{n\alpha\mid n\in N,\alpha\in[a]\}\{n\alpha\mid n\in N,\alpha\in Na\} = nNa=Na=[a]
$$
$$
[a]N=\{\alpha n\mid n\in N,\alpha\in[a]\}\{\alpha n\mid n\in N,\alpha\in aN\} = aNn=aN=[a]
$$
Para el inverso de $[a]$, tenemos que $[a]^{-1}=[a^{-1}]$,
$$
[a][a^{-1}]=aNNa^{-1}=aNa^{-1}=N=e_{G/N}
$$\\

%INDICE=COCIENTE

\textbf{Proposición:} $|G/N|=|G:N|$\\
\textit{Dem:} $|G/N|$ es el número de coclases que es $|G:N|$.\\\\

%NORMALIZADOR

\textbf{Definición:} Dado un subgrupo $H$ de $G$ definimos $N_G(H)=\{g\in G\mid H^g=H\}$ y lo llamamos subgrupo normalizador de $H$ en $G$. Veamos que es subgrupo. La asociativa se cumple ya que se cumple en todo $G$. Veamos que $e\in N_G(H)$, $H^e=e^{-1}He=eHe=H$. Veamos que si $a\in N_G(H)$ entonces $a^{-1}\in N_G(H)$, por ser H subgrupo, $H^{a^{-1}}=(H^{a^{-1}})^{-1}=(aHa^{-1})^{-1}=a^{-1}H^{-1}a=a^{-1}Ha=H$.\\\\

%GRUPO ABELIANO

\textbf{Definición:} Se dice que $G$ es abeliano si la L.C.I. es conmutativa, es decir, $ab=ba\quad\forall a,b\in G$\\\\

%TODO SUBGRUPO ES NORMAL EN UN ABELIANO

\textbf{Proposición:} Si $G$ es abeliano y $H\le G$, entonces $H\trianglelefteq G$.\\
\textit{Dem:} $gH=\{gh\mid h\in H\}=\{hg\mid h\in H\}=Hg$.\\\\

%(Z,+) abeliano

\textbf{Observación:} $(\mathbb{Z},+)$ es abeliano.\\\\

%SUBGRUPOS DE Z

\textbf{Proposición:} La totalidad de los subconjuntos de $(\mathbb{Z},+)$ es $\{0\}\cup\{n\mathbb{Z}\mid n\in\mathbb{N}\}$.\\
\textit{Dem:} $\{0\}$ es el subgrupo trivial, veamos que $n\mathbb{Z}$ es subgrupo. $0=0n\in n\mathbb{Z}$, veamos que suma está bien definida $nz_1+nz_2=n(z_1+z_2)$, claramente si $z$ es múltiplo de $n$, entonces $-z\in n\mathbb{Z}$. Reciprocamente, si $H\le\mathbb{Z}$ supongamos que $H$ es diferente del trivial, entonces $a\in H$ para algún $a\ne0$. Además $-a\in H$ por lo que $H$ tiene enteros positivos. Ahora, sea $n$ el menor entero positivo en $H$, supongamos que $x\in H$, ahora $x=zn+r$ con $0\le r<n$ si $r\ne 0$ hay contradicción con que $n$ es el menor entero positivo, por lo tato, $x$ es múltipolo de $n$ y $H=n\mathbb{Z}$.\\\\

%SUBGRUPOS DE nZ

\textbf{Proposición:} $n\mathbb{Z}\le m\mathbb{Z}\Leftrightarrow m|n$.\\
\textit{Dem:} $\Rightarrow$) Si $n\mathbb{Z}\le m\mathbb{Z}$, entonces $n\in m\mathbb{Z}$ por lo que $n$ es múltiplo de m y $m\mid n$.\\
$\Leftarrow$) Si $m\mid n$, entonces existe $z\in\mathbb{Z}$ tal que $n=zm$ por lo que $n$ es múltiplo de $m$ y se tiene que $n\in m\mathbb{Z}$.\\\\

%SUMA DE nZ

\textbf{Proposición:} $\forall m,n\in\mathbb{Z}, n\mathbb{Z}+m\mathbb{Z}=a\mathbb{Z}$ donde $a=m.c.d.(n,m)$.\\
\textit{Dem:} $n\mathbb{Z}+m\mathbb{Z}$ es subgrupo ya que $n\mathbb{Z}+m\mathbb{Z}=m\mathbb{Z}+n\mathbb{Z}$ por ser $\mathbb{Z}$ es abeliano. Tenemos que $n,m\in n\mathbb{Z}+m\mathbb{Z}$, por lo que $a\mid m$ y $a\mid n$, entonces $a\mid a'=m.c.d.(m,n)$. Por otra parte, existen $z_1,z_2\in\mathbb{Z}$ tales que $n=z_1a'$ y $m=z_2a'$, entonces $a\mathbb{Z}=n\mathbb{Z}+m\mathbb{Z}=z_1a'\mathbb{Z}+z_2a'\mathbb{Z}\subseteq a'\mathbb{Z}$ por lo que $a'\mid a$ y se tiene que $a'=a$.\\\\

%INTERSECCION nZ

\textbf{Proposición:} $\forall m,n\in\mathbb{Z}, n\mathbb{Z}\cap m\mathbb{Z}=b\mathbb{Z}$ donde $b=m.c.m.(n,m)$.\\
\textit{Dem:} Sabemos que $nm\in n\mathbb{Z}\cap m\mathbb{Z}$ por lo que $n\mathbb{Z}\cap m\mathbb{Z}$ es no nulo. Además $n\mathbb{Z},m\mathbb{Z}\le b\mathbb{Z}$ por lo que $n,\mid b$ y $b'\mathbb{Z}\subseteq n\mathbb{Z},m\mathbb{Z}$. Ahora, $b'\mathbb{Z}\subseteq n\mathbb{Z}\cap m\mathbb{Z}$, por lo tanto $b\mid b'$. Si $b'=m.c.m.(n,m)$, entonces $b'\mid b$ y se tiene que $b=b'$.\\\\

%Z/nZ ABELIANO DE ORDEN n

\textbf{Proposición:} Para cada $n\in\mathbb{N}$, $\mathbb{Z}/n\mathbb{Z}$ es un grupo abeliano de orden $n$.\\
\textit{Dem:} $\mathbb{Z}/n\mathbb{Z}=\{m+n\mathbb{Z}\mid m\in\mathbb{Z}\}$. Tenemos que $m=nq+r$ con $0\le r<n$,
$$
m+n\mathbb{Z}=(nq+r)+n\mathbb{Z}=(nq+n\mathbb{Z})+(r+n\mathbb{Z})=r+n\mathbb{Z}
$$
por lo que,
$$
\mathbb{Z}/n\mathbb{Z}=\{0+n\mathbb{Z},1+n\mathbb{Z},\cdots,(n-1)+n\mathbb{Z}\}
$$
Veamos que estos elementos son distintos dos a dos, $i+n\mathbb{Z}=j+n\mathbb{Z}\Leftrightarrow i-j\in n\mathbb{Z}$ pero $|i-j|<n$ por lo que $i=j$. Se tiene que $|\mathbb{Z}/n\mathbb{Z}|=n$.\\\\

%IDENTIDAD DE BEZOUT

\textbf{Proposición:} Sean $n,m\in\mathbb{N}$, Entonces existen enteros $r$ y $s$ tales que $1=nr+ms$ si y solo si $m.c.d.(n,m)=1$.\\
\textit{Dem:} Sabemos que $n\mathbb{Z}+n\mathbb{Z}=m.c.d.(n,m)\mathbb{Z}$ y $1\in m.c.d.(n,m)\mathbb{Z}\Leftrightarrow m.c.d=(n,m)=1$. (También es posible $m.c.d=(n,m)=-1$, depende de como se defina $m.c.d.$, independientemente, es lo mismo.)\\\\

%ECUACION DIOFANTICA

\p La ecuación difántica $ax+by=c$ tiene al menos una solución en $\bb{Z}\times\bb{Z}$ si y solo si $m.c.d.(a,b)$ divide a $c$. Si $(x_0,y_0)$ es una solución, entonces el conjunto de todas las soluciones es
$$
\{ (x_0-(d/b)z+(a/d)z)\mid z\in \bb{Z}\}
$$
\dem Sabemos que $n\mathbb{Z}+n\mathbb{Z}=m.c.d.(a,b)\mathbb{Z}$, entonces existe solución si y solo si $m.c.d.(a,b)\mid c$. Ahora, supongamos que $(x_0,y_0)$ es una solución,
$$
ax_0+by_0=c
$$
$$
ax+by=c
$$
Restando, obtenemos
$$
a(x-x_0)=-b(y-y_0)
$$
$$
(\frac{a}{m.c.d.(a,b)})(x-x_0)=-(\frac{b}{m.c.d.(a,b)})(y-y_0)
$$
Y como $m.c.d.(\frac{a}{m.c.d.(a,b)},\frac{b}{m.c.d.(a,b)})=1$, se tiene que $\frac{a}{m.c.d.(a,b)}$ divide a %%
\\\\

%HOMOMORFISMO
\textbf{Definición:} Sean $(G_1,\cdot)$ y $(G_2,*)$ grupos. Decimos que $f\::\:G_1\longrightarrow G_2$ es un homomorfismo de de grupos, si verifica
$$
f(x\cdot y)=f(x)*f(y)\quad\forall x,y\in G_1
$$\\

%EJEMPLO HOMOMORFISMO


\textbf{Ejemplo:} Homomorfismo trivial $f(x)=e_2 \quad \forall x\in G_1$.\\\\

%MONO/EPI/ISO

\textbf{Definición:} Un homomorfismo inyectivo se llama monomorfismo.\\
Un homomorfismo suprayectivo se llama epimorfismo.\\
Un homomorfismo biyectivo se llama isomorfismo.\\
Si existe un isomorfismo entre $G_1$ y $G_2$ decimos que son isomorfos y escribimos $G_1\simeq G_2$.\\\\

%ISOMORFO RELACIÓN DE EQ

\textbf{Proposición:} Ser isomorfo es relación de equivalencia.\\
\textit{Dem:} Propiedad reflexiva: $\begin{array}{cccc}
    1_G\::&G&\longrightarrow&G\\
        &g&\longmapsto&g
\end{array}$ es isomorfismo.\\
Propiedad simétrica: Si existe $\varphi$ isomorfismo $\begin{array}{cccc}
    \varphi\::&G_1&\longrightarrow&G_2\\
        &g&\longmapsto&\varphi(g)
\end{array}$ Por ser biyectiva existe $\tilde{\varphi}$ bien definida $\begin{array}{cccc}
    \tilde{\varphi}\::&G_2&\longrightarrow&G_1\\
        &\varphi(g)&\longmapsto&g
\end{array}$ Veamos que $\tilde{\varphi}$ es isomorfismo.
$$
\tilde{\varphi}(\underbrace{x}_{=\varphi(g)}*\underbrace{y}_{=\varphi(h)})=\tilde{\varphi}(\varphi(g)*\varphi(h))=\tilde{\varphi}(\varphi(g\cdot h))=g\cdot h = \tilde{\varphi}(\varphi(g))\cdot\tilde{\varphi}(\varphi(h))=\tilde{\varphi}(x)\cdot\tilde{\varphi}(y)
$$
Propiedad transitiva: Sabemos que la composición de funciones biyectivas es biyectiva. Veamos que la composición de homomorfismos es homomorfismo. Sean $(G_1, \overset{1}{*})$, $(G_2, \overset{2}{*})$ y $(G_3, \overset{3}{*})$ grupos. Si $f$ y $g$ son homomorfismos, entonces $g\circ f$ es homomorfismo.
$$
\begin{array}{cccccc}
    G_1 & \overset{g}{\longrightarrow} & G_2 & \overset{f}{\longrightarrow} & G_3\\
    g_1 &\longmapsto&g_2&\longmapsto&g_3
     & 
\end{array}
$$
$$
f(g(x\overset{1}{*}y))=f(g(x)\overset{2}{*}g(y))=f(g(x))\overset{3}{*}f(g(y))
$$\\

%ENDO/AUTO

\textbf{Definición:} Si $(G_1,\cdot)=(G_2,*)=G$ se llama endomorfismo, y si es biyectivo automorfismo.\\\\

%AUTOMORFISMO IDENTIDAD

\textbf{Ejemplo:} Automorfismo identidad $1_G(x)=x \quad \forall x\in G$.\\\\

%PROPIEDADES BASICAS

\textbf{Proposición:} Si  $f\::\:G_1\longrightarrow G_2$ es un homomorfismo de de grupos, entonces se verifican\\
$$
\begin{array}{ll}
\text{\textit{i)}} & \text{$f(e_1)=e_2$}\\
\text{\textit{ii)}} & \text{$f(a^{-1})=f(a)^{-1}$}\\
\text{\textit{iii)}} & \text{Si $o(a)=n<\infty$, entonces $f(a)^n=e_2$}\\
\text{\textit{iv}} & \text{Si $H_1\le G_1$, entonces $f(H_1)\le G_2$}\\
\text{\textit{v)}} & \text{Si $H_2\le G_2$, entonces $f^1(H_2)\le G_1$}\\
\end{array}
$$
\textit{Dem:} \textit{i)} $f(e_1)=f(e_1\cdot e_1)=f(e_1)*f(e_1)$,  entonces $f(e_1)=f(e_1)*f(e_1) \Leftrightarrow f(e_1)^{-1} * f(e_1)= f(e_1)^{-1}*f(e_1)*f(e_1) \Leftrightarrow e_2=e_2*f(e_1)=f(e_1)$.\\ \textit{ii)} Veamos que $f(a)*f(a^-1)=e_2=f(a^{-1})*f(a)$.Aplicando \textit{i)} $f(a)*f(a^{-1})=f(a\cdot a^{-1})=f(e_1)=e_2=f(e_1)=f(a^{-1}\cdot a)=f(a^{-1})*f(a)$.\\
\textit{iii)} $f(a^n)=f(a)*\overset{n}{\cdots}*f(a)=f(a)^n$,entonces $e_2=f(e_1)=f(a^{o(a)})=f(a)^{o(a)}$\\
\textit{iv)}
$$
\begin{array}{rcl}
e_2\in f(H_1) && f(\underbrace{e_1}_{\in H_1})=e_2 \\
f(y)\in f(H_1)\Rightarrow f(y)^{-1}\in f(H_1) && f(y)^{-1}=f(\underbrace{y^{-1}}_{\in H_1})\in f(H_1)
\end{array}
$$
\textit{v)}
$$
\begin{array}{rcl}
e_1\in f^{-1}(H_2) && f^{-1}(\underbrace{e_2}_{\in H_2})\ni e_1 \\
y\in f^{-1}(H_2)\Rightarrow y^{-1}\in f^{-1}(H_1) && f(y)\in H_2 \Rightarrow f(y)^{-1}\in H_2\Rightarrow f(y^{-1})\in H_2 \Rightarrow y^{-1}\in f^{-1}(H_2)
\end{array}
$$\\

%NUCLEO

\textbf{Definición:} Dado un homorfismo $f$ de $(G_1,\cdot)$ en $(G_2,*)$. Llamamos núcleo de $f$ a $Ker(f)=\{ g\in G_1 \mid f(g)=e_2 \}$.\\\\

%KER NORMAL

\textbf{Proposición:} 
Dado un homorfismo $f$ de $(G_1,\cdot)$ en $(G_2,*)$. $Ker(f)\trianglelefteq G_1$\\
\textit{Dem:} Primero veamos que es subgrupo. $e_1\in Ker(f)$ ya que $f(e_1)=e_2$. Si $x\in Ker(f)$, entonces $x^{-1}\in Ker(f)$ ya que $e_2=f(e_1)=f(x^{-1} x)=f(x^{-1})*f(x)=f(x^{-1})*e_2=f(x^{-1})$. Ahora veamos que es subgrupo normal probando que $x\in Ker(f) \Rightarrow y^{-1}xy\in Ker(f)$. Observamos que $f(y^{-1}xy)=f(y^{-1})*f(x)*f(y)=f(y)^{-1}*e-2*f(y)=e_2$.\\\\

%INYECTIVO KER

\textbf{Proposición:} Dado un homorfismo $f$ de $(G_1,\cdot)$ en $(G_2,*)$. $f$ es inyectivo sii $Ker(f)=\{e_1\}$.\\
\textit{Dem:} Si $f$ es inyectiva $f(a)=e_2=f(e_1)$, entonces $a=e_1$. Si $f(a)=f(b)$, entonces $e_2=f(a)*f(b)^{-1}=f(a)*f(b^{-1})=f(ab^{-1})\in Ker(f)=\{e_1\}$, ahora $ab^{-1}=e_1$ y $a=b$.\\\\

%PRIMER TEOREMA DE ISOMORFIA

\textbf{Primer teorema de isomorfía de grupos:} Dado un homorfismo $f$ de $G_1,\cdot)$ en $(G_2,*)$. $G_1/Ker(f)\simeq f(G_1)$.\\
\textit{Dem:} Consideramos la siguiente aplicación
$$
\begin{array}{cccc}
    \bar{f}\::&G_1/Ker(f)&\longrightarrow&f(G_1)\\
        &aKer(f)&\longmapsto&f(a)
\end{array}
$$
Veamos que es isomorfismo.Primero que es suprayectiva pues todo elemento de $f(G_1)$ es de la forma $f(x)$ con $x\in G_1$. Veamos que es inyectiva, si $aKer(f)\in Ker(\bar{f})$ entonces $f(a)=e_2$ por lo que $a\in Ker(f)$ y $aKer(f)=Ker(f)$. Veamos que es homomorfismo $\bar{f}(aKer(f)\cdot bKer(f))=\bar{f}((ab)Ker(f))=f(ab)=f(a)*f(b)=\bar{f}(aKer(f))*\bar{f}(bKer(f))$.\\\\

%EPIMORFISMO CANONICO

\textbf{Definición:} Sea $G$ un grupo y $N\trianglelefteq G$. Definimos el epimorfismo canónico
$$
\begin{array}{cccc}
    \pi\::&G&\longrightarrow&G/N\\
        &g&\longmapsto&gN
\end{array}
$$
Es epimorfismo ya que todo elemento de $G/N$ es de la forma $xN$ y por lo tanto es imagen de $x$. Por otra parte, $\pi(xy)=(xy)N=(xN)(yN)=\pi(x)\pi(y)$.\\\\

%KER EPI

\textbf{Proposición:} Si $f\::\:G\longrightarrow H$ es homomorfismo, entonces $Ker(f)\trianglelefteq G$ y $
\begin{array}{cccc}
    \pi\::&G&\longrightarrow&g/Ker(f)\\
        &a&\longmapsto&aKer(f)
\end{array}
$ es un epimorfismo con $Ker(\pi)=Ker(f)$.\\
\textit{Dem:} $Ker(f)\trianglelefteq G$ ya está probado, y $x\in Ker(\pi)$ entonces $xKer(f)=Ker(f)$ y $x\in Ker(f)$. Ahora, si $x\in Ker(f)$, entonces $\pi(x)=xKer(f)=Ker(f)$ y $x \in Ker(\pi)$.\\\\

%DESCOMPOSICION CANONICA

\textbf{Definición:} Dado un homorfismo $f$ de $G_1,\cdot)$ en $(G_2,*)$.Se tiene la descomposición canonica del homomorfismo $$f=\underbrace{\iota}_\text{monomorfismo inclusión} \circ \underbrace{\bar{f}}_\text{isomorfismo} \circ \underbrace{\pi}_\text{epimorfismo canónico}$$
$$
\overset{f}{\overrightarrow{\begin{array}{cccccccccccc}
    G_1&\overset{\pi}{\longrightarrow}&G_1/Ker(f)&\overset{\bar{f}}{\longrightarrow}&f(G_1)&\overset{\iota}{\longrightarrow}&G_2\\
    a&\longmapsto&aKer(f)&\longmapsto&f(a)&\longmapsto&f(a)
\end{array}}}
$$\\

%SUBGRUPOS COCIENTE

\textbf{Teorema:} Sea $N\trianglelefteq G$. Entonces todo subgrupo de $G/N$ es de la forma $H/N$ con $N\subseteq H \le G$.\\
\textit{Dem:} Suponganmos que $K\le G/N$, Sea $H=\{ x\in G \mid xN\in K \}$. Veamos que $H$ es subgrupo. Tenemos que $e\in H$ ya que $e_1N=N$ que es el elemento neutro en $G/N$ y por lo tanto esta en $K$. Veamos que si $y\in H$, entonces $y^{-1}\in H$. Tenemos que $yN\in K$ entonces por ser $N$ subgrupo normal $K\ni(yN)^{-1}=N^{-1}y^{-1}=Ny^{-1}=y^{-1}N$ por lo que $y^{-1}\in H$. Para ver que $N\subseteq H$, si $n\in N$, entonces $nN=N\in K$ por ser el elemento neutro y $n\in H$.\\\\

%SUBGRUPOS COCIENTE

\textbf{Teorema:} Sea $N\trianglelefteq G$. Entonces si $N\le H\le G$ entonces $H/N\le G/N$.\\
\textit{Dem:} Veamos que es subgrupo. Si $x,y\in H$, $xN,yN\in H/N$ entonces $xy\in H$ y $(xy)N\in H/N$. Si $e\in H$, entonces $eN=N$. Si $x\in H$, $xN\in H/N$, entonces $x^{-1}\in H$ y $x^{-1}N\in H/N$, $(xN)(x^{-1}N)=(Nx)(x^{-1})=N$.\\\\

%SEGUNDO TEOREMA DE ISOMORFIA

\textbf{Segundo teorema de isomorfía de grupos:} Sea $G$ un grupo, $N\trianglelefteq G$ y $H\le G$, entonces
$$
\begin{array}{ll}
    \text{\textit{i)}} & \text{$N\trianglelefteq NH$} \\
    \text{\textit{ii)}} &  \text{$N\cap H \trianglelefteq H$}\\
    \text{\textit{iii)}} & \text{$NH/N\simeq H/(N\cap H)$}
\end{array}
$$
\textit{Dem:} \textit{i)} Veamos que $(nh)^{-1}N(nh)=N$. $(nh)^{-1}N(nh)=h^{-1}n^{-1}Nnh=h^{-1}Nh=N$.\\
\textit{ii)} Veamos que si $h \in H$, entonces $h^{-1}(N\cap H)h=N\cap H$. $(h^{-1}Nh)\cap(h^{-1}Hh)=N\cap H$.\\
\textit{iii)} La aplicación $
\begin{array}{cccc}
    \bar{f}\::&NH/N&\longrightarrow&H/(N\cap H)\\
        &hnN&\longmapsto&h(N\cap H)
\end{array}
$ es un isomorfismo de grupos. Está bien definida ya que si $hN=h'N$ entonces $h^{-1}h'\in N,H$ y $h(N\cap H)=h'(N\cap H)$. Es homomorfismo ya que $f((hN)(h'N))=f((hh')N) = (hh')(N\cap H)=h(N\cap H)h'(N\cap H))=f(hN)f(h'N)$. Es inyectiva ya que si $x\in Ker(f)$, entonces $x\in (N\cap H)$ y $xN=N$, es decir, $Ker(f)=\{N\}$. Es suprayectiva ya que dado un $h\in H$ existe $hN$ tal que $f(hN)=h(N\cap H)$.\\\\

%TERCER TEOREMA DE ISOMORFIA

\textbf{Tercer teorema de isomorfia de grupos:} Sea $G$ un grupo y $N\subseteq M$ dos subgrupos normales de $G$. Entonces se verifican
$$
\begin{array}{ll}
    \text{\textit{i)}} & M/N \trianglelefteq G/N \\
    \text{\textit{ii)}} & (G/N)/(M/N)\simeq G/M
\end{array}
$$
\textit{Dem:} Sabemos que $M/N$ es subgrupo.
$$
(gN)^{-1}(mN)(gN)=(g^{-1}mg)N=mN
$$
Consideramos la aplicación $\begin{array}{cccc}
    f\::&G/N&\longrightarrow&G/M\\
        &xN&\longmapsto&xM
\end{array}$, esta bien definida ya que si $xN=yN$, entonces $x^{-1}y\in N\subseteq M$ y $xM=yM$. Es homomorfismo ya que $f((xN)(yN)=f((xy)N)=(xy)M=f(xN)f(yN)$. Es suprayectiva ya que para todo $xM\in G/M$ existe $xN\in G/N$ tal que $f(xN)=xM$. Calculamos $ker(f)$, si $N=f(xM)=xN$ se tiene que $x\in N$ por lo que
$$
Ker(f)=\{xM\mid x\in N\}=M/N
$$
Por el primer teorema de isomorfía, aplicado a $f$,
$$
(G/N)/Ker(f)=(G/N)/(M/N)\simeq f(G/N)=G/M
$$
$$
(G/N)/(M/N)\simeq G/M
$$\\
\end{document}