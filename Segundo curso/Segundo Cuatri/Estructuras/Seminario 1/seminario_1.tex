\documentclass{article}
\usepackage[utf8]{inputenc}
\usepackage{graphicx}
\usepackage[spanish]{babel}
\usepackage{amssymb,amsmath,geometry,xcolor}
\usepackage{tikz,mathtools}
\usepackage{etoolbox} %titulo
\makeatletter %titulo
\patchcmd{\@maketitle}{\vskip 2em}{\vspace*{-3cm}}{}{} %titulo
\makeatother %titulo
\usepackage{vmargin}
\setpapersize{A4}
\setmargins{2.5cm}       % margen izquierdo
{1.5cm}                        % margen superior
{16.5cm}                      % anchura del texto
{23.42cm}                    % altura del texto
{10pt}                           % altura de los encabezados
{1cm}                           % espacio entre el texto y los encabezados
{0pt}                             % altura del pie de página
{2cm}                           % espacio entre el texto y el pie de página
\title{Primer seminario de estructuras algebráicas}
\author{Andoni Latorre Galarraga}
\date{}
\newcommand{\bb}[1]{\mathbb{#1}}
\newcommand{\p}{\textbf{Proposición: }}
\newcommand{\dem}{\textit{Dem: }}
\newcommand{\R}{\mathbb{R}}
\newcommand{\nota}[3][2ex]{
    \underset{\mathclap{
        \begin{tikzpicture}
          \draw[->] (0, 0) to ++(0,#1);
          \node[below] at (0,0) {#3};
        \end{tikzpicture}}}{#2}
}
\begin{document}

\maketitle
\noindent \textbf{2.}\\
\textit{d)}
$$
o(\overline{12})=\frac{100}{m.c.d.(12,100)}=\frac{100}{4}=25
$$
\textit{e)}
$$
o(\overline{-35})=o(\overline{65})=\frac{100}{m.c.d.(65,100)}=\frac{100}{4}=20
$$
\textbf{4.}\\
\textit{c)}
Por el teorema de Euler, los candidatos son los divisores de $\phi(17)=16$.
$$
\overline{14} = \overline{-3}
$$
$$
\begin{array}{ll}
  \overline{-3}^2    & = \overline{9}\\
  \overline{-3}^4    & = \overline{81}=\overline{-4}\\
  \overline{-3}^8    & = \overline{16}=\overline{-1}\\
  \overline{-3}^{16} & = \overline{1}
\end{array}\quad \Rightarrow \quad o(\overline{14})=16
$$
\textit{e)}
Por el teorema de Euler, los candidatos son los divisores de $\phi(25)=25-5=20$.
$$
\begin{array}{ll}
  \overline{2}^2    & =\overline{4}\\
  \overline{2}^4    & =\overline{16}\\
  \overline{2}^5    & =\overline{32}=\overline{7}\\
  \overline{2}^{10} & =\overline{49}=\overline{-1}\\
  \overline{2}^{20} & =\overline{1}
\end{array}\quad \Rightarrow \quad o(\overline{2})=20
$$
\textbf{7.}
Si interpretamos $D_9$ como el subgrupo de $\Sigma_9$ generado por $a=\left(\begin{array}{c}123456789\\234567891\end{array}\right)$ y $b= \left(\begin{array}{c}123456789\\132547698\end{array}\right)$.
$$
a = (123456789) \quad \Rightarrow \quad o(a)=9
$$
$$
b=(23)(45)(67)(89) \quad \Rightarrow \quad o(b)=2
$$
$$
a^3=\left(\begin{array}{c}123456789\\456789123\end{array}\right)
= (147)(258)(369) \quad \Rightarrow \quad o(a^3)=3
$$
\end{document}