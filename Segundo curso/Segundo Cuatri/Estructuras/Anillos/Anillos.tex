\documentclass{article}
\usepackage[utf8]{inputenc}
\usepackage{graphicx}
\usepackage[spanish]{babel}
\usepackage{amssymb,amsmath,geometry,tikz,mathtools, changepage, tipa, array, float, sidecap, mwe}
\usepackage{etoolbox} %titulo
\makeatletter %titulo
\patchcmd{\@maketitle}{\vskip 2em}{\vspace*{-3cm}}{}{} %titulo
\makeatother %titulo
\usepackage{vmargin}
\setpapersize{A4}
\setmargins{2.5cm} % margen izquierdo
{1.5cm} % margen superior
{16.5cm} % anchura del texto
{23.42cm} % altura del texto
{10pt} % altura de los encabezados
{1cm} % espacio entre el texto y los encabezados
{0pt} % altura del pie de página
{2cm} % espacio entre el texto y el pie de página
\title{Anillos}
\author{Aitor Moreno Rebollo}
\date{01/05/2021}

\newcommand{\nota}[3][2ex]{
    \underset{\mathclap{
        \begin{tikzpicture}
          \draw[<-] (0, 0) to ++(0,#1);
          \node[below] at (0,0) {#3};
        \end{tikzpicture}}}{#2}
}

%\newcommand{\noteeq}[1]{\underset{\mathclap{\tikz \node {$\downarrow$} node [below=1ex] {#1};}}{=}} %Anotaciones en igualdades

\begin{document}
\setlength{\parindent}{0cm}
\maketitle

$\textbf{Def.}$ Un anillo es una terna $(A, +, \cdot)$ en la que se cumple:
\begin{adjustwidth}{0.5cm}{}
    i) $(A, +)$ es grupo abeliano.\\
    ii) $(a\cdot b)c = a\cdot (b \cdot c) \ \forall a, b, c \in A$\\
    iii)$a\cdot (b + c) = a\cdot b + a \cdot c$ y $(b + c) \cdot a = b\cdot a + c \cdot a \ \forall a,b,c \in A$
\end{adjustwidth}
Denotaremos la operación definida en el grupo abeliano por $+$, es decir, con notación aditiva. La operación $\cdot$ la denotaremos frecuentemente mediante yuxtaposición para simplificar notación.\\

Un anillo se dice unitario si existe un elemento $1_A \in A$ tal que $1_A \cdot a = a\cdot 1_A = a \ \  \forall a \in A$.\\
Un anillo se dice conmutativo o abeliano si sus elementos conmutan respecto de la operación producto, es decir, si $ab = ba \ \forall a,b \in A$:\\

$\textbf{Def.}$ Si $A$ es un anillo, $a\in A$, $n \in \mathbb{N}$, definimos $na = \underbrace{a + ... + a}_{\text{n veces}}$, y $(-n)a = \underbrace{(-a) + ... + (-a)}_{\text{n veces}} \overset{\text{not}}{=} - a - ... - a$. Definimos también $0\cdot a = 0_A$.\\

$\textbf{Prop.}$ $A$ un anillo, si $a,b\in A$, y $n,m \in \mathbb{Z}$, entonces se cumple:
\begin{adjustwidth}{0.5cm}{}
    i) $(n + m)a = na + ma$\\
    ii)$(nm)a = n(ma)$\\
    iii) $n(a + b) = na + nb$
\end{adjustwidth}
La demostración es trivial 'contando'  la cantidad de veces que aparecen los elementos $a$ y $b$ y sus inversos en cada lado de las igualdades.\\

$\textbf{Def.}$ $A$ un anillo, $a \in A$, $n \in \mathbb{N}$, definimos $a^n = \underbrace{a \cdots a}_{\text{n veces}}$, y definimos también $a^{0} = 1_A$.\\

$\textbf{Def.}$ $A$ un anillo, se dice que $a \in A$ es una unidad (o que es inversible), si existe un elemento $a' \in A$ tal que $aa' = a'a = 1_A$. Denotaremos $a' = a^{-1}$ y lo llamaremos inverso multiplicativo.\\

$\textbf{Not.}$ El conjunto de unidades de $A$ es $\mathfrak{U}(A) = \{a \in A \ | \ \exists a^{-1}\in A, \ a^{-1}a = aa^{-1} = 1_A\}$\\

$\textbf{Def.}$ Sea $a\in \mathfrak{U}(A)$, $n \in \mathbb{N}$, definimos $a^{-n} = (a^{-1})^n$\\



$\textbf{Prop.}$ $A$ un anillo, $a,b \in \mathfrak{U}(A)$, $n,m \in \mathbb{Z}$, entonces:
\begin{adjustwidth}{0.5cm}{}
    i) $a^{n + m} = a^na^m$\\
    ii) $a^{nm} = (a^n)^m$\\
    iii) Si $a$ y $b$ conmutan, entonces $(ab)^n = a^nb^n$
\end{adjustwidth}
La demostración, una vez más, es trivial 'contando' la cantidad de veces que aparecen los elementos en los productorios de derecha e izquierda de las igualdades, y utilizando que $a^{-n} = (a^{-1})^n$ y $a^{0} = 1_A$. Para exponentes positivos, esta proposición se cumple para cualesquiera elementos de $A$.\\

$\textbf{Prop.}$ $A$ un anillo unitario. Entonces $\mathfrak{U}(A)$ es un grupo respecto de la operación multiplicativa definida en $A$.
\begin{adjustwidth}{1cm}{}
    $\textbf{Dem.}$ La operación producto es asociativa. Es evidente que el producto de inversibles es inversible: $a,b \in \mathfrak{U}(A)$, $(ab)^{-1} = b^{-1}a^{-1}$. Además, $1_A \in \mathfrak{U}(A)$ por ser $A$ unitario.\\
\end{adjustwidth}
Si $A$ es anillo abeliano, entonces $\mathfrak{U}(A)$ es anillo abeliano trivialmente.\\

$\textbf{Prop.}$ $A$ un anillo, $a,b \in A$, $n \in \mathbb{Z}$. Entonces:
\begin{adjustwidth}{0.5cm}{}
    i) $0_Aa = a0_A$\\
    ii) $a(-b) = (-a)b = - (ab)$\\
    iii) $(-a)(-b) = ab$\\
    iv) $na = (n1_A)a$
\end{adjustwidth}
\begin{adjustwidth}{1cm}{}
    $\textbf{Dem.}$ \\
    i) $0_Aa = (0_A)a = (0_A + 0_A)a = 0_Aa + 0_Aa \Longleftrightarrow 0_A = 0_Aa$\\
    ii) $a(-b) + ab = a(-b + b) = a0_A 0 = 0_A \Longrightarrow a(-b) = -(ab)$. Análogamente, $(-a)b = -(ab)$\\
    iii) $(-a)(-b) = -(a(-b)) = -(-(ab)) = ab$\\
    iv) $(n1_A)a = (\underbrace{1_A + ... + 1_A}_{\text{n veces}})a = \underbrace{(1_Aa) + ... + (1_Aa)}_{\text{n veces}} = \underbrace{a + ... + a}_{\text{n veces}} = na$\\
\end{adjustwidth}

$\textbf{Def.}$ $A$ un anillo, $a,b \in A\setminus \{0_A\}$ y satisfacen que $ab = 0_A$. Entoncesse dice que $a$ es un divisor de cero a izquierda y $b$ es un divisor de cero a derecha. Si $A$ es abeliano, se dice simplemente que $a$ y $b$ son divisores de cero.\\

$\textbf{Def.}$ Se llama dominio de integridad o anillo íntegro a un anillo conmutativo sin divisores de cero.\\

$\textbf{Def.}$ Se llama cuerpo a un anillo conmutativo $A$ en el que $\mathfrak{U}(A) = A\setminus \{0_A\}$. El anillo trivial \{0\} no es un cuerpo.\\

$\textbf{Prop.}$ Si $A$ es cuerpo, entonces $A$ es dominio de integridad.
\begin{adjustwidth}{1cm}{}
    $\textbf{Dem.}$ Si $A$ es cuerpo entonces es conmutativo y no trivial. Además, todos sus elementos, salvo el $0_A$, admiten inverso multiplicativo. Sea $a \in A \setminus \{0_A\}$. Sea $b\in A$. Consideramos la ecuación $ab = 0_A$ y veamos que $b = 0_A$. En efecto, $ab = 0_A \Longrightarrow a^{-1}ab = a^{-1}0_A = 0_A \Longrightarrow b = 0_A$.\\
\end{adjustwidth}
Obviamente, el recíproco no es cierto, esto es, existen dominios de integridad que no son cuerpos. Un ejemplo es $(\mathbb{Z}, +, \cdot)$. Sin embargo, se propone probar la siguiente proposición:\\

$\textbf{Prop (PROPUESTA).}$ Si $A$ es un dominio de integridad finito, entonces $A$ es cuerpo.
\begin{adjustwidth}{1cm}{}
    $\textbf{Dem.}$
    Utilizaremos primero un pequeño
    \begin{adjustwidth}{0.5cm}{}
        $\textbf{Lema.}$ $a, b\in A$, $ab$ es inversible si y solo si tanto $a$ como $b$ lo son.
        \begin{adjustwidth}{0.5cm}{}
            $\textbf{Dem.}$ $(ab)(ab)^{-1} = 1_A \Longrightarrow (ab)(ab^{-1}) = a(b(ab)^{-1})) = 1_A$, y por tanto $b(ab)^{-1} = a^{-1}$.\\
            Por ser $A$ conmutativo se concluye que $b^{-1} = a(ab)^{-1}$.\\
            La otra implicación ya la hemos visto.
        \end{adjustwidth}
    \end{adjustwidth}

    Ahora, $A = \{1_A, 0_A, a_1, ..., a_n \} $, y $a_ia_j = 0_A  \Longleftrightarrow $ o bien $a_i = 0_A$, o bien $a_j = 0_A$. Todo esto por ser $A$ dominio de integridad finito. Sean ahora dos elementos $a_i, a_j \in A\setminus \{0_A\}$, y consideramos su producto. Si $a_ia_j = 1_A$, entonces son inversos entre sí. Si no, llamamos $a_ia_j = a_{k_1}$, que es distinto de $0_A$ por elección de $a_i$ y $a_j$. Consideramos la sucesión $\{ a_{k_l} \}_{l \in \mathbb{N}}$, de forma que $a_{k_{l-1}} = a_{k_l}^2$, es decir, $a_{k_l} = (a_ia_j)^l$. Es evidente que $\{ a_{k_l} \} \subseteq A\setminus \{0_A\}$, y como este conjunto es finito, entonces necesariamente $a_{k_t} = a_{k_s}$ para ciertos $s,t \in \mathbb{N}$, $s<t$, y por tanto, $a_{k_t} = a_{k_s} \Longrightarrow (a_ia_j)^t = (a_ia_j)^s(a_ia_j)^{t-s} = (a_ia_j)^s \Longrightarrow (a_ia_j)^s(a_ia_j)^{t-s} - (a_ia_j)^s = 0_A \Longrightarrow (a_ia_j)^s((a_ia_j)^{t-s} - 1_A) = 0_A \Longrightarrow (a_ia_j)^{t-s} - 1_A = 0_A \Longrightarrow (a_ia_j)^{t-s} = 1_A \Longrightarrow (a_ia_j)(a_ia_j)^{t-s-1} = 1_A$, y por tanto $(a_ia_j)^{t-s-1}$ es el inverso de $(a_ia_j)$, y es no nulo, pues $(a_ia_j)(a_ia_j)^{t-s-1} = 1_A$. Por tanto $a_i$ es inversible para cada $i$.\\

\end{adjustwidth}

$\textbf{Def.}$ $(A, + \cdot)$, $(B, + , \cdot)$ anillos, se dice que $(B, + , \cdot)$ es subanillo de $A$ si:
\begin{adjustwidth}{0.5cm}{}
    i) $(B. +)$ es subgrupo de $(A, +)$\\
    ii) $ab\in B \ \ \forall a,b \in B$\\
    iii) $1_A \in B$, y por tanto, $1_A = 1_B$
\end{adjustwidth}
Cuando se sobreentienden las operaciones, diremos sencillamente que $B$ es subanillo de $A$. Se propone como ejercicio ver que la condición iii) no se deduce de la i) y la ii).\\

$\textbf{Ejercicio (PROPUESTO).}$ Ver que, en la definición de subanillo, la tercera condición no se deduce de las dos anteriores.
\begin{adjustwidth}{0.5cm}{}
    Lo más sencillo es dar un contraejemplo. $(3\mathbb{Z}/6\mathbb{Z}, +)$ es subgrupo de $(\mathbb{Z}/6\mathbb{Z}, +)$. Consideramos los respectivos anillos con el producto usual de coclases. Se satisface la condición ii), pues los únicos elementos de $3\mathbb{Z}/6\mathbb{Z}$ son $\overline{0}$ y $\overline{3}$, y el producto es $\overline{0} = 0_{\mathbb{Z}/6\mathbb{Z}}$. Sin embargo $1_{\mathbb{Z}/6\mathbb{Z}} = \overline{1} \not \in 3\mathbb{Z}/6\mathbb{Z}$.\\
\end{adjustwidth}





\end{document}