\documentclass{article}
\usepackage[utf8]{inputenc}
\usepackage{graphicx}
\usepackage[spanish]{babel}
\usepackage{amssymb,amsmath,geometry,xcolor}
\usepackage{etoolbox} %titulo
\makeatletter %titulo
\patchcmd{\@maketitle}{\vskip 2em}{\vspace*{-3cm}}{}{} %titulo
\makeatother %titulo
\usepackage{vmargin}
\setpapersize{A4}
\setmargins{2.5cm}       % margen izquierdo
{1.5cm}                        % margen superior
{16.5cm}                      % anchura del texto
{23.42cm}                    % altura del texto
{10pt}                           % altura de los encabezados
{1cm}                           % espacio entre el texto y los encabezados
{0pt}                             % altura del pie de página
{2cm}                           % espacio entre el texto y el pie de página
\title{Tercer Seminario}
\author{Andoni Latorre Galarraga}
\date{}
\newcommand{\bb}[1]{\mathbb{#1}}
\newcommand{\R}{\mathbb{R}}
\usepackage{tikz,mathtools}
\newcommand{\nota}[3][2ex]{
    \underset{\mathclap{
        \begin{tikzpicture}
          \draw[->] (0, 0) to ++(0,#1);
          \node[below] at (0,0) {#3};
        \end{tikzpicture}}}{#2}
}
\begin{document}

\maketitle
\noindent
6. Sean $h\in H_1\cap H_2$ y $g\in G$ como $h\in H_1$ y $H_1\trianglelefteq G$ $g^{-1} h g \in H_1$, además como $h\in H_2$ y $H_2\trianglelefteq G$ $g^{-1} h g \in H_2$ y se tiene que $g^{-1} h g \in H_1\cap H_2$.\\\\
\noindent
9. La condicion es que $G$ sea abeliano.\\
abeliano $\Rightarrow$ automorfismo\\
Es biyectivo porque el inverso siempre existe y es único Además,
$$
\forall a,b\in G \quad f(ab)=b^{-1}a^{-1}\nota{=}{\text{abeliano}}a^{-1}b^{-1}=f(a)f(b)
$$
abeliano $\Leftarrow$ automorfismo\\
$$
\forall a,b\in G \quad f(ab)=f(a)f(b) \Rightarrow b^{-1}a^{-1}=a^{-1}b^{-1} \Rightarrow ab(b^{-1}a^{-1})ba=ab(a^{-1}b^{-1})ba \Rightarrow ba=ab
$$
12. Veamos que $f(ab)=f(a)f(b) \forall a,b\in G$.\\
$n>0$
$$
f(ab)=(ab)^n=\underbrace{ab\cdots ab}_n \nota{=}{abeliano}\underbrace{a\cdots a}_n \underbrace{b\cdots b}_n=a^nb^n=f(a)f(b)
$$
$$
f(ab)=(ab)^{-n}=\underbrace{-(ab)\cdots -(ab)}_n \nota{=}{abeliano}\underbrace{-a\cdots -a}_n \underbrace{-b\cdots -b}_n=a^{-n}b^{-n}=f(a)f(b)
$$
$n=0$
$$
f(ab)=(ab)^0=1=1\cdot 1=a^0b^0=f(a)f(b)
$$
No siempre es automorfismo. Sea $G=(\bb{Z},+)$, $f$ no es suprayectiva ya que con $n=2$ se tiene que $f^{-1}(\{3\})=\emptyset$
\end{document}