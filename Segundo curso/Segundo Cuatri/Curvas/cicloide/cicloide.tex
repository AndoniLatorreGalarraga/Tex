\documentclass{article}
\usepackage[utf8]{inputenc}
\usepackage{graphicx}
\usepackage[spanish]{babel}
\usepackage{amssymb,amsmath,geometry}
\usepackage[hidelinks]{hyperref}
\usepackage{etoolbox} %titulo
\makeatletter %titulo
\patchcmd{\@maketitle}{\vskip 2em}{\vspace*{-3cm}}{}{} %titulo
\makeatother %titulo
\usepackage{vmargin}
\setpapersize{A4}
\setmargins{2.5cm}       % margen izquierdo
{1.5cm}                        % margen superior
{16.5cm}                      % anchura del texto
{23.42cm}                    % altura del texto
{10pt}                           % altura de los encabezados
{1cm}                           % espacio entre el texto y los encabezados
{0pt}                             % altura del pie de página
{2cm}                           % espacio entre el texto y el pie de página
\title{Cicloide}
\author{Aitor Moreno Rebollo y Andoni Latorre Galarraga}
\date{}
%comandos
\newcommand{\bb}[1]{\mathbb{#1}}
\newcommand{\R}{\bb{R}}
\begin{document}
\setlength{\parindent}{0cm}
\maketitle

\section{Descripción}
Una cicloide es la curva que recorre un punto de la circunferencia al rodar.
\begin{center}\begin{align*}
    \includegraphics[scale=0.3]{figuras/cicloide descipcion 1.PNG}&
    \includegraphics[scale=0.3]{figuras/cicloide descipcion 2.PNG}\\
    \includegraphics[scale=0.3]{figuras/cicloide descipcion 3.PNG}&
    \includegraphics[scale=0.3]{figuras/cicloide descipcion 4.PNG}
\end{align*}
\includegraphics[scale=0.3]{figuras/cicloide descipcion 5.PNG}\\
\href{https://www.geogebra.org/calculator/ju6wnwpc}{Geogebra applet}
\end{center}
\section{Parametrización}
$$
\begin{array}{crcl}
\alpha : & (0,2\pi) & \longrightarrow & \bb{R}^2 \\
& t & \longmapsto     & (t-\sin(t),1-\cos(t))
\end{array}
$$
\section{Velocidad}
$$
\begin{array}{crcl}
\alpha' : & (0,2\pi) & \longrightarrow & \bb{R}^2 \\
& t & \longmapsto     & (1-\cos(t),\sin(t))
\end{array}
$$
\section{Diedro de Frenet}
Tenemos que $\begin{array}{crcl}
\alpha'' : & (0,2\pi) & \longrightarrow & \R^2 \\
& t & \longmapsto     & (\sin(t),\cos(t))
\end{array}$ y $\left\| \alpha'(t) \right\|=\sqrt{(1-\cos(t))^2+\sin^2(t)}$
\section{Longitud}
Para calcular la longitud de la cicloide evaluamos la siguiente integral
$$
\int_0^2\pi \left\| \alpha'(t) \right\| dt = \int_0^{2\pi} \sqrt{2-2\cos(t)} dt =\sqrt{2} \int_0^{2\pi} \sqrt{1-\cos(t)} dt = \sqrt{2}\left[ -2 \sqrt{1-\cos(t)}\cot\left(\frac{t}{2}\right)\right]_0^{2\pi} = \sqrt{2}\sqrt{2}4=8

$$
\subsection{Parametrización por Longitud de arco}

\section{Curvatura}

\section{Circunferencia osculatriz}

\section{Curiosidades}
\subsection{Curva braquistocrona}
\subsection{Curva tautócrona}
\end{document}
