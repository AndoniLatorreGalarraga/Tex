\documentclass{article}
\usepackage[utf8]{inputenc}
\usepackage{graphicx}
\usepackage[spanish]{babel}
\usepackage{amssymb,amsmath,geometry,xcolor}
%\pagecolor[rgb]{0,0,0} \color[rgb]{1,1,1}
\usepackage{etoolbox} %titulo
\makeatletter %titulo
\patchcmd{\@maketitle}{\vskip 2em}{\vspace*{-3cm}}{}{} %titulo
\makeatother %titulo
\usepackage{vmargin}
\setpapersize{A4}
\setmargins{2.5cm}       % margen izquierdo
{1.5cm}                        % margen superior
{16.5cm}                      % anchura del texto
{23.42cm}                    % altura del texto
{10pt}                           % altura de los encabezados
{1cm}                           % espacio entre el texto y los encabezados
{0pt}                             % altura del pie de página
{2cm}                           % espacio entre el texto y el pie de página
\title{Problema 32}
\author{Andoni Latorre Galarraga}
\date{}
\newcommand{\bb}[1]{\mathbb{#1}}
\newcommand{\p}[0]{\textbf{Proposición:}}
\newcommand{\dem}[0]{\textit{Dem:}}
\begin{document}
\setlength{\parindent}{0cm}
\maketitle
Supongamos que la curva está contenida en una esfera y tiene curvatura constante.\\
Por ser parte de una esfera,
\begin{equation}\label{eq:sf}
\exists x_0 \mid (x_0-\alpha(s))\bb{T}=0 \quad \forall s
\end{equation}
Derivando,
$$
(x_0-\alpha(s))\bb{T}'(s)-\bb{T}(s)\alpha'(s)=0
$$
$$
k(x_0-\alpha(s))\bb{N}(s)-\bb{T}(s)\bb{T}(s)=0
$$
\begin{equation}\label{eq:1k}
(x_0-\alpha(s))\bb{N}(s)=\frac{1}{k}
\end{equation}
Por otra parte, si escribimos $x_0.\alpha(s)$ en la base $\{\bb{T}(s),\bb{N}(s),\bb{B}(s)\}$,
$$
x_0-\alpha(s)=a(s)\bb{T}(s)+b(s)\bb{N}(s)+c(s)\bb{B}(s)
$$
Por \eqref{eq:sf}, sabemos que $a(s)=0$.
$$
x_0-\alpha(s)=b(s)\bb{N}(s)+c(s)\bb{B}(s)
$$
Por \eqref{eq:1k}, $b(s)=\frac{1}{k}$.
$$
x_0-\alpha(s)=\frac{1}{k}\bb{N}(s)+c(s)\bb{B}(s)
$$
Derivando,
$$
-\alpha'(s)=\frac{1}{k}\bb{N}'(s)+c(s)\bb{B}'(s)+c'(s)\bb{B}(s)
$$
$$
-\alpha(s)=\frac{1}{k}(-k\bb{T}(s)+\tau(s)\bb{B}(s))+c(s)(-\tau(s)\bb{N}(s))+c'(s)\bb{B}(s)
$$
$$
-\bb{T}(s)=-\bb{T}(s)+\bb{N}\underbrace{(-\tau(s)c(s))}_0+\bb{B}(s)\underbrace{\left( \frac{\tau(s)}{k}+c'(s) \right)}_0
$$
Tenemos que
$$
\left\{
    \begin{array}{c}
        -\tau(s)c(s)=0 \\
        \frac{\tau(s)}{k}-c'(s)
    \end{array}\right.
$$
Tenemos dos casos, $\tau(s)=0$ y $\tau(s)\ne0$. En el segundo caso tenemos que $c(s)=0$ por lo que $c'(s)=0$ y $\frac{\tau(s)}{k}=0$. En ambos casos $\tau(s)=0$. Tenemos que la curva es plana y por estar contenida en una esfera es un arco de circunferencia.\\
Para el recíproco, supongamos que es un aco de circunferencia. Evidentemente está contenida en un esfera con el mismo centro y radio $r$ que la circunferencia. También sabemos que la curvatura es $\frac{1}{r}\forall s$.
\end{document}     