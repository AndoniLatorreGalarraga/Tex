\documentclass{article}
\usepackage[utf8]{inputenc}
\usepackage{amsmath,amssymb,cancel}
\usepackage{mathtools}
\usepackage{graphicx}
\DeclarePairedDelimiter{\ceil}{\lceil}{\rceil}
\DeclarePairedDelimiter{\floor}{\lfloor}{\rfloor}
\title{Los Simpsons}
\author{Grupo 1\\\\Jon Pilarte, Andoni Latorre, Aitor Moreno,\\\\Alberto Ruiz de Ázua, Mariana Zaballa, Mikel Navas}
\date{7 de Noviembre del 2020}
\begin{document}
\maketitle
\begin{center}
    \includegraphics[width=1\textwidth]{figuras/keyart_s32simp.jpg}
\end{center}
\newpage
%
%
%EJERCICIO 1
%
%
\section{}
%ENUNCIADO
Formula el último teorema de Fermat. Estudia brevemente la historia detrás del problema. ¿Por qué la fórmula en los Simpson contradice este teorema?\\\\
%SOLUCION
El teorema de Fermat dice que $\not\exists a,b,c\in\mathbb{N}: a^n+b^n=c^n \quad \forall n\in\mathbb{N}-\{1,2\}$.\\\\
Este teorema fue conjeturado por Pierre de Fermat en 1637, pero no fue demostrado hasta 1995 por Andrew Wiles ayudado por el matemático Richard Taylor. La búsqueda de una demostración estimuló el desarrollo de la teoría algebraica de números en el siglo XIX y la demostración del teorema de la modularidad en el siglo XX.\\\\
Aqui un resumen de la búsqueda de la demostración a este teorema.
\begin{center}
 \begin{tabular}{||c c||} 
 \hline
 Año & Acontecimiento \\ [0.5ex] 
 \hline\hline
 1665 & Muere Fermat sin dejar constancia de su demostración. \\ 
 \hline
 1753 & Leonhard Euler demostró el caso $n = 3$. \\
 \hline
 1825 & Adrien-Marie Legendre demostró el caso para $n = 5$. \\
 \hline
 1839 & Lamé demostró el caso n=7. \\
 \hline
 1843 & Ernst Kummer afirma haber demostrado el teorema pero Dirichlet encuentra un error. \\
 \hline
 1995 & Andrew Wiles publica la demostración del teorema. \\ [1ex] 
 \hline
\end{tabular}
\end{center}
Para probar que $1782^{12}+1841^{12}=1922^{12}$ no contradice el teorema veamos que es falso, para ello, trabajaremos módulo $2$.\\
Por contradicción, asumiendo que es verdad, se tiene que:
$$
\begin{array}{rcll}
1782^{12}+1841^{12} & \equiv & 1922^{12} & (\text{mod }2) \\
0^{12} + 1^{12} & \equiv & 0^{12} \\
0 + 1 & \equiv & 0 \\
1 & \equiv & 0 \\
\end{array}
$$
Lo cual es absurdo y concluye la prueba.
\newpage
\section{}
Resuelve cuántas maneras hay de poner un número natural n como suma de dos enteros que cumplan:\\\\
a. ambos pares y no negativos\\\\
Si $n\equiv1 \quad (\text{mod }2)$, no hay soluciones. Si $n\equiv0 \quad (\text{mod }2)$, las soluciones son:
$$
n=2k \quad k \in \mathbb{N}
$$
$$
\left.\begin{array}{c|c}
    x_1 & x_2 \\ \hline
    2\cdot 0 & 2k \\
    2\cdot 1 & 2(k-1) \\
    \vdots & \vdots \\
    2\cdot i & 2(k-i) \\
    \vdots & \vdots \\
    2k & 0
\end{array} \right\}
\begin{array}{l}
    \text{que son } k+1 \text{ soluciones.}\\
    \text{Si } x_1^{(i)}, x_2^{(i)} \text{ s la solución iésima. Si}\\
    \left\{ \begin{array}{c}
        x_1^{(i)} = x_2^{(j)} \\
        x_2^{(i)} = x_1^{(j)}
    \end{array} \right.\text{, hay } \ceil*{\frac{k+1}{2}} \text{ soluciones.}
\end{array}
$$
b. el primero par y el otro múltiplo de 3, ambos no negativos\\\\
Si $n \equiv 0 \quad (\text{mod }2)$, $n=2k \quad k \in \mathbb{N}$. Las soluciones son:
$$
\left. \begin{array}{c|c}
    x_1 & x_2 \\ \hline
    2 \cdot k & 3\cdot 2 \cdot 0 \\
    2k-3\cdot2 & 3\cdot 2 \cdot 1\\
    \vdots & \vdots \\
    2(k-3i) & 3 \cdot 2 \cdot i \\
    \vdots & \vdots \\
    2k-3\cdot 2 \floor*{\frac{k}{3}} & 3\cdot 2 \floor*{\frac{k}{3}}
\end{array}\right\}
\begin{array}{l}
    \text{que son } \floor*{\frac{k}{3}} +1 \text{ soluciones. Si,} \\
     \left\{ \begin{array}{c}
        x_1^{(i)} = x_2^{(j)} \\
        x_2^{(i)} = x_1^{(j)}
    \end{array} \right. \text{Hay }\ceil*{\frac{\floor*{\frac{k}{3}} +1}{2}} \text{ soluciones.}
\end{array}
$$
Si $n \equiv 1 \quad (\text{mod }2)$, $n=2k+1 \quad k \in \mathbb{N}$. Las soluciones son:
$$
\left. \begin{array}{c|c}
    x_1 & x_2 \\ \hline
    3\cdot 1 & 2k+1-3\cdot 1 = 2k-2\\
    3\cdot 3 & 2k+1-3\cdot 3 = 2k-8 \\
    \vdots & \vdots \\
    3(2i-1) & 2k+1-3(2i-1)=2(k-31+2)\\
    \vdots & \vdots \\
    3(2\floor*{\frac{k+4}{3}}-3) & 2(k-3\floor*{\frac{k+4}{3}}+5)\\
    3(2\floor*{\frac{k+4}{3}}-1) & 0
\end{array}\right\}
\floor*{\frac{k+4}{3}} = \floor*{\frac{k+1}{3}}+1
$$\newpage\noindent
c. ¿y si pudiesen ser negativos? Resolver a y b.\\\\
Para el caso de dos pares. De nuevo, si $n\equiv1 \quad (\text{mod }2)$, no hay soluciones. En cambio, si $n\equiv0 \quad (\text{mod }2)$, las soluciones son
$$
\left. \begin{array}{c|c}
    x_1 & x_2 \\ \hline
    n+0 & 0 \\
    n+2 & -2 \\
    \vdots & \vdots \\
    n+2k & -2k \\
    \vdots & \vdots
\end{array}\right\}\text{infinitas}
$$
Para el caso de uno par y el otro multiplo de tres:\\
Si $n\equiv 0 \quad (\text{mod }2) \quad n=2k\quad k \in \mathbb{N}$
$$
\left.\begin{array}{c|c}
    x_1 & x_2 \\ \hline
    \vdots & \vdots \\
    2k+ 3\cdot 2 \cdot i & 3\cdot 2\cdot -i \\
    \vdots & \vdots \\
    2k+3\cdot 2 & 3\cdot 2\cdot -1 \\
    2k & 3\cdot 2 \cdot 0 \\
    2k-3\cdot 2 & 3\cdot 2 \cdot 1 \\
    \vdots & \vdots \\
    2k- 3\cdot 2 \cdot i & 3\cdot 2\cdot i \\
    \vdots & \vdots
\end{array}\right\}\text{infinitas}
$$
Si $n\equiv 1 \quad (\text{mod }2) \quad n=2k+1 \quad k \in \mathbb{N}$
$$
\left.\begin{array}{c|c}
    x_1 & x_2 \\ \hline
    \vdots & \vdots \\
    3(-2i-1) & 2(k+3i+2) \\
    \vdots & \vdots \\
    3\cdot -1 & 2k+4 \\
    3\cdot 1 & 2k-2 \\
    3\cdot 3 & 2k-8 \\
    \vdots & \vdots \\
    3(2i-1) & 2(k-3i+2) \\
    \vdots & \vdots
\end{array}\right\}\text{infinitas}
$$
d. en cada uno de los casos, ¿cuántas formas hay de descomponer el número natural 12? ¿y el 1922?
$$
\begin{array}{c|c|c}
     & 12 & 1922 \\ \hline
    a & 7 & 962 \\ \hline
    b & 2 & 321
\end{array}
$$
\newpage
\section{}
Los números de Fermat son de la forma 
$$
f_n = 2^{2^{n}} + 1
$$
Fermat conjeturó que todos esos números eran primos, pero años después Euler se encargó de refutar esa conjetura. Adivina el primer valor de n para el que $f_n$ no es primo.
$$
F_1 = 2^2 + 1 = 5 \to \text{primo}
$$
$$
F_2 = 2^{2^2}+1 = 2^4 +1 = 17 \to \text{primo}
$$
$$
F_3 = 2^{2^3}+1 = 2^8 +1 = 257 \quad \floor*{\sqrt{257}}= 25
$$
$$
2 \not| 257, 3 \not| 257, 5, 7,11,13,17,19,23\not| 257 \to \text{primo}
$$
$$
F_4 = 2^{2^4}+1 = 2^16 +1 = 65537 \quad \floor*{\sqrt{65537}}= 256
$$
2, 3, 5, 7, 11, 13, 17, 19, 23, 29, 31, 37, 41, 43, 47, 53, 59, 61, 67, 71, 73, 79, 83, 89, 97, 101, 103, 107, 109, 113, 127, 131, 137, 139, 149, 151, 157, 163, 167, 173, 179, 181, 191, 193, 197, 199, 211, 223, 227, 229, 233, 239, 241, 251 $\not|$ 65537
$$
65537 \to \text{primo}
$$
$$
F_5 = 2^{2^5}+1 = 2^32 +1 = 4294967297
$$
Usaremos el siguiente teorema de Euler,\\\\
La suma de dos números cuyos exponentes son múltiplos de $2$, $a^{2^m}+b^{2^m}$, no admiten más divisores que aquellos de la forma $n\cdot 2^{m+1}+1$.\\\\
Por lo tanto, $2^{2^5}}+1$, donde $m=5$, sólo admite divisores de la forma $n\cdot 2^6+1 = n\cdot 64 + 1$, veamos,
$$
\begin{array}{cccl}
    n=1 & 64\cdot1 + 1 =65 & \to & \text{No es primo} \\
    n=2 & 64\cdot2 + 1 =129 & \to & \text{No es primo} \\
    n=3 & 64\cdot3 + 1 =193 & \to & \text{Es primo pero no divide a }4294967297 \\
    n=4 & 64\cdot4 + 1 =257 & \to & \text{Es primo pero no divide a }4294967297 \\
    n=5 & 64\cdot5 + 1 =321 & \to & \text{No es primo} \\
    n=6 & 64\cdot6 + 1 =385 & \to & \text{No es primo} \\
    n=7 & 64\cdot7 + 1 =449 & \to & \text{Es primo pero no divide a }4294967297 \\
    n=8 & 64\cdot8 + 1 =513 & \to & \text{No es primo} \\
    n=9 & 64\cdot9 + 1 =577 & \to & \text{Es primo pero no divide a }4294967297 \\
    n=10 & 64\cdot10 + 1 =641 & \to & \text{Es primo y divide a }4294967297 \text{ con cociente 6700417}
\end{array}
$$
el cual también es un número primo, $\floor*{\sqrt{6700417}}=2588$ al que, por el teorema anterior, sólo pueden dividirlo números de la forma $64\cdot n +1$. Ya hemoos vsto los $10$ primeros y, de los 30 restantes, sólo $6$ de ellos son primos, $(769,1153,1217,1409,1601,2113)$, pero ninguno divide a 2588.
\newpage
\section{}
Sin embargo, vamos a demostrar que el conjunto de números primos es infinito basándonos en esta familia de números. Veámoslo paso a paso:\\\\
a. Demuestra que todos son impares.
$$
2^{2^{n}} + 1 \equiv 0^{2^{n}} + 1 \equiv 0 + 1 \equiv  1  \quad (\text{mod } 2)
$$
b. Demuestra la relación de recurrencia:
$$
f_0 f_1 f_2 \cdots f_{n-1} = f_{n} - 2
$$
\noindent Para demostrar esta relación de recurrencia lo haremos por inducción.\\\\
Primer probamos el caso base para n = 1:
$$
f_0 = f_1 - 2
$$
$$
2^{2^{0}}+1=2^{1}+1=3=5-2=(4+1)-2=(2^2 +1)-2 = (2^{2^{1}} +1)-2 
$$
Ahora que sabemos que se cumple el caso base, utilizamos como hipótesis que se cumple para n-1 y probamos que se cumple para n.
$$
f_0 f_1 f_2 \cdots f_{n-1} = f_n - 2
$$
$$
f_0 f_1 f_2 \cdots f_{n-1} f_n = (f_n - 2)f_n =
$$
$$
{f_n}^2 + 2f_n = 
$$
$$
(2^{2^{n}}+1)^2 -2(2^{2^{n}}+1)=
$$
$$
2^{2^{n+1}}\cancel{+2^{2^{n+1}}}+1 \cancel{-2^{2^{n+1}}}-2 =
$$
$$
(2^{2^{n+1}}+1)-2 =
$$
$$
f_{n+1}-2
$$
c. Demuestra que ningún número de Fermat $f_n$ es divisible por ninguno de los factores que forman los números de Fermat anteriores a él. Es decir, $f_n$ es coprimo
con $f_1,f_2,...,f_{n-1}$.\\\\
$$
f_n \equiv 0 \quad (\text{mod } f_i) \quad 0\leq i < n
$$
$$
f_n=\underbrace{f_0 f_1 f_2 \cdots f_i \cdots f_n}_0 +2 \equiv 0\quad (\text{mod } f_i)
$$
$$
f_n \equiv 2 \equiv 0 \quad (\text{mod } f_i)
$$
Como 2 es primo $f_i = 2$, lo cual es absurdo por \textit{a)} y concluye la prueba.\newpage\noindent
d. Demuestra que el conjunto de los números primos es infinito, utilizando el resultado anterior.\\\\
Del apartado c se deduce que los números de Fermat son coprimos dos a dos. Por lo tanto cada número de Fermat tiene al menos un factor primo que no está presente en la factorización de cualquier otro número de Fermat. Al haber infinitos números de Fermat, hay infinitos primos.\\\\
\section{}
¿Hay contradicción entre los dos ejercicios anteriores? ¿Por qué?\\\\
No, no hay contradicción, que exista un numero de Fermat que sea compuesto no impide que este sea coprimo con todos los demás números de Fermat.
\end{document}