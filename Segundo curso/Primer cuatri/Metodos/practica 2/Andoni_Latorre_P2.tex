\documentclass{article}
\usepackage[utf8]{inputenc}
\usepackage{graphicx}
\usepackage[spanish]{babel}
\usepackage{amsmath}
\usepackage{vmargin}
\setpapersize{A4}
\setmargins{2.5cm}       % margen izquierdo
{1.5cm}                        % margen superior
{16.5cm}                      % anchura del texto
{23.42cm}                    % altura del texto
{10pt}                           % altura de los encabezados
{1cm}                           % espacio entre el texto y los encabezados
{0pt}                             % altura del pie de página
{2cm}                           % espacio entre el texto y el pie de página
\title{Método Doolittle}
\author{Andoni Latorre Galarraga}
\date{}

\begin{document}
\maketitle
\noindent Si tenemos la factorización LU,
$$
\begin{pmatrix}
a_{11} & a_{12} & a_{13} & \cdots & a_{1n}\\ 
a_{21} & a_{22} & a_{23} & \cdots & a_{2n}\\ 
a_{31} & a_{32} & a_{33} & \cdots & a_{3n}\\ 
\vdots & \vdots & \vdots & \ddots & \vdots\\ 
a_{n1} & a_{n2} & a_{n3} & \cdots & a_{nn}
\end{pmatrix}
=
\begin{pmatrix}
1 & 0 & 0 & \cdots & 0\\ 
l_{21} & 1 & 0 & \cdots & 0\\ 
l_{31} & l_{32} & 1 & \cdots & 0\\ 
\vdots & \vdots & \vdots & \ddots & \vdots\\ 
l_{n1} & l_{n2} & l_{n3} & \cdots & 1
\end{pmatrix}
\begin{pmatrix}
u_{11} & u_{12} & u_{13} & \cdots & u_{1n}\\ 
0 & u_{22} & u_{23} & \cdots & u_{2n}\\ 
0 & 0 & u_{33} & \cdots & u_{3n}\\ 
\vdots & \vdots & \vdots & \ddots & \vdots\\ 
0 & 0 & 0 & \cdots & u_{nn}
\end{pmatrix}
$$
Para obtener la primera fila de $A$ multiplicamos la primera fila de $L$ y la $i$-ésima columna de $U$, se tiene
$$
a_{1i}=u_{1i}
$$
Para obtener la segunda fila de $A$ multiplicamos la segunda fila de $L$ y la $i$-ésima columna de $U$, se tiene
$$
a_{2i}=l_{21}u_{1i}+u_{2i} \quad u_{21} = a_{21}-l_{21}u_{i1}
$$
En el caso $i=1$, se tiene $a_{21}=l_{21}u_{11}$ y se obtiene $l_{21}=\frac{a_{21}}{u_{11}}$.
$$
\vdots
$$
Para obtener la $j$-ésima fila de $A$ multiplicamos la $j$-ésima fila de $L$ y la $i$-ésima columna de $U$, se tiene\\\\
$j>i$:
$$
a_{ji}=l_{j1}u_{1i}+l_{j2}u_{2i}+\cdots+l_{ji}u_{ii} \quad l_{ji}=\frac{a_{ji}-l_{j1}u_{1i}-l_{j2}u_{2i}-\cdots-l_{j,i-1}u_{i-1,i}}{u_{ii}}
$$
$j\le i$:
$$
a_{ji}=l_{j1}u_{1i}+l_{j2}u_{2i}+\cdots+l_{j,j-1}u_{j-1,i}+u_{ji} \quad u_{ji}=a_{ji}-l_{j1}u_{1i}-l_{j2}u_{2i}-\cdots-l_{j,j-1}u_{j-1,i}
$$
$$
\vdots
$$
Para obtener la última fila de $A$ multiplicamos la última fila de $L$ y la $i$-ésima columna de $U$, se tiene\\\\
$i\ne n$:
$$
a_{ni}=l_{n1}u_{1i}+l_{n2}u_{2i}+\cdots+l_{ni}u_{ii} \quad l_{ni}=\frac{a_{ni}-l_{n1}u_{1i}-l_{n2}u_{2i}-\cdots-l_{n,i-1}u_{i-1,i}}{u_{ii}}
$$
$i=n$:
$$
a_{nn}=l_{n1}u_{1n}+l_{n2}u_{2n}+\cdots+l_{n,n-1}u_{n-1,i}+u_{nn} \quad u_{nn}=a_{nn}-l_{n1}u_{1n}-l_{n2}u_{2n}-\cdots-l_{n,n-1}u_{n-1,n}
$$
Para calcular el coste, vemos que primero se va desde $j=1$ hasta $j=n$, se tiene $\sum_{j=1}^n$. Luego se va desde $i=1$ hasta $i=j-1$ y desde $i=j$ hasta $i=n$ por separado, se tiene, $\sum_{j=1}^n \left( \sum_{i=1}^{j-1} +\sum_{i=j}^n\right)$. Se hacen $i-1$ sumas y multiplicaciones dentro del for($L$) y luego se hace una resta y una división, se tiene $\sum_{j=1}^n \left( \sum_{i=1}^{j-1} (im + is) +\sum_{i=j}^n\right)$. En el otro for($U$), se hacen $j-1$ sumas y multiplicaciones y luego una resta, se tiene $\sum_{j=1}^n \left( \sum_{i=1}^{j-1} (im + is) +\sum_{i=j}^n ((j-1)m+js)\right)$
$$
=\sum_{j=1}^n \left( (m+s)\sum_{i=1}^{j-1}i +(m+s)\sum_{i=j}^nj-\sum_{i=j}^nm\right)
=\sum_{j=1}^n \left( (m+s)\sum_{i=1}^{n}i -m(n-j+1)\right)
$$
$$
=\sum_{j=1}^n \left( (m+s)\frac{n(n+1)}{2} -m(n-j+1)\right)
=\sum_{j=1}^n \left( m\left(\frac{n(n+1)}{2}+n-j+1\right)+s\frac{n(n+1)}{2} \right)
$$
$$
=m\sum_{j=1}^n\left(\frac{n(n+1)}{2}+n-j+1\right) + s\sum_{j=1}^n \frac{n(n+1)}{2}
=m\sum_{j=1}^n\left(\frac{n^2 + 3n + 2}{2}-j\right) + s \frac{n^3+n^2}{2}
$$
$$
=m\left(\frac{n^3 + 3n^2 + 2n}{2}-\sum_{j=1}^nj\right) + s \frac{n^3+n^2}{2}
=m\left(\frac{n^3 + 3n^2 + 2n}{2}-\frac{n^2+n}{2}\right) + s \frac{n^3+n^2}{2}
$$
$$
=m\frac{n^3 + 2n^2 + n}{2} + s \frac{n^3+n^2}{2}
$$
El algoritmo hace $\frac{n^3 + 2n^2 + n}{2}$ multiplicaciones y $\frac{n^3+n^2}{2}$ sumas. En total $\frac{2n^3 + 3n^2 + n}{2}$ operaciones.
\end{document}