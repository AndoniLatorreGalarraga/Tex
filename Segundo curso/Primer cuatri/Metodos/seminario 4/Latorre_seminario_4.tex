\documentclass{article}
\usepackage[utf8]{inputenc}

\title{Seminario 4}
\author{Andoni Latorre Galarraga}
\date{}

\begin{document}

\maketitle
Sustituimos $x_n = \epsilon_n + \alpha$ en $x_{n+1} = x_n -\frac{x_n - x_{n-1}}{f(x_n)-f(x_{n-1})}f(x_n)$.
$$
\epsilon_{n+1} = \epsilon_n - \frac{\epsilon_n-\epsilon_{n+1}}{f(\epsilon_{n+1} + \alpha)-}f(\epsilon_n + \alpha)
$$
Sustituimos el desrrollo de Taylor $f(x)\approx f^\prime(\alpha) (x-\alpha) + \frac{1}{2} f^{\prime\prime}(\alpha)(x-\alpha)^2$.
$$
f(\epsilon_n+\alpha)\approx \epsilon_n f^\prime(\alpha)(1+\frac{f^{\prime\prime}(\alpha)}{2f^\prime(\alpha)}
$$
$$
\epsilon_{n+1} \approx \epsilon_n - \frac{\epsilon_n(1+\frac{f^{\prime\prime}(\alpha)}{2f^\prime(\alpha)}\epsilon_n)}{1+ \frac{f^{\prime\prime}(\alpha)}{2f^\prime(\alpha)}(e_n + \epsilon_{n+1})}
=
\frac{\frac{f^{\prime\prime}(\alpha)}{2f^\prime(\alpha)}\epsilon_n \epsilon_{n+1}}{1+ \frac{f^{\prime\prime}(\alpha)}{2f^\prime(\alpha)}(e_n + \epsilon_{n+1})}
$$
$$
\approx \frac{f^{\prime\prime}(\alpha)}{2f^\prime(\alpha)}\epsilon_n \epsilon_{n+1}
$$
Ahora:
$$
|\epsilon_{n+1}| = \lambda |\epsilon_n|^p \iff |\frac{f^{\prime\prime}(\alpha)}{2f^\prime(\alpha)}||\epsilon_n||\epsilon_{n+1}| \approx \lambda |\epsilon_n|^p
$$
$$
|\epsilon_n| \approx \left( \frac{|\frac{f^{\prime\prime}(\alpha)}{2f^\prime(\alpha)}|}{\lambda}\right)^{\frac{1}{p-1}} |\epsilon_{n+1}|^{\frac{1}{p-1}}
\Rightarrow
\lambda = \left( \frac{\frac{|f^{\prime\prime}(\alpha)}{2f^\prime(\alpha)}|}{\lambda}\right)^{\frac{1}{p-1}}
$$
Entonces, $p = \frac{1}{p-1}$. Por lo que $p = \frac{1+ \sqrt{5}}{2}$, que es la razón aurea.
\end{document}