\documentclass{article}
\usepackage[utf8]{inputenc}
\usepackage{amssymb,amsmath,geometry}
\title{Tercer Seminario de Álgebra II}
\author{Andoni Latorre Galarraga}
\date{}

\begin{document}

\maketitle
\newgeometry{left=1cm}
\noindent\textbf{5.}
$$
A=
\left(\begin{array}{ccccccccc}
     2 & 3 & 2 & 0 & 0 &-1 & 0 & 1 & 1\\
    -1 &-1 &-1 & 0 & 0 & 0 & 0 &-1 &-1\\
     1 & 2 & 2 & 0 & 0 & 0 & 0 & 1 & 1\\
     0 & 0 & 0 & 1 & 1 & 2 & 0 & 2 & 1\\
     0 & 0 & 0 & 0 & 1 & 1 & 0 & 1 & 0\\
     0 & 0 & 0 & 0 & 0 & 1 & 0 & 0 & 0\\
     0 & 0 & 0 & 0 & 0 & 0 & 1 & 2 & 1\\
     0 & 0 & 0 & 0 & 0 & 0 & 0 & 1 & 0\\
     0 & 0 & 0 & 0 & 0 & 0 & 0 & 0 & 1
\end{array}\right)
$$
Calculamos el polinomio característico:
$$
|\lambda I_9 -A|=
\lambda^9-9\lambda^8+36\lambda^7-84\lambda^6+126\lambda^5-126\lambda^4+84\lambda^3-36\lambda^2+9\lambda-1=(\lambda-1)^9
$$
Se tiene que 1 es la única raiz con multiplicidad de 9.\\
Calculamos el primer subespacios propios generalizado:
$$
\left(
\left(\begin{array}{ccccccccc}
     2 & 3 & 2 & 0 & 0 &-1 & 0 & 1 & 1\\
    -1 &-1 &-1 & 0 & 0 & 0 & 0 &-1 &-1\\
     1 & 2 & 2 & 0 & 0 & 0 & 0 & 1 & 1\\
     0 & 0 & 0 & 1 & 1 & 2 & 0 & 2 & 1\\
     0 & 0 & 0 & 0 & 1 & 1 & 0 & 1 & 0\\
     0 & 0 & 0 & 0 & 0 & 1 & 0 & 0 & 0\\
     0 & 0 & 0 & 0 & 0 & 0 & 1 & 2 & 1\\
     0 & 0 & 0 & 0 & 0 & 0 & 0 & 1 & 0\\
     0 & 0 & 0 & 0 & 0 & 0 & 0 & 0 & 1
\end{array}\right)-I_9
\right)
\left(\begin{array}{c}
    x_1\\
    x_2 \\
    x_3 \\
    x_4 \\
    x_5 \\
    x_6 \\
    x_7 \\
    x_8 \\
    x_9
\end{array}\right)
=
\left(\begin{array}{c}
    0\\
    0 \\
    0 \\
    0 \\
    0 \\
    0 \\
    0 \\
    0 \\
    0
\end{array}\right)
$$
$$
\Rightarrow
\left(\begin{array}{c}
    x_1\\
    x_2 \\
    x_3 \\
    x_4 \\
    x_5 \\
    x_6 \\
    x_7 \\
    x_8 \\
    x_9
\end{array}\right)
=
\left(\begin{array}{c}
    x_3- \frac{3x_9}{2}\\
    x_2 \\
    \frac{x_9}{2}-x_2 \\
    x_4 \\
    -x_9 \\
    \frac{x_9}{2} \\
    x_7 \\
    \frac{-x_9}{2}\\%
    x_9
\end{array}\right)
$$
El subespacio es de de dimensión 4, y como la multiplicidad es 9 tenemos que calcular el segundo subespacio:
$$
\left(
\left(\begin{array}{ccccccccc}
     2 & 3 & 2 & 0 & 0 &-1 & 0 & 1 & 1\\
    -1 &-1 &-1 & 0 & 0 & 0 & 0 &-1 &-1\\
     1 & 2 & 2 & 0 & 0 & 0 & 0 & 1 & 1\\
     0 & 0 & 0 & 1 & 1 & 2 & 0 & 2 & 1\\
     0 & 0 & 0 & 0 & 1 & 1 & 0 & 1 & 0\\
     0 & 0 & 0 & 0 & 0 & 1 & 0 & 0 & 0\\
     0 & 0 & 0 & 0 & 0 & 0 & 1 & 2 & 1\\
     0 & 0 & 0 & 0 & 0 & 0 & 0 & 1 & 0\\
     0 & 0 & 0 & 0 & 0 & 0 & 0 & 0 & 1
\end{array}\right)-I_9
\right)^2
\left(\begin{array}{c}
    x_1\\
    x_2 \\
    x_3 \\
    x_4 \\
    x_5 \\
    x_6 \\
    x_7 \\
    x_8 \\
    x_9
\end{array}\right)
=
\left(\begin{array}{c}
    0\\
    0 \\
    0 \\
    0 \\
    0 \\
    0 \\
    0 \\
    0 \\
    0
\end{array}\right)
$$
$$
\Rightarrow
\left(\begin{array}{c}
    x_1 \\
    x_2 \\
    x_3 \\
    x_4 \\
    x_5 \\
    x_6 \\
    x_7 \\
    x_8 \\
    x_9
\end{array}\right)
=
\left(\begin{array}{c}
    x_1 \\
    x_2 \\
    -x_2-x_8 \\
    x_4 \\
    x_5 \\
    -x_8 \\
    x_7 \\
    x_8 \\
    x_9
\end{array}\right)
$$
El subespacio es de de dimensión 7, y como la multiplicidad es 9 tenemos que calcular el tercer subespacio:
$$
\left(
\left(\begin{array}{ccccccccc}
     2 & 3 & 2 & 0 & 0 &-1 & 0 & 1 & 1\\
    -1 &-1 &-1 & 0 & 0 & 0 & 0 &-1 &-1\\
     1 & 2 & 2 & 0 & 0 & 0 & 0 & 1 & 1\\
     0 & 0 & 0 & 1 & 1 & 2 & 0 & 2 & 1\\
     0 & 0 & 0 & 0 & 1 & 1 & 0 & 1 & 0\\
     0 & 0 & 0 & 0 & 0 & 1 & 0 & 0 & 0\\
     0 & 0 & 0 & 0 & 0 & 0 & 1 & 2 & 1\\
     0 & 0 & 0 & 0 & 0 & 0 & 0 & 1 & 0\\
     0 & 0 & 0 & 0 & 0 & 0 & 0 & 0 & 1
\end{array}\right)-I_9
\right)^3
\left(\begin{array}{c}
    x_1\\
    x_2 \\
    x_3 \\
    x_4 \\
    x_5 \\
    x_6 \\
    x_7 \\
    x_8 \\
    x_9
\end{array}\right)
=
\left(\begin{array}{c}
    0\\
    0 \\
    0 \\
    0 \\
    0 \\
    0 \\
    0 \\
    0 \\
    0
\end{array}\right)
$$
$$
\Rightarrow
\left(\begin{array}{c}
    x_1 \\
    x_2 \\
    x_3 \\
    x_4 \\
    x_5 \\
    x_6 \\
    x_7 \\
    x_8 \\
    x_9
\end{array}\right)
=
\left(\begin{array}{c}
    x_1 \\
    x_2 \\
    x_3 \\
    x_4 \\
    x_5 \\
    x_6 \\
    x_7 \\
    x_8 \\
    x_9
\end{array}\right)
$$
Ahora tenemos:
$$
\begin{array}{c|cc}
    V_3(1) \text{ dimensión }9 & v_1 & v_2\\
    V_2(1) \text{ dimensión }7 \\
    V_1(1) \text{ dimensión }4
\end{array}
$$
Donde donde $\{\overline{v_1},\overline{v_2}\}$ es base de $V_3(1)/V_2(1)$. Completamos la una base de $V_2(1)$ a una de $V_3(1)$ de manera que $v_1\notin V_2(1)$ y $v_2 \notin V_2(1)+<v_1>$.
$$
\beta_{V_2(1)}=
\left\{
\left(\begin{array}{c}
    1 \\
    0 \\
    0 \\
    0 \\
    0 \\
    0 \\
    0 \\
    0 \\
    0
\end{array}\right)
,
\left(\begin{array}{c}
    0 \\
    1 \\
    -1 \\
    0 \\
    0 \\
    0 \\
    0 \\
    0 \\
    0
\end{array}\right)
,
\left(\begin{array}{c}
    0 \\
    0 \\
    0 \\
    1 \\
    0 \\
    0 \\
    0 \\
    0 \\
    0
\end{array}\right)
,
\left(\begin{array}{c}
    0 \\
    0 \\
    0 \\
    0 \\
    1 \\
    0 \\
    0 \\
    0 \\
    0
\end{array}\right)
,
\left(\begin{array}{c}
    0 \\
    0 \\
    0 \\
    0 \\
    0 \\
    0 \\
    1 \\
    0 \\
    0
\end{array}\right)
,
\left(\begin{array}{c}
    0 \\
    0 \\
    -1 \\
    0 \\
    0 \\
    -1 \\
    0 \\
    1 \\
    0
\end{array}\right)
,
\left(\begin{array}{c}
    0 \\
    0 \\
    0 \\
    0 \\
    0 \\
    0 \\
    0 \\
    0 \\
    1
\end{array}\right)
\right\}
$$
Podemos completar con:
$$
v_1=
\left(\begin{array}{c}
    0 \\
    0 \\
    0 \\
    0 \\
    0 \\
    1 \\
    0 \\
    1 \\
    0
\end{array}\right)
v_2 =
\left(\begin{array}{c}
    0 \\
    1 \\
    1 \\
    0 \\
    0 \\
    0 \\
    0 \\
    0 \\
    0
\end{array}\right)
$$
Ahora:
$$
\begin{array}{c|cc}
    V_3(1) \text{ dimensión }9 & (0,0,0,0,0,1,0,1,0) & (0,1,1,0,0,0,0,0,0)\\
    V_2(1) \text{ dimensión }7 & (A-I_9)v_1 & (A-I_9)v_2\\
    V_1(1) \text{ dimensión }4 & (A-I_9)((A-I_9)v_1) & (A-I_9)((A-I_9)v_2)
\end{array}
$$
$$
\begin{array}{c|cc}
    V_3(1) \text{ dimensión }9 & (0,0,0,0,0,1,0,1,0) & (0,1,1,0,0,0,0,0,0)\\
    V_2(1) \text{ dimensión }7 & (0,-1,1,4,2,0,2,0,0) & (5,-3,3,0,0,0,0,0,0)\\
    V_1(1) \text{ dimensión }4 & (-1,1,-1,2,0,0,0,0,0) & (2,-2,2,0,0,0,0,0,0)
\end{array}
$$
$$
\begin{array}{c|cccc}
    V_3(1) \text{ dimensión }9 & (0,0,0,0,0,1,0,1,0) & (0,1,1,0,0,0,0,0,0)\\
    V_2(1) \text{ dimensión }7 & (0,-1,1,4,2,0,2,0,0) & (5,-3,3,0,0,0,0,0,0) & (0,0,0,0,0,0,0,0,1)\\
    V_1(1) \text{ dimensión }4 & (-1,1,-1,2,0,0,0,0,0) & (2,-2,2,0,0,0,0,0,0) & (1,-1,1,1,0,0,1,0,0)
\end{array}
$$
$$
\begin{array}{c|cccc}
    V_3(1) \text{ dimensión }9 & (0,0,0,0,0,1,0,1,0) & (0,1,1,0,0,0,0,0,0)\\
    V_2(1) \text{ dimensión }7 & (0,-1,1,4,2,0,2,0,0) & (5,-3,3,0,0,0,0,0,0) & (0,0,0,0,0,0,0,0,1)\\
    V_1(1) \text{ dimensión }4 & (-1,1,-1,2,0,0,0,0,0) & (2,-2,2,0,0,0,0,0,0) & (1,-1,1,1,0,0,1,0,0) & (-1,0,\frac{1}{2},0,-1,\frac{1}{2},0,\frac{-1}{2},1)
\end{array}
$$
Los bloques de Jordan deverían de tener tamaño $3,3,2,1$.
Se tiene que:
\begingroup
\tiny
\begin{equation*}
\underbrace{
\left(\begin{array}{ccccccccc}
       0&	   0&	   0&	  0&	   0&	   1/2&	   0&	  1/2&	0\\
       0&	   0&	   0&	  0&	 1/2&	   1/2&	   0&	 -1/2&	0\\
       0&	   0&	   0&	1/2&	-1/2&	  -1/2&	-1/2&	  1/2&	0\\
       0&	 1/2&	 1/2&	  0&	   0&	  -1/4&	   0&	  1/4&	0\\
     1/2&	 1/4&	-1/4&	  0&	 1/4&	   7/8&	   0&	 -7/8&	0\\
    -3/4&	-5/8&	 5/8&	1/4&	-3/8&	-23/16&	-3/4&	23/16&	0\\
       0&	   0&	   0&	  0&	   0&	    -1&	   0&	    1&	1\\
       0&	   0&	   0&	  0&	  -1&	    -1&	   1&	    1&	0\\
       0&	   0&	   0&	  0&	   0&	     1&	   0&	   -1&	0
\end{array}\right)}_{P^{-1}}
\underbrace{
\left(\begin{array}{ccccccccc}
     2 & 3 & 2 & 0 & 0 &-1 & 0 & 1 & 1\\
    -1 &-1 &-1 & 0 & 0 & 0 & 0 &-1 &-1\\
     1 & 2 & 2 & 0 & 0 & 0 & 0 & 1 & 1\\
     0 & 0 & 0 & 1 & 1 & 2 & 0 & 2 & 1\\
     0 & 0 & 0 & 0 & 1 & 1 & 0 & 1 & 0\\
     0 & 0 & 0 & 0 & 0 & 1 & 0 & 0 & 0\\
     0 & 0 & 0 & 0 & 0 & 0 & 1 & 2 & 1\\
     0 & 0 & 0 & 0 & 0 & 0 & 0 & 1 & 0\\
     0 & 0 & 0 & 0 & 0 & 0 & 0 & 0 & 1
\end{array}\right)}_{A}
\underbrace{
\left(\begin{array}{ccccccccc}
    0&	 0&	-1&	0&	 5&	 2&	0&	 1&	  -1\\
    0&	-1&	 1&	1&	-3&	-2&	0&	-1&	   0\\
    0&	 1&	-1&	1&	 3&	 2&	0&	 1&	 1/2\\
    0&	 4&	 2&	0&	 0&	 0&	0&	 1&	   0\\
    0&	 2&	 0&	0&	 0&	 0&	0&	 0&	  -1\\
    1&	 0&	 0&	0&	 0&	 0&	0&	 0&	 1/2\\
    0&	 2&	 0&	0&	 0&	 0&	0&	 1&	   0\\
    1&	 0&	 0&	0&	 0&	 0&	0&	 0&	-1/2\\
    0&	 0&	 0&	0&	 0&	 0&	1&	 0&	   1\\
\end{array}\right)}_{P}
\end{equation*}
\endgroup
\normal
$$
=
\underbrace{
\left(\begin{array}{ccccccccc}
    1&	0&	0&	0&	0&	0&	0&	0&	0&
    1&	1&	0&	0&	0&	0&	0&	0&	0&
    0&	1&	1&	0&	0&	0&	0&	0&	0&
    0&	0&	0&	1&	0&	0&	0&	0&	0&
    0&	0&	0&	1&	1&	0&	0&	0&	0&
    0&	0&	0&	0&	1&	1&	0&	0&	0&
    0&	0&	0&	0&	0&	0&	1&	0&	0&
    0&	0&	0&	0&	0&	0&	1&	1&	0&
    0&	0&	0&	0&	0&	0&	0&	0&	1&
\end{array}\right)}_{J}
$$
Con los bloques de Jordan de tamaño $3,3,2,1$ que se esperaban.
\newpage
\textbf{8.}
a)
$$
|A|=\left| \begin{array}{ccc}
     2 & 2 & 1 \\
     2 & 3 & 2 \\
     1 & 1 & 2 
\end{array}\right| = 3
$$
$$
|B|=\left| \begin{array}{ccc}
     2 & 0 & -3 \\
     1 & 2 & -2 \\
     -1 & -1 & 3 
\end{array}\right| = 5
$$
El determinante es diferente por lo que no son semejantes. Ya que, A y B semejantes $\Rightarrow$ det(A) = det(B).\\\\
b) Veamos que ambass matrices tienen la misma forma canónica de Jordan.
$$
C=
\left(\begin{matrix}
-1 & 5 & -1 & 3 \\
0 & -2 & 1 & 0 \\
0 & -1 & 0 & 0 \\
0 & -4 & 1 & 2
\end{matrix}\right)
$$
Calculamos los valores propios:
$$
\left|\begin{matrix}
\lambda+1 & -5 & 1 & -3 \\
0 & \lambda+2 & -1 & 0 \\
0 & 1 & \lambda & 0 \\
0 & 4 & -1 & \lambda-2
\end{matrix}\right|
= (\lambda+1)^3(\lambda-2)
$$
$$
dim\left(ker\left(\left(\begin{matrix}
-1 & 5 & -1 & 3 \\
0 & -2 & 1 & 0 \\
0 & -1 & 0 & 0 \\
0 & -4 & 1 & 2
\end{matrix}\right)+I_4\right)\right)=1
$$
$$
dim\left( ker\left(\left(\left(\begin{matrix}
-1 & 5 & -1 & 3 \\
0 & -2 & 1 & 0 \\
0 & -1 & 0 & 0 \\
0 & -4 & 1 & 2
\end{matrix}\right)+I_4\right)^2\right)\right)=2
$$
$$
dim\left( ker\left(\left(\left(\begin{matrix}
-1 & 5 & -1 & 3 \\
0 & -2 & 1 & 0 \\
0 & -1 & 0 & 0 \\
0 & -4 & 1 & 2
\end{matrix}\right)+I_4\right)^3\right)\right)=3
$$
$$
dim\left(ker\left(\left(\begin{matrix}
-1 & 5 & -1 & 3 \\
0 & -2 & 1 & 0 \\
0 & -1 & 0 & 0 \\
0 & -4 & 1 & 2
\end{matrix}\right)-2 I_4\right)\right)=1
$$
Por lo que la forma canónica de Jordan es:
$$
\left(\begin{matrix}
-1 & 0 & 0 & 0 \\
1 & -1 & 0 & 0 \\
0 & 1 & -1 & 0 \\
0 & 0 & 0 & 2
\end{matrix}\right)
$$
Ahora la otra matriz,
$$
D=
\left(\begin{matrix}
-1 & 4 & -1 & -2 \\
0 & -1 & 1 & -1 \\
0 & -3 & 0 & 2 \\
0 & -3 & 1 & 1
\end{matrix}\right)
$$
Calculamos los valores propios:
$$
\left|\begin{matrix}
\lambda+1 & -4 & 1 & 2 \\
0 & \lamda+1 & -1 & 1 \\
0 & 3 & \lambda & -2 \\
0 & 3 & -1 & \lamda-1
\end{matrix}\right|
= (\lambda+1)^3(\lambda-2)
$$
$$
dim\left(ker\left(\left(\begin{matrix}
-1 & 4 & -1 & -2 \\
0 & -1 & 1 & -1 \\
0 & -3 & 0 & 2 \\
0 & -3 & 1 & 1
\end{matrix}\right)+I_4\right)\right)=1
$$
$$
dim\left( ker\left(\left(\left(\begin{matrix}
-1 & 4 & -1 & -2 \\
0 & -1 & 1 & -1 \\
0 & -3 & 0 & 2 \\
0 & -3 & 1 & 1
\end{matrix}\right)+I_4\right)^2\right)\right)=2
$$
$$
dim\left( ker\left(\left(\left(\begin{matrix}
-1 & 4 & -1 & -2 \\
0 & -1 & 1 & -1 \\
0 & -3 & 0 & 2 \\
0 & -3 & 1 & 1
\end{matrix}\right)+I_4\right)^3\right)\right)=3
$$
$$
dim\left(ker\left(\left(\begin{matrix}
-1 & 4 & -1 & -2 \\
0 & -1 & 1 & -1 \\
0 & -3 & 0 & 2 \\
0 & -3 & 1 & 1
\end{matrix}\right)-2 I_4\right)\right)=1
$$
Por lo que la forma canónica de Jordan es:
$$
\left(\begin{matrix}
-1 & 0 & 0 & 0 \\
1 & -1 & 0 & 0 \\
0 & 1 & -1 & 0 \\
0 & 0 & 0 & 2
\end{matrix}\right)
$$
Que son la misma y por lo tanto son semejantes.
\end{document}
