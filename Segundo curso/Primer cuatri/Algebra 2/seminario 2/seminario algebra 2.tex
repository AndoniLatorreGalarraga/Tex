\documentclass{article}
\usepackage[utf8]{inputenc}
\usepackage{amssymb}
\title{Segundo Seminario de Álgebra II}
\author{Andoni Latorre Galarraga}
\date{}

\begin{document}

\maketitle
\noindent \textbf{1.}\\
%
%
%a)
%
%
\textit{a)}
Escribimos la matriz correspondiente al endomorfismo:\\
$$
\left( \begin{array}{ccc}
    5 & 2 & -3\\
    4 & 5 & -4\\
    6 & 4 & -4
\end{array} \right)
$$
Calculamos el polinomio caraterístico:
$$
c_A(x) = |xI-A| = 
\left| \begin{array}{ccc}
    x-5 & -2 & 3\\
    -4 & x-5 & 4\\
    -6 & -4 & x+4
\end{array} \right| =
x^3-6x^2+11x-6 = (x-1)(x-2)(x-3)
$$
Se tiene que los valores propios son $1,2,3$. Calculamos los subespacios asociados:\
Cuando $\lambda = 1$:
$$
\left( \begin{array}{ccc}
    5 & 2 & -3\\
    4 & 5 & -4\\
    6 & 4 & -4
\end{array} \right)
\left( \begin{array}{c}
      x \\
      y \\
      z
\end{array} \right) = 1
\left( \begin{array}{c}
      x \\
      y \\
      z
\end{array} \right)
$$
$$
\Leftrightarrow \left\{ \begin{array}{c}
     4x+2y-3z =0  \\
     4x+4y-4z =0  \\
     6x+4y-5z =0
\end{array} \right.
\Leftrightarrow \left( \begin{array}{c}
      x \\
      y \\
      z
\end{array} \right) =
\left( \begin{array}{c}
      \frac{\mu}{2} \\
      \frac{\mu}{2} \\
      \mu
\end{array} \right) \forall \mu \in \mathbb{R}
$$
Cuando $\lambda = 2$:
$$
\left( \begin{array}{ccc}
    5 & 2 & -3\\
    4 & 5 & -4\\
    6 & 4 & -4
\end{array} \right)
\left( \begin{array}{c}
      x \\
      y \\
      z
\end{array} \right) = 2
\left( \begin{array}{c}
      x \\
      y \\
      z
\end{array} \right)
$$
$$
\Leftrightarrow \left\{ \begin{array}{c}
     3x+2y-3z =0  \\
     4x+3y-4z =0  \\
     6x+4y-6z =0
\end{array} \right.
\Leftrightarrow \left( \begin{array}{c}
      x \\
      y \\
      z
\end{array} \right) =
\left( \begin{array}{c}
      \mu \\
      0 \\
      \mu
\end{array} \right) \forall \mu \in \mathbb{R}
$$
Cuando $\lambda = 3$:
$$
\left( \begin{array}{ccc}
    5 & 2 & -3\\
    4 & 5 & -4\\
    6 & 4 & -4
\end{array} \right)
\left( \begin{array}{c}
      x \\
      y \\
      z
\end{array} \right) = 3
\left( \begin{array}{c}
      x \\
      y \\
      z
\end{array} \right)
$$
$$
\Leftrightarrow \left\{ \begin{array}{c}
     2x+2y-3z =0  \\
     4x+2y-4z =0  \\
     6x+4y-7z =0
\end{array} \right.
\Leftrightarrow \left( \begin{array}{c}
      x \\
      y \\
      z
\end{array} \right) =
\left( \begin{array}{c}
      \frac{\mu}{2} \\
      \mu \\
      \mu
\end{array} \right) \forall \mu \in \mathbb{R}
$$
Como la multiplicidad algebraica del cada valor propio vcomo raiz del polinomio caracteristico coincide con la dimensión del subespacio propio asociado a dicho valor propio en todos los casos, se tiene que el endomorfismo es diagonalizable. Se tiene:
$$
P^{-1}AP =
\left( \begin{array}{ccc}
    -4 & -2 & 4\\
    2 & 0 & -1\\
    2 & 2 & -2
\end{array} \right)
\left( \begin{array}{ccc}
    5 & 2 & -3\\
    4 & 5 & -4\\
    6 & 4 & -4
\end{array} \right)
\left( \begin{array}{ccc}
    \frac{1}{2} &  1 & \frac{1}{2}\\
    \frac{1}{2} &  0 & 1\\
    1 & 1 & 1
\end{array} \right) =
\left( \begin{array}{ccc}
    1 & 0 & 0\\
    0 & 2 & 0\\
    0 & 0 & 3
\end{array} \right)
$$\\
%
%
%b)
%
%
\textit{b)}
Escribimos la matriz correspondiente al endomorfismo:\\
$$
\left( \begin{array}{ccc}
    8 & 15 & -36\\
    8 & 21 & -46\\
    5 & 12 & -27
\end{array} \right)
$$
Calculamos el polinomio caraterístico:
$$
c_A(x) = |xI-A| = 
\left| \begin{array}{ccc}
    x-8 & -15 & 36\\
    -8 & x-21 & 46\\
    -5 & -12 & x+27
\end{array} \right| =
x^3-2x^2-3x-6 = (x-2)(x-\sqrt{3})(x+\sqrt{3})
$$
El único valor propio en $\mathbb{Q}$ es $2$. Calculamos el subespacio asociads:\
$$
\left( \begin{array}{ccc}
    8 & 15 & -36\\
    8 & 21 & -46\\
    5 & 12 & -27
\end{array} \right)
\left( \begin{array}{c}
      x \\
      y \\
      z
\end{array} \right) = 2
\left( \begin{array}{c}
      x \\
      y \\
      z
\end{array} \right)
$$
$$
\Leftrightarrow \left\{ \begin{array}{c}
     6x+15y-36z =0  \\
     8x+19y-46z =0  \\
     5x+12y-29z =0
\end{array} \right.
\Leftrightarrow \left( \begin{array}{c}
      x \\
      y \\
      z
\end{array} \right) =
\left( \begin{array}{c}
      \mu \\
      2\mu \\
      \mu
\end{array} \right) \forall \mu \in \mathbb{Q}
$$
Al no tener todos sus valores propios en $\mathbb{Q}$ no es diagonalizable.\\\\
%
%
%c)
%
%
\textit{c)}
Escribimos la matriz correspondiente al endomorfismo:\\
$$
\left( \begin{array}{ccc}
    8 & 15 & -36\\
    8 & 21 & -46\\
    5 & 12 & -27
\end{array} \right)
$$
Calculamos el polinomio caraterístico:
$$
c_A(x) = |xI-A| = 
\left| \begin{array}{ccc}
    x-8 & -15 & 36\\
    -8 & x-21 & 46\\
    -5 & -12 & x+27
\end{array} \right| =
x^3-2x^2-3x-6 = (x-2)(x-\sqrt{3})(x+\sqrt{3})
$$
Los valores propios son: $2,\pm\sqrt{3}$. Calculamos los subespacios asociados:\\
Cuando $\lambda=2$:
$$
\left( \begin{array}{ccc}
    8 & 15 & -36\\
    8 & 21 & -46\\
    5 & 12 & -27
\end{array} \right)
\left( \begin{array}{c}
      x \\
      y \\
      z
\end{array} \right) = 2
\left( \begin{array}{c}
      x \\
      y \\
      z
\end{array} \right)
$$
$$
\Leftrightarrow \left\{ \begin{array}{c}
     6x+15y-36z =0  \\
     8x+19y-46z =0  \\
     5x+12y-29z =0
\end{array} \right.
\Leftrightarrow \left( \begin{array}{c}
      x \\
      y \\
      z
\end{array} \right) =
\left( \begin{array}{c}
      \mu \\
      2\mu \\
      \mu
\end{array} \right) \forall \mu \in \mathbb{R}
$$
cuando $\lambda=\sqrt{3}$:
$$
\left( \begin{array}{ccc}
    8 & 15 & -36\\
    8 & 21 & -46\\
    5 & 12 & -27
\end{array} \right)
\left( \begin{array}{c}
      x \\
      y \\
      z
\end{array} \right) = \sqrt{3}
\left( \begin{array}{c}
      x \\
      y \\
      z
\end{array} \right)
$$
$$
\Leftrightarrow \left\{ \begin{array}{c}
     (-\sqrt{3}+8)x+15y-36z =0  \\
     8x+(-\sqrt{3}+21)y-46z =0  \\
     5x+12y-(-\sqrt{3}-27)z =0
\end{array} \right.
\Leftrightarrow \left( \begin{array}{c}
      x \\
      y \\
      z
\end{array} \right) =
\left( \begin{array}{c}
      (\sqrt{3}+3)\mu \\
      \frac{-\sqrt{3}+3}{3}\mu \\
      \mu
\end{array} \right) \forall \mu \in \mathbb{R}
$$
cuando $\lambda=-\sqrt{3}$:
$$
\left( \begin{array}{ccc}
    8 & 15 & -36\\
    8 & 21 & -46\\
    5 & 12 & -27
\end{array} \right)
\left( \begin{array}{c}
      x \\
      y \\
      z
\end{array} \right) = \sqrt{3}
\left( \begin{array}{c}
      x \\
      y \\
      z
\end{array} \right)
$$
$$
\Leftrightarrow \left\{ \begin{array}{c}
     (\sqrt{3}+8)x+15y-36z =0  \\
     8x+(\sqrt{3}+21)y-46z =0  \\
     5x+12y-(\sqrt{3}-27)z =0
\end{array} \right.
\Leftrightarrow \left( \begin{array}{c}
      x \\
      y \\
      z
\end{array} \right) =
\left( \begin{array}{c}
      (-\sqrt{3}+3)\mu \\
      \frac{\sqrt{3}+3}{3}\mu \\
      \mu
\end{array} \right) \forall \mu \in \mathbb{R}
$$
Como la multiplicidad algebraica del cada valor propio vcomo raiz del polinomio caracteristico coincide con la dimensión del subespacio propio asociado a dicho valor propio en todos los casos, se tiene que el endomorfismo es diagonalizable. Se tiene:
$$
P^{-1}AP =
$$
$$
\left( \begin{array}{ccc}
    1 & 3 & -6\\
    \frac{\sqrt{3}-1}{2} & \frac{2\sqrt{3}-3}{2} & \frac{-5\sqrt{3}+7}{2}\\
    \frac{-\sqrt{3}-1}{2} & \frac{-2\sqrt{3}-3}{2} & \frac{5\sqrt{3}+7}{2}
\end{array} \right)
\left( \begin{array}{ccc}
    8 & 15 & -36\\
    8 & 21 & -46\\
    5 & 12 & -27
\end{array} \right)
\left( \begin{array}{ccc}
    1 & \sqrt{3}+3 & -\sqrt{3}+3\\
    2 & \frac{-\sqrt{3}+3}{3} & \frac{\sqrt{3}+3}{3}\\
    1 & 1 & 1
\end{array} \right)
$$
$$
=
\left( \begin{array}{ccc}
    1 & 0 & 0\\
    0 & \sqrt{3} & 0\\
    0 & 0 & -\sqrt{3}
\end{array} \right)
$$\\
%
%
%d)
%
%
\textit{d)}
Escribimos la matriz correspondiente al endomorfismo:\\
$$
\left( \begin{array}{ccc}
    4 & 7 & -5\\
    -4 & 5 & 0\\
    1 & 9 & -4
\end{array} \right)
$$
Calculamos el polinomio caraterístico:
$$
c_A(x) = |xI-A| = 
\left| \begin{array}{ccc}
    x-4 & -7 & -5\\
    4 & x-5 & 0\\
    -5 & -9 & x+4
\end{array} \right| =
x^3-5x^2+17x-13 = (x-1)(x-2+3i)(x+2-3i)
$$
el único valor propio en $\mathbb{Q}$ es: $1$. Calculamos el subespacio asociado:\\
Cuando $\lambda=1$:
$$
\left( \begin{array}{ccc}
    4 & 7 & -5\\
    -4 & 5 & 0\\
    1 & 9 & -4
\end{array} \right)
\left( \begin{array}{c}
      x \\
      y \\
      z
\end{array} \right) = 1
\left( \begin{array}{c}
      x \\
      y \\
      z
\end{array} \right)
$$
$$
\Leftrightarrow \left\{ \begin{array}{c}
     3x+7y-5z =0  \\
     -4x+4y =0  \\
     x+9y-5z =0
\end{array} \right.
\Leftrightarrow \left( \begin{array}{c}
      x \\
      y \\
      z
\end{array} \right) =
\left( \begin{array}{c}
      \frac{\mu}{2} \\
      \frac{\mu}{2} \\
      \mu
\end{array} \right) \forall \mu \in \mathbb{Q}
$$
Al no estar todos los valores propios en el cuerpo, no es diagonaizable.\\\\
%
%
%e)
%
%
\textit{e)}
Escribimos la matriz correspondiente al endomorfismo:\\
$$
\left( \begin{array}{ccc}
    4 & 7 & -5\\
    -4 & 5 & 0\\
    1 & 9 & -4
\end{array} \right)
$$
Calculamos el polinomio caraterístico:
$$
c_A(x) = |xI-A| = 
\left| \begin{array}{ccc}
    x-4 & -7 & -5\\
    4 & x-5 & 0\\
    -5 & -9 & x+4
\end{array} \right| =
x^3-5x^2+17x-13 = (x-1)(x-2+3i)(x+2-3i)
$$
Los valores propios son: $1,2-3i,-2+3i$. Calculamos los subespacios asociados:\\
Cuando $\lambda=1$:
$$
\left( \begin{array}{ccc}
    4 & 7 & -5\\
    -4 & 5 & 0\\
    1 & 9 & -4
\end{array} \right)
\left( \begin{array}{c}
      x \\
      y \\
      z
\end{array} \right) = 1
\left( \begin{array}{c}
      x \\
      y \\
      z
\end{array} \right)
$$
$$
\Leftrightarrow \left\{ \begin{array}{c}
     3x+7y-5z =0  \\
     -4x+4y =0  \\
     x+9y-5z =0
\end{array} \right.
\Leftrightarrow \left( \begin{array}{c}
      x \\
      y \\
      z
\end{array} \right) =
\left( \begin{array}{c}
      \frac{\mu}{2} \\
      \frac{\mu}{2} \\
      \mu
\end{array} \right) \forall \mu \in \mathbb{Q}
$$
Cuando $\lambda=2-3i$:
$$
\left( \begin{array}{ccc}
    4 & 7 & -5\\
    -4 & 5 & 0\\
    1 & 9 & -4
\end{array} \right)
\left( \begin{array}{c}
      x \\
      y \\
      z
\end{array} \right) = 1
\left( \begin{array}{c}
      x \\
      y \\
      z
\end{array} \right)
$$
$$
\Leftrightarrow \left\{ \begin{array}{c}
     (2+3i)x+7y-5z =0  \\
     -4x+(3+3i)y =0  \\
     x+9y-(6+3i)z =0
\end{array} \right.
\Leftrightarrow \left( \begin{array}{c}
      x \\
      y \\
      z
\end{array} \right) =
\left( \begin{array}{c}
      (\frac{12}{17}+\frac{3}{17}i)\mu \\
      (\frac{10}{17}-\frac{6}{17}i)\mu \\
      \mu
\end{array} \right) \forall \mu \in \mathbb{Q}
$$
Cuando $\lambda=2-3i$:
$$
\left( \begin{array}{ccc}
    4 & 7 & -5\\
    -4 & 5 & 0\\
    1 & 9 & -4
\end{array} \right)
\left( \begin{array}{c}
      x \\
      y \\
      z
\end{array} \right) = 1
\left( \begin{array}{c}
      x \\
      y \\
      z
\end{array} \right)
$$
$$
\Leftrightarrow \left\{ \begin{array}{c}
     (2-3i)x+7y-5z =0  \\
     -4x+(3-3i)y =0  \\
     x+9y-(6-3i)z =0
\end{array} \right.
\Leftrightarrow \left( \begin{array}{c}
      x \\
      y \\
      z
\end{array} \right) =
\left( \begin{array}{c}
      (\frac{12}{17}-\frac{3}{17}i)\mu \\
      (\frac{10}{17}+\frac{6}{17}i)\mu \\
      \mu
\end{array} \right) \forall \mu \in \mathbb{Q}
$$
Como la multiplicidad algebraica del cada valor propio vcomo raiz del polinomio caracteristico coincide con la dimensión del subespacio propio asociado a dicho valor propio en todos los casos, se tiene que el endomorfismo es diagonalizable. Se tiene:
$$
P^{-1}AP=
$$
$$
\left( \begin{array}{ccc}
     -4 & -2 & 4 \\
     \frac{4-i}{2} & \frac{6+7i}{6} & \frac{-9-2i}{6} \\
     \frac{4+i}{2} & \frac{6-7i}{6} & \frac{-9+2i}{6}
\end{array}\right)
\left( \begin{array}{ccc}
     4 & 7 & -5 \\
     -4 & 5 & 0 \\
     1 & 9 & -4 
\end{array}\right)
\left( \begin{array}{ccc}
     \frac{1}{2} & \frac{12\mu}{17}+\frac{(3\mu)i}{17} & \frac{12\mu}{17}-\frac{(3\mu)i}{17} \\
     \frac{1}{2} & \frac{10\mu}{17}-\frac{(6\mu)i}{17} & \frac{10\mu}{17}+\frac{(6\mu)i}{17} \\
     1 & 1 & 1 
\end{array}\right)
$$
$$
=
\left( \begin{array}{ccc}
     1 & 0 & 0 \\
     0 & 2-3i & 0 \\
     0 & 0 & 2+3i 
\end{array}\right)
$$\\
%
%
%f)
%
%
\textit{f)}
Escribimos la matriz correspondiente al endomorfismo:\\
$$
\left( \begin{array}{ccc}
    4 & 2 & -5\\
    6 & 4 & -9\\
    5 & 3 & -7
\end{array} \right)
$$
Calculamos el polinomio caraterístico:
$$
c_A(x) = |xI-A| = 
\left| \begin{array}{ccc}
    x-4 & -2 & 5\\
    -6 & x-4 & 9\\
    -5 & -3 & x+7
\end{array} \right| =
x^3-x^2 = x^2(x-1)
$$
Losvalores propios son: $1,0$. Calculamos los subespacios asociados:\\
Cuando $\lambda=0$:
$$
\left( \begin{array}{ccc}
    4 & 2 & -5\\
    6 & 4 & -9\\
    5 & 3 & -7
\end{array} \right)
\left( \begin{array}{c}
      x \\
      y \\
      z
\end{array} \right) = 1
\left( \begin{array}{c}
      x \\
      y \\
      z
\end{array} \right)
$$
$$
\Leftrightarrow \left\{ \begin{array}{c}
     4x+2y-5z =0  \\
     6x+4y-9z=0  \\
     5x+3y-7z =0
\end{array} \right.
\Leftrightarrow \left( \begin{array}{c}
      x \\
      y \\
      z
\end{array} \right) =
\left( \begin{array}{c}
      \frac{\mu}{2} \\
      \frac{3\mu}{2} \\
      \mu
\end{array} \right) \forall \mu \in \mathbb{Q}
$$
Cuando $\lambda=1$:
$$
\left( \begin{array}{ccc}
    4 & 2 & -5\\
    6 & 4 & -9\\
    5 & 3 & -7
\end{array} \right)
\left( \begin{array}{c}
      x \\
      y \\
      z
\end{array} \right) = 1
\left( \begin{array}{c}
      x \\
      y \\
      z
\end{array} \right)
$$
$$
\Leftrightarrow \left\{ \begin{array}{c}
     3x+2y-5z =0  \\
     6x+3y-9z=0  \\
     5x+3y-8z =0
\end{array} \right.
\Leftrightarrow \left( \begin{array}{c}
      x \\
      y \\
      z
\end{array} \right) =
\left( \begin{array}{c}
      \mu \\
      \mu \\
      \mu
\end{array} \right) \forall \mu \in \mathbb{Q}
$$
La multiplicidad algebráica de $0$ no coincide con la dimensión del subespacio asociado. Por lo tanto, no es diagonalizable.
%
%
%Ejercicio 2
%
%
\\
\noindent \textbf{2.}\\
Dado que $A$ es inversible, se tiene que $|A| \ne 0$. Ahora, sabemos que el término independiente del polinomio caraterístico es $(-1)^n |A|$. Por lo tanto:
$$
c_A (0) = (-1)^n |A| \ne 0
$$
Por lo que $0$ no es valor propio al no ser raiz del polinomio característico.\\
Sabemos que $\exists v$ tal que:
$$
Av=\lambda v \Rightarrow v = A^{-1} \lambda v\Rightarrow \lambda^{-1} v = A^{-1} v
$$
Por lo que $\lambda^{-1}$ es valor propio de $A^{-1}$.\\\\
%
%
%Ejercicio 3
%
%
\noindent \textbf{3.} Por inducción:
$$
\exists v : f^2(v) = f(f(v)) = f(\lambda v) = \lambda f(v) = \lambda\lambda v =\lambda^2v \quad (1)
$$
$$
f^3(v) = f^2(f(v)) = f^2(\lambda v) = \underbrace{\lambda f^2(v) = \lambda \lambda^2 v}_{(1)} = \lambda^3 v \quad (2)
$$
$$
f^4(v) = f^3(f(v)) = f^3(\lambda v) = \underbrace{\lambda f^3(v) = \lambda \lambda^3 v}_{(2)} = \lambda^4 v \quad (3)
$$
$$
\vdots
$$
$$
f^{n+1}(v) = f^n(f(v)) = f^n(\lambda v) = \underbrace{\lambda f^n(v) = \lambda \lambda^n v}_{(n-1)} = \lambda^{n+1} v \quad (n)
$$
$$
\vdots
$$
\end{document}
