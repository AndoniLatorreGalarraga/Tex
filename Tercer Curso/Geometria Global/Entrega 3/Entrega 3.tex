\documentclass{article}
\usepackage[utf8]{inputenc}
\usepackage{graphicx}
\usepackage[spanish]{babel}
\usepackage{amssymb,amsmath,geometry,multicol,spalign,hyperref}
\usepackage[usenames,dvipsnames]{xcolor}
\usepackage{tikz,mathtools}
\usepackage{pgfplots}
\pgfplotsset{every axis/.append style={
                    axis x line=middle,    % put the x axis in the middle
                    axis y line=middle,    % put the y axis in the middle
                    axis line style={<->,color=blue}, % arrows on the axis
                    xlabel={$x$},          % default put x on x-axis
                    ylabel={$y$},          % default put y on y-axis
            }}
\usepackage{etoolbox} %titulo
\makeatletter %titulo
\patchcmd{\@maketitle}{\vskip 2em}{\vspace*{-3cm}}{}{} %titulo
\makeatother %titulo
\usepackage{vmargin}
\setpapersize{A4}
\setmargins{2.5cm}       % margen izquierdo
{1.5cm}                        % margen superior
{16.5cm}                      % anchura del texto
{23.42cm}                    % altura del texto
{10pt}                           % altura de los encabezados
{1cm}                           % espacio entre el texto y los encabezados
{0pt}                             % altura del pie de página
{2cm}                           % espacio entre el texto y el pie de página
\title{Entrega 3}
\author{Andoni Latorre Galarraga}
\date{}
\newcommand{\bb}[1]{\mathbb{#1}}
\newcommand{\nota}[3][2ex]{
    \underset{\mathclap{
        \begin{tikzpicture}
          \draw[->] (0, 0) to ++(0,#1);
          \node[below] at (0,0) {#3};
        \end{tikzpicture}}}{#2}
}
\begin{document}

\maketitle

\textcolor{WildStrawberry}{Problema:}\\
Sea $\textcolor{blue}{\alpha:[0,2\pi] \longrightarrow} \bb{R}^2$ dada por $\textcolor{blue}{\alpha(t)= (2\cos t -1)(\cos t, \sen t)}$.\\
\textit{i)} Representar $\textcolor{blue}{\alpha([0, 2\pi])}$.\\
\begin{tikzpicture}
    \begin{axis}[
            xmin=-1,xmax=4,
            ymin=-2,ymax=2,
            grid=both,
            ]
            \addplot [domain=0:360,samples=360]({(2*cos(x)-1)*cos(x)},{(2*cos(x)-1)*sin(x)}); 
    \end{axis}
\end{tikzpicture}\\
\textit{ii)} ¿Es $\textcolor{blue}{\alpha}$ simple?\\
No es simple por no ser inyectiva. $\textcolor{blue}{\alpha}(\frac{\pi}{3})=\textcolor{blue}{\alpha}(\frac{5\pi}{3})=(0,0)$.\\
\textit{iii)} ¿Es $\textcolor{blue}{\alpha}$ convexa?\\
No es convexa, evidentemente la recta tangente en $\textcolor{blue}{\alpha}(0)$ que es $x= 1$ corta la curva en otros dos puntos. Además veremos que solo tiene 2 vértices, por el teorema de los 4 vértices no puede ser convexa.\\
\textit{iv)} Calcular los vértices de $\textcolor{blue}{\alpha}$.\\
Calculamos la \href{https://www.wolframalpha.com/input?i=Curvature+of+%28%282cos%28t%29-1%29*cos%28t%29%2C%282cos%28t%29-1%29*sin%28t%29%29}{\textcolor{WildStrawberry}{\underline{curvatura}}} y su \href{https://www.wolframalpha.com/input?i=derivative+%289+-+6+cos%28t%29%29%2F%285+-+4+cos%28t%29%29%5E%283%2F2%29}{\textcolor{WildStrawberry}{\underline{derivada}}}.
$$
\alpha(t) = ((2\cos t -1)\cos t, (2\cos t -1) \sen t)
$$
$$
\alpha'(t) = ( {-2\sen t \cos t - (2\cos t -1)\sen t}, {-2\sen t \sen t + (2\cos t -1)\cos t}) =
$$
$$
= ({\sen t - 4 \cos t \sen t}, {2(\cos^2 t -\sen^2 t)-\cos t})
$$
$$
\alpha''(t) = (\cos t - 4 (-\sen t \sen t + \cos t \cos t), 2(-2\cos t \sen t - 2 \sen t \cos t) + \sen t)=
$$
$$
= (\cos t {+ 4(\sen^2 t - \cos^2 t)}, {\sen t(1 - 8 \cos t)})
$$
$$
k_2(t) = \frac{\alpha''(t) \cdot \mathcal{J}\alpha'(t)}{\left\| \alpha'(t) \right\|^3} =
$$
$$
= \frac{(\cos t + 4(\sen^2 t - \cos^2 t), \sen t(1 - 8 \cos t)) \cdot \mathcal{J}(\sen t - 4 \cos t \sen t, 2(\cos^2 t -\sen^2 t)-\cos t)}{\left\| (\sen t - 4 \cos t \sen t, 2(\cos^2 t -\sen^2 t)-\cos t) \right\|^3} =
$$
$$
= \frac{{-(\cos t + 4(\sen^2 t - \cos^2 t)) (2(\cos^2 t -\sen^2 t)-\cos t) + (\sen t - 8\sen t \cos t) (\sen t - 4 \cos t \sen t)}}{{\left(\sqrt{\left( \sen t - 4 \cos t \sen t \right)^2+\left( 2(\cos^2 t -\sen^2 t)-\cos t \right)^2}\right)^3}} =
$$
Abreviamos $s = \sen t$ y $c = \cos t$
$$
= \frac{{-(c  + 4(s^2  - c^2 )) (2(c^2  -s^2 )-c ) + (s  - 8s  c ) (s  - 4 c  s )}}{{(\sqrt{( s  - 4 c  s  )^2+( 2(c^2  -s^2 )-c  )^2})^3}} =
$$
Sustituimos $s^2 = 1-c^2$
$$
= \frac{{-(c  + 4(\textcolor{red}{1-c^2}  - c^2 )) (2(c^2  -\textcolor{red}{(1-c^2)} )-c ) + (s  - 8s  c ) (s  - 4 c  s )}}{{(\sqrt{( s  - 4 c  s  )^2+( 2(c^2  -\textcolor{red}{(1-c^2)} )-c  )^2})^3}} =
$$
$$
= \frac{{-(c  + 4( 1-c^2 - c^2 )) (2(c^2  \textcolor{red}{-1+c^2} )-c ) + (s  - 8s  c ) (s  - 4 c  s )}}{{(\sqrt{( s  - 4 c  s  )^2+( 2(c^2 \textcolor{red}{-1+c^2} )-c  )^2})^3}} =
$$
$$
= \frac{{-(c  + 4( 1-\textcolor{red}{2c^2} )) (2(\textcolor{red}{2c^2}  -1 )-c ) + (s  - 8s  c ) (s  - 4 c  s )}}{{(\sqrt{( s  - 4 c  s  )^2+( 2(\textcolor{red}{2c^2}  -1 )-c  )^2})^3}} =
$$
$$
= \frac{{-(c  +\textcolor{red}{ 4 - 8c^2 }) (\textcolor{red}{4c^2 - 2} - c ) + (s  - 8s  c ) (s  - 4 c  s )}}{{(\sqrt{( s  - 4 c  s  )^2+( 2(2c^2  -1 )-c  )^2})^3}} =
$$
$$
= \frac{{-(c  + 4 - 8c^2 ) (4c^2 - 2 - c ) + \textcolor{red}{s^2}(1 - 8 c ) (1 - 4 c )}}{{(\sqrt{( s  - 4 c  s  )^2+( 2(2c^2  -1 )-c  )^2})^3}} =
$$
$$
= \frac{{-(c  + 4 - 8c^2 ) (4c^2 - 2 - c ) + \textcolor{red}{(1-c^2)}(1 - 8 c ) (1 - 4 c )}}{{(\sqrt{( s  - 4 c  s  )^2+( 2(2c^2  -1 )-c  )^2})^3}} =
$$
$$
= \frac{{-(c  + 4 - 8c^2 ) (4c^2 - 2 - c ) \textcolor{red}{-32 c^4 + 12 c^3 + 31 c^2 - 12 c + 1}}}{{(\sqrt{( s  - 4 c  s  )^2+( 2(2c^2  -1 )-c  )^2})^3}} =
$$
$$
= \frac{{-(\textcolor{red}{-32 c^4 + 12 c^3 + 31 c^2 - 6 c - 8}) + -32 c^4 + 12 c^3 + 31 c^2 - 12 c + 1}}{{(\sqrt{( s  - 4 c  s  )^2+( 2(2c^2  -1 )-c  )^2})^3}} =
$$
$$
= \frac{\textcolor{red}{9-6c}}{{(\sqrt{( s  - 4 c  s  )^2+( 2(2c^2  -1 )-c  )^2})^3}} =
$$
$$
= \frac{9-6c}{{(\sqrt{\textcolor{red}{s^2}( 1 - 4 c )^2+( 2(2c^2  -1 )-c  )^2})^3}} =
$$
$$
= \frac{9-6c}{{(\sqrt{\textcolor{red}{(1 - c^2)}( 1 - 4 c )^2+( 2(2c^2  -1 )-c  )^2})^3}} =
$$
$$
= \frac{9-6c}{{(\sqrt{(1 - c^2)( \textcolor{red}{1 - 8c +16c^2} )+( \textcolor{red}{4c^2 - 2} - c )^2})^3}} =
$$
$$
= \frac{9-6c}{{(\sqrt{\textcolor{red}{-16 c^4 + 8 c^3 + 15 c^2 - 8 c + 1} +( 4c^2 - 2 - c )^2})^3}} =
$$
$$
= \frac{9-6c}{{(\sqrt{-16 c^4 + 8 c^3 + 15 c^2 - 8 c + 1 +\textcolor{red}{16 c^4 - 8 c^3 - 15 c^2 + 4 c + 4}})^3}} =
$$
$$
= \frac{9-6c}{{(\sqrt{\textcolor{red}{5-4c}})^3}} =
$$
$$
= \frac{9-6 \cos t}{(5-4\cos t)^{3/2}}
$$
Derivamos
$$
k_2'(t) =
-\frac{(9-6 \cos t)((5-4\cos t)^{3/2})' - (9-6 \cos t)'((5-4\cos t)^{3/2})}{(5-4\cos t)^3}
$$
$$
=-\frac{(9-6 \cos t)\frac{3}{2}(5-4\cos t)^{1/2}(4 \sen t) - 6 \sen t(5-4\cos t)^{3/2}}{(5-4\cos t)^3}=
$$
$$
=-\frac{(5-4\cos t)^{1/2}((9-6 \cos t)\frac{3}{2}(4 \sen t) - 6 \sen t(5-4\cos t))}{(5-4\cos t)^3}=
$$
$$
=-\frac{(9-6 \cos t)(6 \sen t) - 6 \sen t(5-4\cos t)}{(5-4\cos t)^{5/2}}=
$$
$$
=-\frac{6 \sen t(9-6 \cos t- 5 + 4\cos t)}{(5-4\cos t)^{5/2}}=
$$
$$
=-\frac{6 \sen t(4 - 2 \cos t)}{(5-4\cos t)^{5/2}}=
$$
$$
=\frac{12 \sen t(\cos t -2)}{(5- 4 \cos t)^{5/2}}
$$
Tenemos ceros, y por lo tanto vértices, en $0, \pi$ y $2\pi$, es decir, en $(3,0)$ y $(1,0)$.
\end{document}