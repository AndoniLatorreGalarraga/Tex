\documentclass{article}
\usepackage[utf8]{inputenc}
\usepackage{graphicx}
\usepackage[spanish]{babel}
\usepackage{amssymb,amsmath,geometry,multicol,spalign,hyperref}
\usepackage[usenames,dvipsnames]{xcolor}
\usepackage{tikz,mathtools}
\usepackage{pgfplots}
\pgfplotsset{every axis/.append style={
                    axis x line=middle,    % put the x axis in the middle
                    axis y line=middle,    % put the y axis in the middle
                    axis line style={<->,color=blue}, % arrows on the axis
                    xlabel={$x$},          % default put x on x-axis
                    ylabel={$y$},          % default put y on y-axis
            }}
\usepackage{etoolbox} %titulo
\makeatletter %titulo
\patchcmd{\@maketitle}{\vskip 2em}{\vspace*{-3cm}}{}{} %titulo
\makeatother %titulo
\usepackage{vmargin}
\setpapersize{A4}
\setmargins{2.5cm}       % margen izquierdo
{1.5cm}                        % margen superior
{16.5cm}                      % anchura del texto
{23.42cm}                    % altura del texto
{10pt}                           % altura de los encabezados
{1cm}                           % espacio entre el texto y los encabezados
{0pt}                             % altura del pie de página
{2cm}                           % espacio entre el texto y el pie de página
\title{Entrega 7}
\author{Andoni Latorre Galarraga}
\date{}
\newcommand{\bb}[1]{\mathbb{#1}}
\newcommand{\R}{\bb{R}}
\newcommand{\nota}[3][2ex]{
    \underset{\mathclap{
        \begin{tikzpicture}
          \draw[->] (0, 0) to ++(0,#1);
          \node[below] at (0,0) {#3};
        \end{tikzpicture}}}{#2}
}
\begin{document}

\maketitle

\textcolor{WildStrawberry}{Problema:}\\
Ver si las curvas coordenadas son geodésicas para una superficie de revolución.

\textcolor{WildStrawberry}{Solución:}\\
Si la superficie viene parametrizada por $\textcolor{blue}{\bb{X}}(u,v)=(f(u)\cos v, f(u)\sen v, g(u))$, las curvas coordenadas son
$$
\textcolor{blue}{\alpha_u} : \textcolor{blue}{\bb{X}}(u_0, t) = (f(u_0)\cos t, f(u_0)\sen t, g(u_0))
$$
$$
\textcolor{blue}{\alpha_v} : \textcolor{blue}{\bb{X}}(t, v_0) = (f(t)\cos v_0, f(t)\sen v_0, g(t))
$$
Para que las curvas sean geodésicas necesitamos que la componente tangencial de sus segundas derivadas sea nula.
$$
\textcolor{blue}{\bb{X}}_u = (f'(u) \cos v, f'(u) \sen v, g'(u))
$$
$$
\textcolor{blue}{\bb{X}}_v = (-f(u)\sen v, f(u)\cos v, 0)
$$
$$
N = \textcolor{blue}{\bb{X}}_u \land \textcolor{blue}{\bb{X}}_v =
\left|\begin{array}{ccc}
    i & j & k \\
    f'(u) \cos v & f'(u) \sen v & g'(u) \\
    -f(u)\sen v & f(u)\cos v & 0
\end{array}\right|=
$$
$$
=i
\left|\begin{array}{cc}
    f'(u)\sen v & g'(u) \\
    f(u)\cos v & 0 \\
\end{array}\right|
- j
\left|\begin{array}{cc}
    f'(u)\cos v & g'(u) \\
    -f(u)\sen v & 0 \\
\end{array}\right|
+ k
\left|\begin{array}{cc}
    f'(u)\cos v & f'(u)\sen v \\
    -f(u)\sen v & f(u)\cos v \\
\end{array}\right| =
$$
$$
=i g'(u)f(u)\cos v + j g'(u)f(u)\sen v + k (f'(u)f(u))=
$$
$$
=(g'(u)f(u) \cos v, g'(u)f(u)\sen v, f'(u)f(u))
$$
Empecemos por $\textcolor{blue}{\alpha_u}$. Veamos si está parametrizada por longitud de arco:
$$
\textcolor{blue}{\alpha_u}' = (-f(u_0) \sen t, f(u_o) \cos t, 0)
$$
$$
\| \textcolor{blue}{\alpha_u'} \| = |f(u_o)|
$$
Tenemos que,
$$
\beta_u' = \frac{\textcolor{blue}{\alpha_u}'}{f(u_o)}
$$
Donde $\beta_u$ es $\textcolor{blue}{\alpha_u}$ parametrizada por longitud de arco.
$$
\beta_u' = (-\sen t, \cos t, 0)
$$
$$
\beta_u'' = (-\cos t, -\sen t, 0)
$$
Como $\beta_u'\beta_u'' = 0$, bastaría con ver que $\beta_u''\cdot (N \land \beta_u') = 0$.
$$
N \land \beta_u' =
\left|\begin{array}{ccc}
    i & j & k \\
    g'(u_0)f(u_0) \cos t & g'(u_0)f(u_0)\sen t & f'(u_0)f(u_0) \\
    -\sen t & \cos t & 0 
\end{array}\right| =
$$
$$
= ( f'(u_0)f(u_0)\cos t , -f'(u_0)f(u_0)\sen t , g'(u_0)f(u_0) \cos^2 t + g'(u_0)f(u_0)\sen^2 t ) =
$$
$$
= ( f'(u_0)f(u_0)\cos t , -f'(u_0)f(u_0)\sen t , g'(u_0)f(u_0) )
$$
$$
\beta_u''\cdot (N \land \beta_u') =
(-\cos t, -\sen t, 0) \cdot ( f'(u_0)f(u_0)\cos t , -f'(u_0)f(u_0)\sen t , g'(u_0)f(u_0) ) =
$$
$$
= -f'(u_0)f(u_0)\cos^2 t + f'(u_0)f(u_0)\sen^2 t = (f'(u_0)f(u_0))(1-2\cos^2 t)
$$
Si $f(u_0) = 0$, $\| \textcolor{blue}{\alpha_u}'\| = 0$ y la curva no es regular. Cuando $f'(u_0)=0$, $\textcolor{blue}{\alpha_u}$ es geodésica.\\
Veamos que ocurre con $\textcolor{blue}{\alpha_v}$.
$$
\textcolor{blue}{\alpha_v}' = (f'(t)\cos v_0, f'(t) \sen v_0, g'(t))
$$
$$
\| \textcolor{blue}{\alpha_v}' \| = \sqrt{f'(t)^2 + g'(t)^2}
$$
$$
\beta_v ' = \frac{ \textcolor{blue}{\alpha_v}' }{\sqrt{f'(t)^2 + g'(t)^2}}
$$
$$
\beta_v '' = ( \cos v_0 h(t), \sen v_0 h(t), l(t) )
$$
Con
$$
h(t) = \frac{f''(t) \sqrt{f'(t)^2 + g'(t)^2} - f'(t) \frac{f'(t)f''(t) + g'(t)g''(t)}{\sqrt{f'(t)^2 + g'(t)^2}}}{f'(t)^2 + g'(t)^2}
$$
$$
l(t) = \frac{g''(t) \sqrt{f'(t)^2 + g'(t)^2} - g'(t) \frac{f'(t)f''(t) + g'(t)g''(t)}{\sqrt{f'(t)^2 + g'(t)^2}}}{f'(t)^2 + g'(t)^2}
$$
$$
\beta'_v\beta''_v = \frac{ \textcolor{blue}{\alpha_v}' }{\| \textcolor{blue}{\alpha_v}' \|}\beta''_v =
\frac{ 1}{\| \textcolor{blue}{\alpha_v}' \|} (f'(t)\cos v_0, f'(t) \sen v_0, g'(t)) \cdot ( \cos v_0 h(t), \sen v_0 h(t), l(t) )
$$
$$
= \frac{ 1}{\| \textcolor{blue}{\alpha_v}' \|}(f'(t)h(t)\cos^2 v_0 + f'(t)h(t)\sen^2 v_0 + g'(t)l(t)) = \frac{ 1}{\| \textcolor{blue}{\alpha_v}' \|}( f'(t)h(t) + g'(t)l(t) )
$$
Nos centramos en $f'(t)h(t) + g'(t)l(t)$.
$$
f'(t)h(t) + g'(t)l(t) =
$$
$$
= \frac{
    f'(t)f''(t) \sqrt{f'(t)^2 + g'(t)^2} - f'(t)^2 \frac{f'(t)f''(t) + g'(t)g''(t)}{\sqrt{f'(t)^2 + g'(t)^2}} +
    g'(t)g''(t) \sqrt{f'(t)^2 + g'(t)^2} - g'(t)^2 \frac{f'(t)f''(t) + g'(t)g''(t)}{\sqrt{f'(t)^2 + g'(t)^2}}}{f'(t)^2 + g'(t)^2}=
$$
$$
= \frac{
    (f'(t)f''(t) + g'(t)g''(t)) \sqrt{f'(t)^2 + g'(t)^2} - (f'(t)^2 + g'(t)^2) \frac{f'(t)f''(t) + g'(t)g''(t)}{\sqrt{f'(t)^2 + g'(t)^2}}}{f'(t)^2 + g'(t)^2}
$$
Entonces,
$$
\beta'_v\beta''_v =
\frac{
    (f'(t)f''(t) + g'(t)g''(t)) - (f'(t)^2 + g'(t)^2) \frac{f'(t)f''(t) + g'(t)g''(t)}{f'(t)^2 + g'(t)^2}}{f'(t)^2 + g'(t)^2}
= 0
$$
Veamos que ocurre con $\beta_v''\cdot(N \land \textcolor{blue}{\alpha_v}')$.
$$
N \land \textcolor{blue}{\alpha_v}' =
\left|\begin{array}{ccc}
    i & j & k \\
    g'(t)f(t) \cos v_0 & g'(t)f(t)\sen v_0 & f'(t)f(t) \\
    f'(t)\cos v_0 & f'(t) \sen v_0 & g'(t) \\
\end{array}\right| =
$$
$$
= ( g'(t)^2f(t)\sen v_0 - f'(t)^2f(t)\sen v_o , -g'(t)^2f(t)\cos v_0 + f'(t)^2f(t)\cos v_0, 0 )
$$
$$
= ( \sen v_0 (g'(t)^2f(t)- f'(t)^2f(t)) , -\cos v_0(g'(t)^2f(t) - f'(t)^2f(t)), 0 )
$$
$$
= (g'(t)^2f(t)- f'(t)^2f(t))(\sen v_0, -\cos v_0, 0 )
$$
$$
\frac{\beta_v''\cdot(N \land \textcolor{blue}{\alpha_v}')}{(g'(t)^2f(t)- f'(t)^2f(t))} =
(\sen v_0, -\cos v_0, 0 ) \cdot
( \cos v_0 h(t), \sen v_0 h(t), l(t) ) = 0
$$
Y se tiene que $\textcolor{blue}{\alpha_v}$ es geodésica.
\end{document}