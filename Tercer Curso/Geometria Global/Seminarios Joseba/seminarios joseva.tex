\documentclass{article}
\usepackage[utf8]{inputenc}
\usepackage{graphicx}
\usepackage[spanish]{babel}
\usepackage{amssymb,amsmath,geometry,multicol,spalign,hyperref}
\usepackage[usenames,dvipsnames]{xcolor}
\usepackage{tikz,mathtools}
\usepackage{pgfplots}
\pgfplotsset{every axis/.append style={
                    axis x line=middle,    % put the x axis in the middle
                    axis y line=middle,    % put the y axis in the middle
                    axis line style={<->,color=blue}, % arrows on the axis
                    xlabel={$x$},          % default put x on x-axis
                    ylabel={$y$},          % default put y on y-axis
            }}
\usepackage{etoolbox} %titulo
\makeatletter %titulo
\patchcmd{\@maketitle}{\vskip 2em}{\vspace*{-3cm}}{}{} %titulo
\makeatother %titulo
\usepackage{vmargin}
\setpapersize{A4}
\setmargins{2.5cm}       % margen izquierdo
{1.5cm}                        % margen superior
{16.5cm}                      % anchura del texto
{23.42cm}                    % altura del texto
{10pt}                           % altura de los encabezados
{1cm}                           % espacio entre el texto y los encabezados
{0pt}                             % altura del pie de página
{2cm}                           % espacio entre el texto y el pie de página
\title{Tareas propuestas en clase}
\author{Andoni Latorre Galarraga}
\date{}
\newcommand{\bb}[1]{\mathbb{#1}}
\newcommand{\R}{\bb{R}}
\newcommand{\nota}[3][2ex]{
    \underset{\mathclap{
        \begin{tikzpicture}
          \draw[->] (0, 0) to ++(0,#1);
          \node[below] at (0,0) {#3};
        \end{tikzpicture}}}{#2}
}
\usepackage[percent]{overpic}
\begin{document}

\maketitle
\begin{multicols}{2}
\section*{\textcolor{WildStrawberry}{Tarea 1}}
\noindent $\bullet$ Calcular la proyección estereografica de la esfera $\textcolor{blue}{\mathcal{S}}: x^2 + y^2 + (z-1)^2 = 1$ sobre el plano $\textcolor{blue}{XY}$.\\
\\
Consideramos el polo norte de la esfera $\textcolor{blue}{B} = \textcolor{blue}{(0,0,2)}$ y el punto en la esfera $\textcolor{blue}{E} = \textcolor{blue}{(x,y,z)}$
\begin{center}
    \begin{overpic}[scale = 0.2]{figures/esfera.png}
        \put (50,39) {$\textcolor{blue}{B}$}
        \put (61,31) {$\textcolor{blue}{E}$}
        \put (82,17) {$\textcolor{blue}{C}$}
    \end{overpic}
\end{center}
Consideramos la recta que pasa por $\textcolor{blue}{B}$ y $\textcolor{blue}{E}$.
$$
\textcolor{blue}{r} : \textcolor{blue}{B} + \lambda(\textcolor{blue}{E} - \textcolor{blue}{B} ) =  (\lambda x, \lambda y, 2 - \lambda(z-2) )
$$
Al interscar $\textcolor{blue}{r}$ con el plano $\textcolor{blue}{XY}$ para obtener el punto $\textcolor{blue}{C}$.
$$
0 = 2 - \lambda(z-2) \quad \Rightarrow \quad \lambda = \frac{2}{z-2}
$$
$$
\Rightarrow \quad \textcolor{blue}{C} = (\frac{2x}{z-2}, \frac{2y}{z-2}, 0)
$$
Ahora tenemos la aplicación $\textcolor{blue}{\bb{X}^{-1}}$ dada por
$$
\begin{array}{crcl}
\textcolor{blue}{\bb{X}^{-1}} : &\textcolor{blue}{\mathcal{S}} & \longrightarrow & \bb{R}^2 \\
& (x,y,x) & \longmapsto     & (u,v) = (\frac{2x}{z-2}, \frac{2y}{z-2})
\end{array}
$$
Calculamos $\textcolor{blue}{\bb{X}}$.
$$
x^2 + y^2 + (z-1)^2 = 1
$$
$$
(\frac{u}{2}(z-2))^2 + (\frac{v}{2}(z-2))^2 + z^2-2z+1 = 1
$$
$$
\frac{u^2+v^2}{4}(z^2-4z+4) + z^2-2z = 0
$$
$$
\frac{u^2+v^2+4}{4} z^2 - (u^2+v^2+2) z + u^2+v^2  = 0
$$
Resolviendo la caudrática en $z$.
$$
z = 2\frac{(u^2+v^2+2) \pm 2}{u^2+v^2+4}
$$
Con el $+$ tenemos $z=2$ pero como el polo norte $\textcolor{blue}{B}$ no es parte de la proyección tomamos la otra solución.
$$
z = 2\frac{(u^2+v^2+2)-2}{u^2+v^2+4} = \frac{2u^2 + 2v^2}{u^2+v^2+4}
$$
$$
x = \frac{u}{2}(z-2))^2 = \frac{u}{2}(\frac{2u^2 + 2v^2}{u^2+v^2+4}-2)) = \frac{-4u}{u^2 + v^2 + 4}
$$
$$
y = \frac{v}{2}(z-2))^2 = \frac{v}{2}(\frac{2u^2 + 2v^2}{u^2+v^2+4}-2)) = \frac{-4v}{u^2 + v^2 + 4}
$$
$$
\begin{array}{crcl}
\textcolor{blue}{\bb{X}} : & \bb{R}^2 & \longrightarrow & \textcolor{blue}{S} \\
& (u,v) & \longmapsto     & (\frac{-4u}{u^2 + v^2 + 4} , \frac{-4v}{u^2 + v^2 + 4}, \frac{2u^2 + 2v^2}{u^2+v^2+4})
\end{array}
$$

\section*{\textcolor{WildStrawberry}{Tarea 2}}

\noindent $\bullet$ Sea $\textcolor{blue}{\alpha: \bb{R}\longrightarrow \bb{R}^3}$ una curva regular $\textcolor{blue}{C^\infty}$ y $\textcolor{blue}{\bb{X}(u,v)=\alpha(u)+v\bb{N}(u)}$ una carta paramétrica. También se supone $\textcolor{blue}{\left\| \frac{d\alpha}{du} \right\|=1}$. Calcular las formas fundamentales.\\
\\
Calculamos $\textcolor{blue}{\bb{X}_u}$ y $\textcolor{blue}{\bb{X}_v}$.
$$
\begin{array}{rl}
    \textcolor{blue}{\bb{X}_u} = & \bb{T} + v (-\kappa \bb{T} + \tau \bb{B}) \\
    = & (1 - v\kappa) \bb{T} + v \tau \bb{B}\\
    \textcolor{blue}{\bb{X}_v} = & \bb{N}
\end{array}
$$
Sean el punto  $p=\textcolor{blue}{\bb{X}(u_0, v_0)}$ y la base de $T_p$ (el espacio tangente a la superficie en $p$) $\beta = \{ \textcolor{blue}{\bb{X}_u}, \textcolor{blue}{\bb{X}_u} \}$ tenemos
$$
(I_p)_\beta =
\left(\begin{array}{cc}
    \langle \textcolor{blue}{\bb{X}_u} , \textcolor{blue}{\bb{X}_u} \rangle & \langle \textcolor{blue}{\bb{X}_u} , \textcolor{blue}{\bb{X}_v} \rangle \\
    \langle \textcolor{blue}{\bb{X}_v} , \textcolor{blue}{\bb{X}_u} \rangle & \langle \textcolor{blue}{\bb{X}_v} , \textcolor{blue}{\bb{X}_v} \rangle \\
\end{array}\right)
$$
$$
\langle \textcolor{blue}{\bb{X}_u} , \textcolor{blue}{\bb{X}_u} \rangle = \langle (1 - v\kappa) \bb{T} + v \tau \bb{B}, (1 - v\kappa) \bb{T} + v \tau \bb{B} \rangle =
$$
$$
= (1 - v\kappa)^2 + (v \tau)^2 = 1 - 2 v\kappa + (v^2+\tau^2)\kappa^2
$$
$$
\langle \textcolor{blue}{\bb{X}_u} , \textcolor{blue}{\bb{X}_v} \rangle = \langle \textcolor{blue}{\bb{X}_v} , \textcolor{blue}{\bb{X}_u} \rangle = \langle (1 - v\kappa) \bb{T} + v \tau \bb{B}, \bb{N} \rangle = 0
$$
$$
\langle \textcolor{blue}{\bb{X}_v} , \textcolor{blue}{\bb{X}_v} \rangle = \langle \bb{N}, \bb{N} \rangle = 1
$$
$$
(I_p)_\beta =
\left(\begin{array}{cc}
    1 + 2 v\kappa + (v^2+\tau^2)\kappa^2 & 0 \\
    0 & 1 \\
\end{array}\right)
$$
Tenemos que la primera forma fundamental es
$$
\begin{array}{crcl}
I_p : & T_p\times T_p & \longrightarrow & \bb{R} \\
& ((x_1, y_1), (x_2, y_2)) & \longmapsto     & \left(\begin{array}{cc}
    x_1 & y_1 \\
\end{array}\right) (I_p)_\beta \left(\begin{array}{cc}
    x_2 \\
    y_2
\end{array}\right)
\end{array}
$$
Donde las coordenadas de los vectores de $T_p$ estan dadas en la base $\beta$.
$$
I_p((x_1, y_1), (x_2, y_2)) = x_1x_2(1 + 2 v\kappa + (v^2+\tau^2)\kappa^2) + y_1 y_2
$$
Para calcular la segunda forma fundamental comenzamos con el cálculo de la aplicación de Gauss(1777-1855).
$$
\textcolor{blue}{\bb{X}_u} \times \textcolor{blue}{\bb{X}_v} =
\left|\begin{array}{ccc}
    \bb{T} & \bb{N} & \bb{B} \\
    1 - v\kappa & 0 & v \tau \\
    0 & 1 & 0    
\end{array}\right|
= -v\tau \bb{T} + (1 - v\kappa) \bb{B}
$$
$$
N_p = \frac{\textcolor{blue}{\bb{X}_u} \times \textcolor{blue}{\bb{X}_v}}{\left\| \textcolor{blue}{\bb{X}_u} \times \textcolor{blue}{\bb{X}_v} \right\|
} = \frac{-v\tau \bb{T} + (1 - v\kappa) \bb{B}}{\sqrt{(\tau^2+\kappa^2)v^2-2v\kappa+1}}
$$
Calculamos las segundas desivadas de la carta $\textcolor{blue}{\bb{X}}$.
$$
\begin{array}{rl}
    \textcolor{blue}{\bb{X}_{uu}} = & -v\kappa'\bb{T} + (1- v\kappa) \kappa \bb{N} \\
    + & v (\tau' \bb{B} + \tau (-\tau \bb{B})) \\
    = & -v\kappa' \bb{T} + (\kappa - v\kappa^2) \bb{N} + (v\tau' - \tau^2)\bb{B} \\
    \textcolor{blue}{\bb{X}_{uv}} = & -\kappa \bb{T} + \tau \bb{B} \\
    \textcolor{blue}{\bb{X}_{vu}} = & -\kappa \bb{N} + \tau \bb{B}\\
    \textcolor{blue}{\bb{X}_{vv}} = & 0
\end{array}
$$
Ahora podemos calcular la segunda forma fundamental.
$$
\langle \textcolor{blue}{\bb{X}_{uu}}, N \rangle =
$$
$$
=\frac{\left\| -v\kappa' \bb{T} + (\kappa - v\kappa^2) \bb{N} + (v\tau' - \tau^2)\bb{B} \right\|^2}{\sqrt{(\tau^2+\kappa^2)v^2-2v\kappa+1}}=
$$
$$
=\frac{v^2\kappa'^2 + \kappa^2 + -2 v\kappa^3 + v^2 \kappa^4 + v^2\tau'^2 -2v\tau^2\tau'+\tau^4}{\sqrt{(\tau^2+\kappa^2)v^2-2v\kappa+1}}=
$$
$$
=\frac{(\kappa'^2+\kappa^4+\tau'^2)v^2 - 2(\kappa^3+\tau^2\tau') v +\kappa^2 + \tau^4}{\sqrt{(\tau^2+\kappa^2)v^2-2v\kappa+1}}
$$
$$
\langle \textcolor{blue}{\bb{X}_{uv}}, N \rangle =\langle \textcolor{blue}{\bb{X}_{vu}}, N \rangle =
$$
$$
=\frac{(-\kappa \bb{T} + \tau \bb{B})(-v\tau \bb{T} + (1 + v\kappa) \bb{B})}{\sqrt{(\tau^2+\kappa^2)v^2-2v\kappa+1}} =
$$
$$
=\frac{\tau + \kappa\tau v}{\sqrt{(\tau^2+\kappa^2)v^2-2v\kappa+1}}
$$
$$
\langle \textcolor{blue}{\bb{X}_{vv}}, N \rangle = 0
$$
$$
(II_p)_\beta =
\left(\begin{array}{cc}
    \langle \textcolor{blue}{\bb{X}_{uu}}, N \rangle & \langle \textcolor{blue}{\bb{X}_{uv}}, N \rangle \\
    \langle \textcolor{blue}{\bb{X}_{vu}}, N \rangle & \langle \textcolor{blue}{\bb{X}_{vv}}, N \rangle \\
\end{array}\right) =
$$
$$
\frac{
\left(\begin{array}{cc}
    (\kappa'^2+\kappa^4+\tau'^2)v^2 - 2(\kappa^3+\tau^2\tau') v +\kappa^2 + \tau^4 & \tau \\
    \tau & 0 \\
\end{array}\right)
}{\sqrt{(\tau^2+\kappa^2)v^2-2v\kappa+1}}
$$
Tenemos que la segunda forma fundamental es
$$
II_p((x_1,y_1),(x_2,y_2) =
\left(\begin{array}{cc}
    x_1 & y_1 \\
\end{array}\right) (II_p)_\beta \left(\begin{array}{cc}
    x_2 \\
    y_2
\end{array}\right)
$$

\section*{\textcolor{WildStrawberry}{Tarea 3}}
\noindent $\bullet$ Calcular la distancia entre el punto $\textcolor{blue}{(0,0,2)}$ y $\textcolor{blue}{\bb{S}^2}$.\\
\\
Veamos que $\textcolor{blue}{\bb{S}^2}$ es una 2-variedad de clase $C^1$ para aplicar el teorema de los multiplicadores de Lagrange (1736-1813). Tenemos que $\forall x \in \textcolor{blue}{\bb{S}^2}$ existe un abierto en $\bb{R}^3$, en este caso $\bb{R}^3$ es el abierto y una función $F:\bb{R}^3\rightarrow \R$ de clase $C^1$ tal que\\
i) $rg(DF(x))=1=3-2$\\
ii) $\textcolor{blue}{\bb{S}^2}\cap \bb{R}^3 = \{x\in \bb{R}^3 \::\: F(x)=0\}$\\
Es suficiente con tomar $F(x,y,z)= x^2+y^2+z^2-1$. ya que la condición ii) es evidente y $DF = (2x, 2y, 2z)$ solo es igual a 0 en el origen que no pertenece a $\textcolor{blue}{\bb{S}^2}$. Si $f(x,y,z)=x^2+y^2+(z-2)^2$ y $f_{|\textcolor{blue}{\bb{S}^2}}$ tiene un extremo relativo en $x_0$ entonces, $x_0$ es un punto estacionario de $g=f+\lambda F$.
$$
g = x^2+y^2+(z-2)^2 + \lambda (x^2+y^2+z^2-1)
$$
$$
Dg(x,y,x) = (2x(1+\lambda), 2y(1+\lambda), z(1+2\lambda)-2)
$$
$$
\left\{\begin{array}{l}
    0 = 2x(1+\lambda) \\
    0 = 2y(1+\lambda) \\
    0 = z(1+2\lambda)-2 \\
    x^2+y^2+z^2=1
\end{array}\right.
$$
Cuando $x=y=0$ tenemos que $z=\pm 1$. Cuando $x \ne 0$ 0 $y \ne 0$  es necesario que $\lambda = -1$ y se tiene que
$$
0 = z(1-2)-2 \quad \Rightarrow \quad z=-2
$$
pero no hay ningun punto en en $\textcolor{blue}{\bb{S}^2}$ con $z=-2$. Veamos cuanto vale $f$ en $(0,0,\pm 1)$.
$$
f(0,0,1) = 1
$$
$$
f(0,0,-1) = 9
$$
Tenemos que $(0,0,1)$ es el punto de $\textcolor{blue}{\bb{S}^2}$ más cercano a $(0,0,2)$ y por lo tanto $d(\textcolor{blue}{\bb{S}^2}, (0,0,2)) = 1$.

\section*{\textcolor{WildStrawberry}{Tarea 4}}
\noindent $\bullet$ Dada una curva reglar $\textcolor{blue}{\alpha : (f(t), g(t))}$, probar que la superficie de revolución dada por la carta $\textcolor{blue}{\mathcal{S}}: \textcolor{blue}{\bb{X}}(u,v)=(f(u)\cos(v), f(u)\sen(v), g(u))$ es orientable si $\textcolor{blue}{\mathcal{S}}\cap \langle (0,0,1) \rangle = \emptyset$. Ver bajo que condiciones es orientable cuando $\textcolor{blue}{\mathcal{S}}\cap \langle (0,0,1) \rangle \ne \emptyset$.\\
\\
Calculamos el campo normal unitario a la Superficie.
$$
\textcolor{blue}{\bb{X}_u} = (f'(u) \cos(v), f'(u) \sen(v), g'(u))
$$
$$
\textcolor{blue}{\bb{X}_v} = (-\textcolor{blue}{f(u)}\sen(v), \textcolor{blue}{f(u)}\cos(v), 0)
$$
$$
\textcolor{blue}{\bb{X}_u} \times \textcolor{blue}{\bb{X}_v} =
\left|\begin{array}{ccc}
    i & j & k \\
    f'(u) \cos(v) & f'(u) \sen(v) & g'(u) \\
    -\textcolor{blue}{f(u)}\sen(v) & \textcolor{blue}{f(u)}\cos(v) & 0
\end{array}\right|=
$$
$$
=i
\left|\begin{array}{cc}
    f'(u)\sen(v) & g'(u) \\
    \textcolor{blue}{f(u)}\cos(v) & 0 \\
\end{array}\right|
- j
\left|\begin{array}{cc}
    f'(u)\cos(v) & g'(u) \\
    -\textcolor{blue}{f(u)}\sen(v) & 0 \\
\end{array}\right|
$$
$$
+ k
\left|\begin{array}{cc}
    f'(u)\cos(v) & f'(u)\sen(v) \\
    -\textcolor{blue}{f(u)}\sen(v) & \textcolor{blue}{f(u)}\cos(v) \\
\end{array}\right| =
$$
$$
=i g'(u)\textcolor{blue}{f(u)}\cos(v) + j g'(u)\textcolor{blue}{f(u)}\sen(v) + k (f'(u)\textcolor{blue}{f(u)})=
$$
$$
=(g'(u)\textcolor{blue}{f(u)}\cos(v), g'(u)\textcolor{blue}{f(u)}\sen(v), (f'(u)\textcolor{blue}{f(u)}))
$$
$$
\frac{\textcolor{blue}{\bb{X}_u}\times \textcolor{blue}{\bb{X}_v}}{\left\| \textcolor{blue}{\bb{X}_u}\times \textcolor{blue}{\bb{X}_v} \right\|} =
$$
$$
= \frac{(g'(u)\textcolor{blue}{f(u)}\cos(v), g'(u)\textcolor{blue}{f(u)}\sen(v), (f'(u)\textcolor{blue}{f(u)}))}{\textcolor{blue}{f(u)}\sqrt{f'(u)^2+g'(u)^2}}
$$
Como la curva es regular $f'(u)$ y $g'(u)$ nunca se anulan a la vez y al no haber intersección con $\langle (0,0,1) \rangle$, tampoco se anula $\textcolor{blue}{f(u)}$. Por lo tanto el campo normal unitario es continuo y la superficie es orientable.\\
En el caso en que $\textcolor{blue}{\mathcal{S}}\cap \langle (0,0,1) \ne \emptyset$, si $f(u_0)=0$ es necesario que
$$
\lim_{u\to u_0} \frac{(g'(u)\cos(v),g'(u)\sen(v),f'(u))}{\sqrt{f'(u)^2+g'(u)^2}}
$$
no dependa de $v$, es decir $g'(u_0) \to 0$. Se tiene que si $\textcolor{blue}{\mathcal{S}}\cap \langle (0,0,1) \ne \emptyset$ y $\lim_{u\to u_0} g'(u) = 0$, entonces la superficie de revolución es orientable.
\end{multicols}
\end{document}