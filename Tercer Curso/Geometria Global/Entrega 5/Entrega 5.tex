\documentclass{article}
\usepackage[utf8]{inputenc}
\usepackage{graphicx}
\usepackage[spanish]{babel}
\usepackage{amssymb,amsmath,geometry,multicol,spalign,hyperref}
\usepackage[usenames,dvipsnames]{xcolor}
\usepackage{tikz,mathtools}
\usepackage{pgfplots}
\pgfplotsset{every axis/.append style={
                    axis x line=middle,    % put the x axis in the middle
                    axis y line=middle,    % put the y axis in the middle
                    axis line style={<->,color=blue}, % arrows on the axis
                    xlabel={$x$},          % default put x on x-axis
                    ylabel={$y$},          % default put y on y-axis
            }}
\usepackage{etoolbox} %titulo
\makeatletter %titulo
\patchcmd{\@maketitle}{\vskip 2em}{\vspace*{-3cm}}{}{} %titulo
\makeatother %titulo
\usepackage{vmargin}
\setpapersize{A4}
\setmargins{2.5cm}       % margen izquierdo
{1.5cm}                        % margen superior
{16.5cm}                      % anchura del texto
{23.42cm}                    % altura del texto
{10pt}                           % altura de los encabezados
{1cm}                           % espacio entre el texto y los encabezados
{0pt}                             % altura del pie de página
{2cm}                           % espacio entre el texto y el pie de página
\title{Entrega 5}
\author{Andoni Latorre Galarraga}
\date{}
\newcommand{\bb}[1]{\mathbb{#1}}
\newcommand{\R}{\bb{R}}
\newcommand{\nota}[3][2ex]{
    \underset{\mathclap{
        \begin{tikzpicture}
          \draw[->] (0, 0) to ++(0,#1);
          \node[below] at (0,0) {#3};
        \end{tikzpicture}}}{#2}
}
\begin{document}

\maketitle
\textcolor{WildStrawberry}{Problema:}\\
Probar que, si $\textcolor{blue}{\bb{X}:\mathcal{U}\in \bb{R}^2 \longrightarrow \bb{X}(\mathcal{U})\subset S }$, tal que $F = 0$, entonces
$$
K = - \frac{1}{2\sqrt{EG}}\left[ \left(\frac{E_v}{\sqrt{EG}}\right)_v + \left(\frac{G_u}{\sqrt{EG}}\right)_u \right]
$$

\textcolor{WildStrawberry}{Solución:}\\
Por definición de los símbolos de Christoffel
$$
\left\{\begin{array}{l}
    \textcolor{blue}{\bb{X}}_{uu} = \Gamma_{11}^1 \textcolor{blue}{\bb{X}}_u + \Gamma_{11}^2 \textcolor{blue}{\bb{X}}_v + e N \\
    \textcolor{blue}{\bb{X}}_{uv} = \Gamma_{12}^1 \textcolor{blue}{\bb{X}}_u + \Gamma_{12}^2 \textcolor{blue}{\bb{X}}_v + f N \\
    \textcolor{blue}{\bb{X}}_{vv} = \Gamma_{22}^1 \textcolor{blue}{\bb{X}}_u + \Gamma_{22}^2 \textcolor{blue}{\bb{X}}_v + g N
\end{array}\right.
$$
Como $F = 0$, $\textcolor{blue}{\bb{X}}_u \textcolor{blue}{\bb{X}}_v = 0$. Despejamos los 6 símbolos de Christoffel.

$$
\textcolor{blue}{\bb{X}}_u \textcolor{blue}{\bb{X}}_{uu} = \Gamma_{11}^1 \textcolor{blue}{\bb{X}}_u \textcolor{blue}{\bb{X}}_u + \Gamma_{11}^2 \textcolor{blue}{\bb{X}}_u\textcolor{blue}{\bb{X}}_v + e \textcolor{blue}{\bb{X}}_u N
$$
$$
\textcolor{blue}{\bb{X}}_u \textcolor{blue}{\bb{X}}_{uu} = \Gamma_{11}^1 E
$$

$$
\textcolor{blue}{\bb{X}}_u \textcolor{blue}{\bb{X}}_{uv} = \Gamma_{12}^1 \textcolor{blue}{\bb{X}}_u \textcolor{blue}{\bb{X}}_u + \Gamma_{12}^2 \textcolor{blue}{\bb{X}}_u\textcolor{blue}{\bb{X}}_v + f \textcolor{blue}{\bb{X}}_u N
$$
$$
\textcolor{blue}{\bb{X}}_u \textcolor{blue}{\bb{X}}_{uv} = \Gamma_{12}^1 E
$$

$$
\textcolor{blue}{\bb{X}}_u \textcolor{blue}{\bb{X}}_{vv} = \Gamma_{22}^1 \textcolor{blue}{\bb{X}}_u \textcolor{blue}{\bb{X}}_u + \Gamma_{22}^2 \textcolor{blue}{\bb{X}}_u\textcolor{blue}{\bb{X}}_v + g \textcolor{blue}{\bb{X}}_u N
$$
$$
\textcolor{blue}{\bb{X}}_u \textcolor{blue}{\bb{X}}_{vv} = \Gamma_{22}^1 E
$$

$$
\textcolor{blue}{\bb{X}}_v \textcolor{blue}{\bb{X}}_{uu} = \Gamma_{11}^1 \textcolor{blue}{\bb{X}}_v \textcolor{blue}{\bb{X}}_u + \Gamma_{11}^2 \textcolor{blue}{\bb{X}}_v\textcolor{blue}{\bb{X}}_v + e \textcolor{blue}{\bb{X}}_u N
$$
$$
\textcolor{blue}{\bb{X}}_v \textcolor{blue}{\bb{X}}_{uu} = \Gamma_{11}^2 G
$$

$$
\textcolor{blue}{\bb{X}}_v \textcolor{blue}{\bb{X}}_{uv} = \Gamma_{12}^1 \textcolor{blue}{\bb{X}}_v \textcolor{blue}{\bb{X}}_u + \Gamma_{12}^2 \textcolor{blue}{\bb{X}}_v\textcolor{blue}{\bb{X}}_v + f \textcolor{blue}{\bb{X}}_v N
$$
$$
\textcolor{blue}{\bb{X}}_v \textcolor{blue}{\bb{X}}_{uv} = \Gamma_{12}^2 G
$$

$$
\textcolor{blue}{\bb{X}}_v \textcolor{blue}{\bb{X}}_{vv} = \Gamma_{22}^1 \textcolor{blue}{\bb{X}}_v \textcolor{blue}{\bb{X}}_u + \Gamma_{22}^2 \textcolor{blue}{\bb{X}}_v\textcolor{blue}{\bb{X}}_v + g \textcolor{blue}{\bb{X}}_v N
$$
$$
\textcolor{blue}{\bb{X}}_v \textcolor{blue}{\bb{X}}_{vv} = \Gamma_{22}^2 G
$$
Ahora observamos que
$$
(\textcolor{blue}{\bb{X}}_u\textcolor{blue}{\bb{X}}_u)_u = \textcolor{blue}{\bb{X}}_u \textcolor{blue}{\bb{X}}_{uu} + \textcolor{blue}{\bb{X}}_{uu} \textcolor{blue}{\bb{X}}_u =
2 \textcolor{blue}{\bb{X}}_{uu}\textcolor{blue}{\bb{X}}_u
$$
$$
E_u = 2 \textcolor{blue}{\bb{X}}_{uu}\textcolor{blue}{\bb{X}}_u
$$

$$
(\textcolor{blue}{\bb{X}}_u\textcolor{blue}{\bb{X}}_u)_v = \textcolor{blue}{\bb{X}}_u \textcolor{blue}{\bb{X}}_{uv} + \textcolor{blue}{\bb{X}}_{uv} \textcolor{blue}{\bb{X}}_u =
2 \textcolor{blue}{\bb{X}}_{uv}\textcolor{blue}{\bb{X}}_u
$$
$$
E_v = 2 \textcolor{blue}{\bb{X}}_{uv}\textcolor{blue}{\bb{X}}_u
$$

$$
(\textcolor{blue}{\bb{X}}_v\textcolor{blue}{\bb{X}}_v)_u = \textcolor{blue}{\bb{X}}_v \textcolor{blue}{\bb{X}}_{vu} + \textcolor{blue}{\bb{X}}_{vu} \textcolor{blue}{\bb{X}}_v =
2 \textcolor{blue}{\bb{X}}_{vu}\textcolor{blue}{\bb{X}}_v
$$
$$
G_u = 2 \textcolor{blue}{\bb{X}}_{vu}\textcolor{blue}{\bb{X}}_v
$$

$$
(\textcolor{blue}{\bb{X}}_v\textcolor{blue}{\bb{X}}_v)_v = \textcolor{blue}{\bb{X}}_v \textcolor{blue}{\bb{X}}_{vv} + \textcolor{blue}{\bb{X}}_{vv} \textcolor{blue}{\bb{X}}_v =
2 \textcolor{blue}{\bb{X}}_{vv}\textcolor{blue}{\bb{X}}_v
$$
$$
G_v = 2 \textcolor{blue}{\bb{X}}_{vv}\textcolor{blue}{\bb{X}}_v
$$
De donde obtenemos las siguientes expresiones para los símbolos de Christoffel
$$
\Gamma_{11}^1 = \frac{E_u}{2E}
$$
$$
\Gamma_{12}^1 = \frac{E_v}{2E}
$$
$$
\Gamma_{22}^1 = \frac{-G_u}{2E}
$$
$$
\Gamma_{11}^2 = \frac{-E_v}{2G}
$$
$$
\Gamma_{12}^2 = \frac{G_u}{2G}
$$
$$
\Gamma_{22}^2 = \frac{G_v}{2G}
$$
Sustituimos en la ecuación de la curvatura en términos de los símbolos de Christoffel.
$$
K = - \frac{1}{E} \left( \left( \Gamma_{12}^2 \right)_u - \left( \Gamma_{11}^2 \right)_v + \Gamma_{12}^1\Gamma_{11}^2 - \Gamma_{11}^1\Gamma_{12}^2 + \left( \Gamma_{12}^2 \right)^2 - \Gamma_{11}^2\Gamma_{22}^2 \right) =
$$
$$
= - \frac{1}{E} \left( \left( \frac{G_u}{2G} \right)_u - \left(\frac{-E_v}{2G} \right)_v + \frac{E_v}{2E}\frac{-E_v}{2G} - \frac{E_u}{2E}\frac{G_u}{2G} + \left( \frac{G_u}{2G} \right)^2 - \frac{-E_v}{2G}\frac{G_v}{2G} \right) =
$$
$$
= - \frac{1}{2E} \left( \left( \frac{G_u}{G} \right)_u + \left(\frac{E_v}{G} \right)_v - \frac{E_v^2}{2EG} - \frac{E_uG_u}{2EG} + \frac{G_u^2}{2G^2} + \frac{E_vG_v}{2G^2} \right) =
$$
$$
= - \frac{1}{2E} \left( \frac{G_{uu}G-G_uG_u}{G^2} + \frac{E_{vv}G-E_vG_v}{G^2} - \frac{E_v^2}{2EG} - \frac{E_uG_u}{2EG} + \frac{G_u^2}{2G^2} + \frac{E_vG_v}{2G^2} \right) =
$$
$$
= - \frac{1}{2EG} \left( G_{uu}-\frac{G_u^2}{G} + E_{vv} - \frac{E_vG_v}{G} - \frac{E_v^2}{2E} - \frac{E_uG_u}{2E} + \frac{G_u^2}{2G} + \frac{E_vG_v}{2G} \right)
$$
$$
= - \frac{1}{2EG} \left( G_{uu} + E_{vv} - \frac{E_v^2}{2E} - \frac{E_uG_u}{2E} - \frac{G_u^2}{2G} - \frac{E_vG_v}{2G} \right)
$$
$$
= - \frac{1}{2EG} \left( G_{uu} + E_{vv} - E_v\frac{E_vG}{2EG} - G_u\frac{E_uG}{2EG} - G_u\frac{EG_u}{2EG} - E_v\frac{EG_v}{2EG} \right)
$$
$$
= - \frac{1}{2EG}\left( E_{vv} - E_v \frac{E_vG+EG_v}{2EG} + G_{uu}- G_u \frac{E_uG+EG_u}{2EG} \right)
$$
$$
= - \frac{1}{2}\left( \frac{E_{vv} - E_v \frac{E_vG+EG_v}{2EG}}{EG} + \frac{G_{uu}- G_u \frac{E_uG+EG_u}{2EG}}{EG} \right)
$$
$$
= - \frac{1}{2\sqrt{EG}}\left( \frac{E_{vv}\sqrt{EG} - E_v \frac{E_vG+EG_v}{2\sqrt{EG}}}{EG} + \frac{G_{uu}\sqrt{EG}- G_u \frac{E_uG+EG_u}{2\sqrt{EG}}}{EG} \right)
$$
$$
= - \frac{1}{2\sqrt{EG}}\left( \left(\frac{E_v}{\sqrt{EG}}\right)_v + \left(\frac{G_u}{\sqrt{EG}}\right)_u \right)
$$
\end{document}