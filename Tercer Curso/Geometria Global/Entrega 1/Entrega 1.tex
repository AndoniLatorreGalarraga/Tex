\documentclass{article}
\usepackage[utf8]{inputenc}
\usepackage{graphicx}
\usepackage[spanish]{babel}
\usepackage{amssymb,amsmath,geometry,multicol,spalign}
\usepackage[usenames,dvipsnames]{xcolor}
\usepackage{tikz,mathtools}
\usepackage{etoolbox} %titulo
\makeatletter %titulo
\patchcmd{\@maketitle}{\vskip 2em}{\vspace*{-3cm}}{}{} %titulo
\makeatother %titulo
\usepackage{vmargin}
\setpapersize{A4}
\setmargins{2.5cm}       % margen izquierdo
{1.5cm}                        % margen superior
{16.5cm}                      % anchura del texto
{23.42cm}                    % altura del texto
{10pt}                           % altura de los encabezados
{1cm}                           % espacio entre el texto y los encabezados
{0pt}                             % altura del pie de página
{2cm}                           % espacio entre el texto y el pie de página
\title{Entrega 1}
\author{Andoni Latorre Galarraga}
\date{}
\newcommand{\bb}[1]{\mathbb{#1}}
\newcommand{\nota}[3][2ex]{
    \underset{\mathclap{
        \begin{tikzpicture}
          \draw[->] (0, 0) to ++(0,#1);
          \node[below] at (0,0) {#3};
        \end{tikzpicture}}}{#2}
}
\begin{document}

\maketitle

\section*{\textcolor{WildStrawberry}{Conceptos}}

\textcolor{red}{Topología}\\
Una \textcolor{red}{topología} $\textcolor{blue}{\tau_X}\subseteq \mathcal{P}(\textcolor{blue}{X})$ sobre un conjunto $\textcolor{blue}{X}$ es un subconjunto del conjunto potencia de $\textcolor{blue}{X}$ que cumple las siguentes propiedades,
$$
\begin{array}{ll}
    i) & \{\emptyset, \textcolor{blue}{X}\}\subseteq \textcolor{blue}{\tau_X} \\
    ii) & \forall\mathcal{U},\mathcal{V}\in \textcolor{blue}{\tau_X}, \quad \mathcal{U}\cap\mathcal{V}\in \textcolor{blue}{\tau_x} \\
    iii) & \forall \sigma \subseteq \textcolor{blue}{\tau_x}, \quad \bigcup_{\mathcal{U}\in \sigma} \mathcal{U} \in \textcolor{blue}{\tau_X}
\end{array}
$$

\textcolor{red}{Espacio topológico}\\
Un \textcolor{red}{espacio topológico} es un par ordenado $\textcolor{blue}{(X,\tau_X)}$ donde $\textcolor{blue}{X}$ es un conjunto y $\textcolor{blue}{\tau_X}$ es una topología sobre $\textcolor{blue}{X}$.\\

$\textcolor{red}{T_2}$ \textcolor{red}{(de Hausdorff)}\\
Un espacio topológico $\textcolor{blue}{(X,\tau_X)}$ se dice $\textcolor{red}{T_2}$ si
$$
\forall a,b\in \textcolor{blue}{X} \text{ con } a \ne b, \: \exists \mathcal{U},\mathcal{V}\in \textcolor{blue}{\tau_X} \: : \: a\in\mathcal{U}, b\in\mathcal{V}, \mathcal{U}\cap\mathcal{V}=\emptyset
$$
, es decir, todo par de puntos distintos se puede separar por abiertos disjuntos.\\

\textcolor{red}{Compacto}\\
Un espacio topológico $\textcolor{blue}{(X,\tau_X)}$ se dice \textcolor{red}{compacto} si
$$
\forall \{\mathcal{U}_i\}_{i\in I} \subseteq \textcolor{blue}{\tau_X}\: :\: \bigcup_{i\in I} \mathcal{U}_i = \textcolor{blue}{X} \quad, \exists J\subseteq I \: :\: |J|\le\infty, \bigcup_{j\in J} \mathcal{U}_j = \textcolor{blue}{X}
$$
, es decir, todo recubrimiento por abiertos de $\textcolor{blue}{X}$ tiene un subrecubrimiento finito.\\

\textcolor{red}{Aplicación continua}\\
Se dice que la aplicación $\textcolor{blue}{f:(X,\tau_X)\longrightarrow (Y, \tau_Y)}$ es \textcolor{red}{continua} si
$$
\forall \mathcal{U}\in \textcolor{blue}{\tau_Y}, f^{-1}(\mathcal{U})\in \textcolor{blue}{\tau_X}
$$
, es decir, la antiimagen de cualquier abierto es un abierto.\\\\

\textcolor{red}{Aplicación topologicamente cerrada}\\
Se dice que la aplicación $\textcolor{blue}{f:(X,\tau_X)\longrightarrow (Y, \tau_Y)}$ es \textcolor{red}{cerrada} si
$$
\forall \mathcal{U} \in \textcolor{blue}{\tau_x}, \quad \textcolor{blue}{f}(\mathcal{U}^c)^c\in \textcolor{blue}{\tau_Y}
$$
, es decir, la imagen de todo cerrado es un cerrado.\\\\

\textcolor{red}{Homeomorfismo}\\
Se dice que la aplicación $\textcolor{blue}{f:(X,\tau_X)\longrightarrow (Y, \tau_Y)}$ es un \textcolor{red}{homeomorfismo} si es continua, biyectiva y su inversa es continua.\\

\textcolor{red}{Homeomorfo}\\
Los espacios topológicos $\textcolor{blue}{(X,\tau_x)}$ y $\textcolor{blue}{(Y,\tau_Y)}$ se dicen \textcolor{red}{homeomorfos} si existe un homeomorfismo $\textcolor{blue}{f:(X,\tau_X)\longrightarrow (Y, \tau_Y)}$.\\

\textcolor{red}{Identificación}\\
Se dice que la aplicación $\textcolor{blue}{f:(X,\tau_X)\longrightarrow (Y, \tau_Y)}$ es una \textcolor{red}{identificación} si es sobreyectiva y $\mathcal{U}\in \textcolor{blue}{\tau_Y} \Leftrightarrow f^{-1}(\mathcal{U})\in \textcolor{blue}{\tau_X}$.\\

\textcolor{red}{Cociente}\\
Dada una aplicacón sobreyectiva $\textcolor{blue}{f:(X, \tau_X) \longleftarrow Y}$ se define la \textcolor{red}{topología cociente} sobre $\textcolor{blue}{Y}$ como:
$$
\textcolor{blue}{\tau_Y} = \textcolor{blue}{\{\mathcal{V}\subseteq Y \: :\: f^{-1}(\mathcal{V})\in\tau_X\}}
$$

\end{document}