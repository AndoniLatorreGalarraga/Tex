\documentclass{article}
\usepackage[utf8]{inputenc}
\usepackage{graphicx}
\usepackage[spanish]{babel}
\usepackage{amssymb,amsmath,geometry,multicol,spalign}
\usepackage[usenames,dvipsnames]{xcolor}
\usepackage{tikz,mathtools}
\usepackage{etoolbox} %titulo
\makeatletter %titulo
\patchcmd{\@maketitle}{\vskip 2em}{\vspace*{-3cm}}{}{} %titulo
\makeatother %titulo
\usepackage{vmargin}
\setpapersize{A4}
\setmargins{2.5cm}       % margen izquierdo
{1.5cm}                        % margen superior
{16.5cm}                      % anchura del texto
{23.42cm}                    % altura del texto
{10pt}                           % altura de los encabezados
{1cm}                           % espacio entre el texto y los encabezados
{0pt}                             % altura del pie de página
{2cm}                           % espacio entre el texto y el pie de página
\title{Entrega 2}
\author{Andoni Latorre Galarraga}
\date{}
\newcommand{\bb}[1]{\mathbb{#1}}
\newcommand{\nota}[3][2ex]{
    \underset{\mathclap{
        \begin{tikzpicture}
          \draw[->] (0, 0) to ++(0,#1);
          \node[below] at (0,0) {#3};
        \end{tikzpicture}}}{#2}
}
\begin{document}

\maketitle
\textcolor{WildStrawberry}{Proposición:}\\
Sean $\textcolor{blue}{\tilde{\Phi}_1, \tilde{\Phi}_2:X\longrightarrow \bb{R}}$ dos elevaciones de una aplicación continua $\textcolor{blue}{\Phi: X \longrightarrow \bb{S}^1}$, donde $\textcolor{blue}{X}$ es un espacio topológico conexo, entonces $\textcolor{red}{\exists k\in \bb{Z} \: :\: \tilde{\Phi}_2-\tilde{\Phi}_1 = 2k\pi}$.\\

\textcolor{WildStrawberry}{\textit{Dem:}}\\
Por ser $\textcolor{blue}{\tilde{\Phi}_1},\textcolor{blue}{\tilde{\Phi}_2}$ elevaciones de $\textcolor{blue}{\Phi}$, tenmos que $\textcolor{blue}{\tilde{\Phi}} = exp \circ \textcolor{blue}{\tilde{\Phi}_1} = exp \circ \textcolor{blue}{\Phi}_2$. Por la periodicidad de $exp$, tenemos que $\textcolor{blue}{\tilde{\Phi}_1} - \textcolor{blue}{\tilde{\Phi}_2} = 2\pi k(x)$ donde $k:\textcolor{blue}{X}\rightarrow \bb{Z}$. Como $\textcolor{blue}{X}$ es conexo, $(\frac{\textcolor{blue}{\tilde{\Phi}_1}- \textcolor{blue}{\tilde{\Phi}_2}}{2\pi})(\textcolor{blue}{X})=k(\textcolor{blue}{X})$ es conexo ya que $\frac{\textcolor{blue}{\tilde{\Phi}_1}- \textcolor{blue}{\tilde{\Phi}_2}}{2\pi}=k$ es continua. Pero los conexos en $\bb{Z}$ son los puntos por lo tanto, $k$ es contante.
\end{document}