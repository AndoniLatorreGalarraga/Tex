\documentclass{article}
\usepackage[utf8]{inputenc}
\usepackage{graphicx}
\usepackage[spanish]{babel}
\usepackage{amssymb,amsmath,geometry,multicol,spalign,hyperref}
\usepackage[usenames,dvipsnames]{xcolor}
\usepackage{tikz,mathtools}
\usepackage{pgfplots}
\pgfplotsset{every axis/.append style={
                    axis x line=middle,    % put the x axis in the middle
                    axis y line=middle,    % put the y axis in the middle
                    axis line style={<->,color=blue}, % arrows on the axis
                    xlabel={$x$},          % default put x on x-axis
                    ylabel={$y$},          % default put y on y-axis
            }}
\usepackage{etoolbox} %titulo
\makeatletter %titulo
\patchcmd{\@maketitle}{\vskip 2em}{\vspace*{-3cm}}{}{} %titulo
\makeatother %titulo
\usepackage{vmargin}
\setpapersize{A4}
\setmargins{2.5cm}       % margen izquierdo
{1.5cm}                        % margen superior
{16.5cm}                      % anchura del texto
{23.42cm}                    % altura del texto
{10pt}                           % altura de los encabezados
{1cm}                           % espacio entre el texto y los encabezados
{0pt}                             % altura del pie de página
{2cm}                           % espacio entre el texto y el pie de página
\title{Entrega 8}
\author{Andoni Latorre Galarraga}
\date{}
\newcommand{\bb}[1]{\mathbb{#1}}
\newcommand{\R}{\bb{R}}
\newcommand{\nota}[3][2ex]{
    \underset{\mathclap{
        \begin{tikzpicture}
          \draw[->] (0, 0) to ++(0,#1);
          \node[below] at (0,0) {#3};
        \end{tikzpicture}}}{#2}
}
\begin{document}

\maketitle
\section*{\textcolor{red}{Definiciones}}

\textcolor{WildStrawberry}{Campo vectorial sobre una curva}\\
Si $\textcolor{blue}{\alpha}: I\subseteq \bb{R}^2 \longrightarrow \mathcal{S}\subseteq \bb{R}^3$ es una curva sobre una superficie $\mathcal{S}$, un \textcolor{WildStrawberry}{campo vectorial sobre} $\textcolor{blue}{\alpha}$ es una aplicación $w:I\longrightarrow \bb{R}^3$. Este campo se dice \textcolor{WildStrawberry}{diferenciable} si $w$ es diferenciable y se dice \textcolor{WildStrawberry}{tangente} si satisface $w(t)\in T_{\textcolor{blue}{\alpha}(t)}(\mathcal{S})$ para todo $t\in I$.\\\\

\textcolor{WildStrawberry}{Derivada covariante de un campo sobre un curva}\\
Si $\textcolor{blue}{w}: I \subseteq \bb{R} \longrightarrow \bb{R}^3$ es un campo vectorial diferenciable sobre $\textcolor{blue}{\alpha}: I\subseteq \bb{R}^2 \longrightarrow \textcolor{blue}{\mathcal{S}}\subseteq \bb{R}^3$, con $\textcolor{blue}{\mathcal{S}}$ superficie regular. Entonces, la \textcolor{WildStrawberry}{derivada covariante de} $\textcolor{blue}{w}$ es la componente tangencial de $\textcolor{blue}{w}'$. Se escribe $\frac{D \textcolor{blue}{w}}{dt}$. Ademas, si $N$ es un campo normal unitario de $\textcolor{blue}{\mathcal{S}}$, la componente normal de $\textcolor{blue}{w}'$ es $\textcolor{blue}{w}'\cdot N(\textcolor{blue}{\alpha})$ y se tiene que
$$
\frac{D \textcolor{blue}{w}}{dt}(t) = \textcolor{blue}{w}'(t) - \langle \textcolor{blue}{w}', N(\textcolor{blue}{\alpha}(t)) \rangle N(\textcolor{blue}{\alpha}(t))
$$

\textcolor{WildStrawberry}{Valor algebraico de la derivada covariante}\\
Si $\textcolor{blue}{w}: I \subseteq \bb{R} \longrightarrow \bb{R}^3$ es un campo vectorial diferenciable, tangente y unitario a lo largo de una curva $\textcolor{blue}{\alpha}: I\subseteq \bb{R}^2 \longrightarrow \textcolor{blue}{\mathcal{S}}\subseteq \bb{R}^3$ sobre una superficie orientada $\textcolor{blue}{\mathcal{S}}$. Observamos que $\frac{D \textcolor{blue}{w}}{dt}\cdot N(\textcolor{blue}{\alpha}(t)) = 0$ y $\textcolor{blue}{w}(t)\cdot \textcolor{blue}{w}'(t)=0$, por ser la derivada covariante parte de $T_{\textcolor{blue}{\alpha}(t)}(\textcolor{blue}{\mathcal{S}})$ y por ser $\textcolor{blue}{w}$ unitario. Deducimos
$$
\frac{D \textcolor{blue}{w}}{dt}(t) = \lambda(t) (N(\textcolor{blue}{\alpha}(t)) \land \textcolor{blue}{w}(t) )
$$
Llamamos \textcolor{WildStrawberry}{valor algebraico de la derivada covariante} a $\lambda(t)$ y escribimos $\lambda(t) = \left[\frac{D \textcolor{blue}{w}}{dt}(t)\right]$.\\\\

\textcolor{WildStrawberry}{Curvatura geodésica}\\
Si $\textcolor{blue}{\alpha}: I\subseteq \bb{R}^2 \longrightarrow \textcolor{blue}{\mathcal{S}}\subseteq \bb{R}^3$ es una curva parametrizada por longitud de arco sobre una superficie regular orientada $\textcolor{blue}{\mathcal{S}}$. Llamamos \textcolor{WildStrawberry}{curvatura geodésica de} $\textcolor{blue}{\alpha}$ a $\left[\frac{D \textcolor{blue}{\alpha}'}{dt}\right]$. Escribimos
$$
\left[\frac{D \textcolor{blue}{\alpha}'}{dt}(t)\right] = k_g(t)
$$\\\\

\textcolor{WildStrawberry}{Geodésica}\\
Si $\textcolor{blue}{\alpha}: I\subseteq \bb{R}^2 \longrightarrow \textcolor{blue}{\mathcal{S}}\subseteq \bb{R}^3$ es una curva regular sobre una superficie regular orientada $\textcolor{blue}{\mathcal{S}}$. Se dice que $\textcolor{blue}{\alpha}$ es \textcolor{WildStrawberry}{geodésica} si
$$
\frac{D \textcolor{blue}{\alpha}'}{dt}(t) = 0 \quad \forall t
$$
Es decir $\textcolor{blue}{\alpha}'$ es paralelo a lo largo de $\textcolor{blue}{\alpha}$.\\\\

\textcolor{WildStrawberry}{Proposición:}\\
Las geodésicas están parametrizadas por parámetro proporcional a la longitud de arco. Es decir, Si $\textcolor{blue}{\alpha}: I\subseteq \bb{R}^2 \longrightarrow \textcolor{blue}{\mathcal{S}}\subseteq \bb{R}^3$ es una geodésica. Entonces, $\| \textcolor{blue}{\alpha}' \|$ es constante.

\textcolor{WildStrawberry}{Dem:}\\
Si no es constante $\| \textcolor{blue}{\alpha}'(t) \|^2 = 2 n(t)$. Derivando, $\textcolor{blue}{\alpha}'\textcolor{blue}{\alpha}'' = n'(t)$. Entonces, para algún $t_0$ se tiene $\textcolor{blue}{\alpha}'\textcolor{blue}{\alpha}'' \ne 0$. Como $\textcolor{blue}{\alpha}'$ está en el espacio tangente, ahora es imposible que $\textcolor{blue}{\alpha}''$ tenga derivada covariante nula ya que tiene componente tangencial. Esto contradice que $\textcolor{blue}{\alpha}$ sea geodésica y queda probada la proposición.\\\\

\textcolor{WildStrawberry}{Proposición:}\\
Si $\textcolor{blue}{f}: \bb{R}^3 \longrightarrow \bb{R}^3$ es una isometría y $\textcolor{blue}{\alpha}$ es geodésica en $\textcolor{blue}{\mathcal{S}}$. Entonces, $\textcolor{blue}{f}(\textcolor{blue}{\alpha})$ es geodésica sobre $\textcolor{blue}{f}(\textcolor{blue}{\mathcal{S}})$.

\textcolor{WildStrawberry}{Dem:}\\
Si $\textcolor{blue}{\alpha} = \bb{X}(u,v)$ entonces $\textcolor{blue}{f}(\textcolor{blue}{\alpha})=(\textcolor{blue}{f}\circ\bb{X})(u,v)$ y se tiene
$$
\textcolor{blue}{f}(\textcolor{blue}{\alpha})' = (\textcolor{blue}{f}\circ\bb{X})_u u' + (\textcolor{blue}{f}\circ\bb{X})_v v'
$$
$$
\textcolor{blue}{f}(\textcolor{blue}{\alpha})'' = (\textcolor{blue}{f}\circ\bb{X})_{uu} (u')^2 + (\textcolor{blue}{f}\circ\bb{X})_{uv} u'v' + (\textcolor{blue}{f}\circ\bb{X})_u u'' + (\textcolor{blue}{f}\circ\bb{X})_{vu} u'v' + (\textcolor{blue}{f}\circ\bb{X})_{vv} (v')^2 + (\textcolor{blue}{f}\circ\bb{X})_v v''
$$
Por definición de los símbolos de Christoffel,
$$
\begin{array}{ll}
\textcolor{blue}{f}(\textcolor{blue}{\alpha})'' &
   = (\textcolor{blue}{f}\circ\bb{X})_{uu} (u')^2 \\
 & + (\textcolor{blue}{f}\circ\bb{X})_{uv} u'v' \\
 & + (\textcolor{blue}{f}\circ\bb{X})_u u'' \\
 & + (\textcolor{blue}{f}\circ\bb{X})_{vu} u'v' \\
 & + (\textcolor{blue}{f}\circ\bb{X})_{vv} (v')^2 \\
 & + (\textcolor{blue}{f}\circ\bb{X})_v v''
\end{array}
\Rightarrow
\begin{array}{ll}
\frac{D\textcolor{blue}{f}(\textcolor{blue}{\alpha})'}{dt} &
   = (\Gamma_{11}^1 (\textcolor{blue}{f}\circ\bb{X})_u + \Gamma_{11}^2 (\textcolor{blue}{f}\circ\bb{X})_v ) (u')^2 \\
 & + (\Gamma_{12}^1 (\textcolor{blue}{f}\circ\bb{X})_u + \Gamma_{12}^2 (\textcolor{blue}{f}\circ\bb{X})_v ) u'v' \\
 & + (\textcolor{blue}{f}\circ\bb{X})_u u'' \\
 & + (\Gamma_{12}^1 (\textcolor{blue}{f}\circ\bb{X})_u + \Gamma_{12}^2 (\textcolor{blue}{f}\circ\bb{X})_v ) u'v' \\
 & + (\Gamma_{22}^1 (\textcolor{blue}{f}\circ\bb{X})_u + \Gamma_{22}^2 (\textcolor{blue}{f}\circ\bb{X})_v ) (v')^2 \\
 & + (\textcolor{blue}{f}\circ\bb{X})_v v'' =
\end{array}
$$
$$
\frac{D\textcolor{blue}{f}(\textcolor{blue}{\alpha})'}{dt} =
(\textcolor{blue}{f}\circ\bb{X})_u \left( \Gamma_{11}^1 (u')^2 + 2\Gamma_{12}^1 u'v' + u'' + \Gamma_{22}^1 (v')^2 \right) +
(\textcolor{blue}{f}\circ\bb{X})_v \left( \Gamma_{11}^2 (u')^2 + 2\Gamma_{12}^2 u'v' + \Gamma_{22}^2 (v')^2 + v''\right)
$$
Como $u$ y $v$ satisfacen las ecuaciones diferenciales de las geodésicas, la derivada covariante es nula y $\textcolor{blue}{f}(\textcolor{blue}{\alpha})$ es geodésica.\\\\

\textcolor{WildStrawberry}{Proposición:}\\
Si $\textcolor{blue}{\alpha}: I\subseteq \bb{R}^2 \longrightarrow \textcolor{blue}{\mathcal{S}}\subseteq \bb{R}^3$ es una curva regular parametrizada por longitud de arco sobre una superficie regular orientada $\textcolor{blue}{\mathcal{S}}$ Entonces, $k(s)^2=k_n(s)^2+k_g(s)^2$.

\textcolor{WildStrawberry}{Dem:}\\
Por definición de curvatura normal, $k_n = k (\bb{N}\cdot N)$ donde $N$ es el vector normal a la superficie y $\bb{N}$ es el vector normal a la curva. Por ser $\textcolor{blue}{\alpha}$ parametrizada por longitud de arco, $k\bb{N}=\textcolor{blue}{\alpha}''$. Se tiene que $k_n = \textcolor{blue}{\alpha}''\cdot N$. Ahora, como $\textcolor{blue}{\alpha}'$ es unitario $\frac{D \textcolor{blue}{\alpha'}}{dt} = k_g (N\land \textcolor{blue}{\alpha'})$.
$$
\textcolor{blue}{\alpha}'' = (\textcolor{blue}{\alpha}''\cdot N)N + k_g (N\land \textcolor{blue}{\alpha'})
$$
$$
k \bb{N} = k_n N + k_g (N\land \textcolor{blue}{\alpha'})
$$
Tomado $\| \cdot \|^2$ a ambos lados se tiene $k^2 = k_n^2 + k_g^2$.
\end{document}