\documentclass{article}
\usepackage[utf8]{inputenc}
\usepackage{graphicx}
\usepackage[spanish]{babel}
\usepackage{amssymb,amsmath,geometry,multicol,spalign,hyperref}
\usepackage[usenames,dvipsnames]{xcolor}
\usepackage{tikz,mathtools}
\usepackage{pgfplots}
\pgfplotsset{every axis/.append style={
                    axis x line=middle,    % put the x axis in the middle
                    axis y line=middle,    % put the y axis in the middle
                    axis line style={<->,color=blue}, % arrows on the axis
                    xlabel={$x$},          % default put x on x-axis
                    ylabel={$y$},          % default put y on y-axis
            }}
\usepackage{etoolbox} %titulo
\makeatletter %titulo
\patchcmd{\@maketitle}{\vskip 2em}{\vspace*{-3cm}}{}{} %titulo
\makeatother %titulo
\usepackage{vmargin}
\setpapersize{A4}
\setmargins{2.5cm}       % margen izquierdo
{1.5cm}                        % margen superior
{16.5cm}                      % anchura del texto
{23.42cm}                    % altura del texto
{10pt}                           % altura de los encabezados
{1cm}                           % espacio entre el texto y los encabezados
{0pt}                             % altura del pie de página
{2cm}                           % espacio entre el texto y el pie de página
\title{Entrega 6}
\author{Andoni Latorre Galarraga}
\date{}
\newcommand{\bb}[1]{\mathbb{#1}}
\newcommand{\R}{\bb{R}}
\newcommand{\nota}[3][2ex]{
    \underset{\mathclap{
        \begin{tikzpicture}
          \draw[->] (0, 0) to ++(0,#1);
          \node[below] at (0,0) {#3};
        \end{tikzpicture}}}{#2}
}
\begin{document}

\maketitle

\textcolor{WildStrawberry}{Problema:}\\
Probar que el elipsoide es un ovaloide.

\textcolor{WildStrawberry}{Solución:}\\
El elipsoide, $\textcolor{blue}{\mathcal{E}}$, es compacto y conexo por ser homeomorfo a $\bb{S}^2$.
$$
\begin{array}{crcl}
f : & \bb{R}^3 & \longrightarrow & \bb{R}^3 \\
& (x,y,z) & \longmapsto & (ax, by, cz) \\
& \bb{S}^2 & \longmapsto & \textcolor{blue}{\mathcal{E}}
\end{array}
$$
$f$ es continua por ser lineal. Calculemos la curvatura de Gauss. Tomamos la siguiente parametrización,
$$
\textcolor{blue}{\bb{X}}(u,v) = (a \sen v \cos u, b \sen v \sen u, c \cos v) \quad v\in (0, \pi) \quad u\in(0, 2\pi)
$$
$$
\begin{array}{cc}
    \textcolor{blue}{\bb{X}}_u = (-a \sen v \sen u, b \sen v \cos u, 0) & \textcolor{blue}{\bb{X}}_v = (a \cos v \cos u, b \cos v \sen u, -c \sen v) \\
    \textcolor{blue}{\bb{X}}_{uu} = (-a \sen v \cos u, -b \sen v \sen u, 0) & \textcolor{blue}{\bb{X}}_{vu} = (-a \cos v \sen u, b \cos v \cos u, 0) \\
    \textcolor{blue}{\bb{X}}_{uv} = (-a \cos v \sen u, b \cos v \cos u, 0) & \textcolor{blue}{\bb{X}}_{vv} = (-a \sen v \cos u, -b \sen v \sen u, -c \cos v)
\end{array}
$$
$$
E = \textcolor{blue}{\bb{X}}_u \textcolor{blue}{\bb{X}}_u = a^2 \sen^2 v \sen^2 u + b^2 \sen^2 v \cos^2 u = \sen^2 v (a^2 \sen^2 u + b^2 \cos^2 u) = \sen^2 v ((a^2-b^2)\sen^2 u + b^2)
$$
$$
F = \textcolor{blue}{\bb{X}}_u \textcolor{blue}{\bb{X}}_v = - a^2 \sen v \sen u \cos v \cos u + b^2 \sen v \cos u \cos v \sen u = (b^2 - a^2) \sen v \sen u \cos v \cos u
$$
$$
G = \textcolor{blue}{\bb{X}}_v \textcolor{blue}{\bb{X}}_v = a^2 \cos^2 v \cos^2 u + b^2 \cos^2 v \sen^2 u + c^2 \sen^2 v = \cos^2 v ((a^2-b^2) \cos^2 u + b^2 - c^2) + c^2
$$
$$
\textcolor{blue}{\bb{X}}_u \land \textcolor{blue}{\bb{X}}_v =
\left|\begin{array}{ccc}
    i & j & k \\
    -a \sen v \sen u & b \sen v \cos u & 0 \\
    a \cos v \cos u & b \cos v \sen u & -c \sen v    
\end{array}\right| =
\left(\begin{array}{c}
-bc \sen^2 v \cos u \\
-ac \sen^2 v \sen u \\
-ab \sen v \sen^2 u \cos v - ab \sen v \cos^2 u \cos v
\end{array}\right)
$$
$$
\textcolor{blue}{\bb{X}}_u \land \textcolor{blue}{\bb{X}}_v =
\left(\begin{array}{c}
-bc \sen^2 v \cos u \\
-ac \sen^2 v \sen u \\
-ab \sen v \cos v
\end{array}\right) \quad
\| \textcolor{blue}{\bb{X}}_u \land \textcolor{blue}{\bb{X}}_v \| = \frac{1}{n} \ge 0
$$
$$
N = (-nbc \sen^2 v \cos u, -nac \sen^2 v \sen u, -nab \sen v \cos v)
$$
$$
e = N \textcolor{blue}{\bb{X}}_{uu} =
nabc \sen^3 v \cos^2 u + nabc \sen^3 v \sen^2 u =
nabc \sen^3 v
$$
$$
f = N \textcolor{blue}{\bb{X}}_{uv} =
nabc \sen^2 v \cos u \cos v \sen u - nabc \sen^2 v \sen u \cos u \cos v =
0
$$
$$
g = N \textcolor{blue}{\bb{X}}_{vv} =
nabc \sen^3 v \cos^2 u + nabc \sen^3 v \sen^2 u + nabc \sen v \cos^2 v =
nabc (\sen^3 v + \sen v \cos^2 v) = 
nabc \sen v
$$
$$
K = \frac{eg-f^2}{EG-F^2} = \frac{nabc \sen^3 v nabc \sen v}{EG-F^2} = \frac{(nabc)^2 \sen^4 v}{EG-F^2}
$$
Veamos que $EG - F^2$ es positivo.
$$
EG = \left(\sen^2 v ((a^2-b^2)\sen^2 u + b^2)\right) \left(\cos^2 v ((a^2-b^2) \cos^2 u + b^2 - c^2) + c^2\right) =
$$
$$
= \sen^2 v \left(
  ((a^2-b^2)\sen^2 u + b^2) (\cos^2 v ((a^2-b^2) \cos^2 u + b^2 - c^2) + c^2)
\right) =
$$
$$
\begin{array}{l}
  = \sen^2 v ( \\
  (a^2-b^2)\sen^2 u \cos^2 v ((a^2-b^2) \cos^2 u + b^2 - c^2) + \\
  (a^2-b^2)\sen^2 u c^2 + \\
  b^2 \cos^2 v ((a^2-b^2) \cos^2 u + b^2 - c^2) + \\
  b^2 c^2 ) =
\end{array}
\begin{array}{l}
  = \sen^2 v ( \\
  (a^2-b^2)\sen^2 u \cos^2 v ((a^2-b^2) \cos^2 u + b^2 - c^2) + \\
  c^2(a^2-b^2)\sen^2 u + \\
  (a^2-b^2) b^2 \cos^2 v \cos^2 u + b^4 \cos^2 v - b^2c^2 \cos^2 v + \\
  b^2 c^2 ) =
\end{array}
$$
$$
\begin{array}{l}
  = \sen^2 v ( \\
  (a^2-b^2)\sen^2 u \cos^2 v ((a^2-b^2) \cos^2 u + b^2 - c^2) + \\
  c^2(a^2-b^2)\sen^2 u + \\
  (a^2-b^2) b^2 \cos^2 v \cos^2 u + b^4 \cos^2 v - b^2c^2 \cos^2 v + \\
  b^2 c^2 )
\end{array}
$$
$$
F^2 = (b^2 - a^2)^2 \sen^2 v \sen^2 u \cos^2 v \cos^2 u
$$
$$
\begin{array}{l}
  EG-F^2
  = \sen^2 v ( \\
  + (a^2-b^2)\sen^2 u \cos^2 v ((a^2-b^2) \cos^2 u + b^2 - c^2) \\
  + c^2(a^2-b^2)\sen^2 u \\
  + (a^2-b^2) b^2 \cos^2 v \cos^2 u + b^4 \cos^2 v - b^2c^2 \cos^2 v \\
  + b^2 c^2 \\
  - (b^2 - a^2)^2\sen^2 u \cos^2 v \cos^2 u) =
\end{array}
$$
Es suficiente probar que $\frac{EG-F^2}{\sen^2 v}$ es positivo.
$$
\begin{array}{l}
  (a^2-b^2)\sen^2 u \cos^2 v ((a^2-b^2) \cos^2 u + b^2 - c^2) \\
  + c^2(a^2-b^2)\sen^2 u \\
  + (a^2-b^2) b^2 \cos^2 v \cos^2 u + b^4 \cos^2 v - b^2c^2 \cos^2 v \\
  + b^2 c^2 \\
  - (b^2 - a^2)^2\sen^2 u \cos^2 v \cos^2 u =
\end{array}
\begin{array}{l}
  = (a^2-b^2)\sen^2 u \cos^2 v (a^2-b^2) \cos^2 u \\
  + (a^2-b^2)\sen^2 u \cos^2 v (b^2 - c^2) \\
  + c^2(a^2-b^2)\sen^2 u \\
  + (a^2-b^2) b^2 \cos^2 v \cos^2 u + b^4 \cos^2 v - b^2c^2 \cos^2 v \\
  + b^2 c^2 \\
  - (b^2 - a^2)^2\sen^2 u \cos^2 v \cos^2 u =
\end{array}
$$
$$
\begin{array}{l}
  = (a^2-b^2)^2 \sen^2 u \cos^2 v \cos^2 u \\
  + (a^2-b^2)(b^2 - c^2) \sen^2 u \cos^2 v \\
  + c^2(a^2-b^2)\sen^2 u \\
  + (a^2-b^2) b^2 \cos^2 v \cos^2 u \\
  + b^4 \cos^2 v \\
  + b^2 c^2 \\
  - (b^2 - a^2)^2\sen^2 u \cos^2 v \cos^2 u \\
  - b^2c^2 \cos^2 v =
\end{array}
\begin{array}{l}
  = (a^2-b^2)(b^2 - c^2) \sen^2 u \cos^2 v \\
  + c^2(a^2-b^2)\sen^2 u \\
  + (a^2-b^2) b^2 \cos^2 v \cos^2 u \\
  + b^2 c^2 \\
  + b^4 \cos^2 v \\
  - b^2c^2 \cos^2 v =
\end{array}
$$
$$
\begin{array}{l}
  = (a^2b^2 - a^2c^2 - b^4 + b^2c^2) \sen^2 u \cos^2 v \\
  + (a^2c^2 - b^2c^2)\sen^2 u \\
  + (a^2b^2-b^4) \cos^2 v \cos^2 u \\
  + b^2 c^2 \\
  + b^4 \cos^2 v \\
  - b^2c^2 \cos^2 v =
\end{array}
\begin{array}{l}
  = a^2b^2 \sen^2 u \cos^2 v \\
  - a^2c^2 \sen^2 u \cos^2 v \\
  - b^4 \sen^2 u \cos^2 v \\
  + b^2c^2 \sen^2 u \cos^2 v \\
  + a^2c^2 \sen^2 u \\
  - b^2c^2 \sen^2 u \\
  + a^2b^2 \cos^2 v \cos^2 u \\
  - b^4 \cos^2 v \cos^2 u \\
  + b^2c^2 \\
  + b^4 \cos^2 v \\
  - b^2c^2 \cos^2 v =
\end{array}
$$
$$
\begin{array}{l}
  = a^2b^2 (\sen^2 u \cos^2 v + \cos^2 v \cos^2 u) \\
  + a^2c^2 (\sen^2 u - \sen^2 u \cos^2 v) \\
  + b^2c^2 (1 + \sen^2 u \cos^2 v - \sen^2 u - \cos^2 v)\\
  + b^4 (\cos^2 v - \sen^2 u \cos^2 v - \cos^2 v \cos^2 u) =
\end{array}
\begin{array}{l}
  = a^2b^2 (\sen^2 u \cos^2 v + \cos^2 v \cos^2 u) \\
  + a^2c^2 \sen^2 u (1 - \cos^2 v) \\
  + b^2c^2 (1 + \sen^2 u \cos^2 v - \sen^2 u - (1 - \sen^2 v))\\
  + b^4 \cos^2 v (1 - \sen^2 u - \cos^2 u) =
\end{array}
$$
$$
\begin{array}{l}
  = a^2b^2 (\sen^2 u \cos^2 v + \cos^2 v \cos^2 u) \\
  + a^2c^2 \sen^2 u \sen^2 v \\
  + b^2c^2 (1 + \sen^2 u \cos^2 v - \sen^2 u - 1 + \sen^2 v) =
\end{array}
\begin{array}{l}
  = a^2b^2 (\sen^2 u \cos^2 v + \cos^2 v \cos^2 u) \\
  + a^2c^2 \sen^2 u \sen^2 v \\
  + b^2c^2 (\sen^2 u \cos^2 v - \sen^2 u + \sen^2 v) =
\end{array}
$$
$$
\begin{array}{l}
  = a^2b^2 (\sen^2 u \cos^2 v + \cos^2 v \cos^2 u) \\
  + a^2c^2 \sen^2 u \sen^2 v \\
  + b^2c^2 (\sen^2 u (\cos^2 v - 1) + \sen^2 v) =
\end{array}
\begin{array}{l}
  = a^2b^2 (\sen^2 u \cos^2 v + \cos^2 v \cos^2 u) \\
  + a^2c^2 \sen^2 u \sen^2 v \\
  + b^2c^2 \sen^2 v (1 - \sen^2 u) =
\end{array}
$$
$$
\begin{array}{l}
  = a^2b^2 (\sen^2 u \cos^2 v + \cos^2 v \cos^2 u) \\
  + a^2c^2 \sen^2 u \sen^2 v \\
  + b^2c^2 \sen^2 v \cos^2 u > 0
\end{array}
$$
Tenemos que $K>0$. Los cálculos funcionan cuando $\sen v \ne 0$ que nunca ocurre con $v\in (0, \pi)$. El resto de cartas necesarias para cubrir $\textcolor{blue}{\mathcal{E}}$ son similares a $\textcolor{blue}{\bb{X}}$ y los cálculos son casi idénticos.
\end{document}