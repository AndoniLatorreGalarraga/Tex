\documentclass{article}
\usepackage[utf8]{inputenc}
\usepackage{graphicx}
\usepackage[spanish]{babel}
\usepackage{amssymb,amsmath,geometry,multicol,spalign,hyperref}
\usepackage[usenames,dvipsnames]{xcolor}
\usepackage{tikz,mathtools}
\usepackage{pgfplots}
\pgfplotsset{every axis/.append style={
                    axis x line=middle,    % put the x axis in the middle
                    axis y line=middle,    % put the y axis in the middle
                    axis line style={<->,color=blue}, % arrows on the axis
                    xlabel={$x$},          % default put x on x-axis
                    ylabel={$y$},          % default put y on y-axis
            }}
\usepackage{etoolbox} %titulo
\makeatletter %titulo
\patchcmd{\@maketitle}{\vskip 2em}{\vspace*{-3cm}}{}{} %titulo
\makeatother %titulo
\usepackage{vmargin}
\setpapersize{A4}
\setmargins{2.5cm}       % margen izquierdo
{1.5cm}                        % margen superior
{16.5cm}                      % anchura del texto
{23.42cm}                    % altura del texto
{10pt}                           % altura de los encabezados
{1cm}                           % espacio entre el texto y los encabezados
{0pt}                             % altura del pie de página
{2cm}                           % espacio entre el texto y el pie de página
\title{Cifrado Payfair}
\author{Andoni Latorre Galarraga y Mariana Zaballa Bernabé}
\date{}
\newcommand{\bb}[1]{\mathbb{#1}}
\newcommand{\R}{\bb{R}}
\newcommand{\nota}[3][2ex]{
    \underset{\mathclap{
        \begin{tikzpicture}
          \draw[->] (0, 0) to ++(0,#1);
          \node[below] at (0,0) {#3};
        \end{tikzpicture}}}{#2}
}
\begin{document}

\maketitle

\section*{Cifrado}

Utlizamos un bucle \textcolor{blue}{While Wend} para recorrer el mensaje. La variable $k$ es la posición del mensaje que vamos a leer. Dependiendo de si tenemos que introducir un caracter sin sentido $(\odot)$ avanzamos $k$ por 1 o por 2.
$$
\begin{array}{ccc}
    a_1 & a_2 &  \\
    \text{letra} & \text{letra} & k \leftarrow k + 2 \\
    \text{letra} & \odot & k \leftarrow k + 1 \\
\end{array}
$$
En la última posición nos aseguramos de aumentar el valor de $k$ para que no se cumpla la condición $k\le l$.
\\
Cuando ambas letras están en la misma fila/columna queremos sumar 1 módulo 5. Pero en vez de tener el resultado en $\{0,1,2,3,4\}$ lo queremos en $\{1,2,3,4,5\}$. Hemos logrado esto de la siguiente manera
$$
( x  \: \text{\textcolor{blue}{Mod}} \: 5 ) +1
$$
\\
Para cuando no comparten ni fila ni columna hemos hecho una observación,
$$
\begin{array}{c|ccc}
     & j_1 & j_2 \\ \hline
    i_1 & a_1 &  \\
    i_2 &  & a_2 
\end{array}\quad , \quad
\begin{array}{c|ccc}
    & j_2 & j_1 \\ \hline
   i_2 & a_2 &  \\
   i_1 &  & a_1 
\end{array}
\Rightarrow sig(j_1-j_2) = sig(i_1-1_2) \Rightarrow (j_1-j_2)(i_1-i_2) > 0
$$
$$
\begin{array}{c|ccc}
     & j_1 & j_2 \\ \hline
    i_1 &  & a_1 \\
    i_2 & a_2 & 
\end{array}\quad , \quad
\begin{array}{c|ccc}
    & j_1 & j_2 \\ \hline
    i_2 &  & a_2 \\
    i_1 & a_1 & 
\end{array}
\Rightarrow sig(j_1-j_2) \ne sig(i_1-1_2) \Rightarrow (j_1-j_2)(i_1-i_2) < 0
$$
Por eso utlizamos $(j_1-j_2)(i_1-i_2) > 0$ como condición en el \textcolor{blue}{If}. Calculamos $b_1, b_2$ como corresponde en cada caso
$$
\begin{array}{c|ccc}
     & j_1 & j_2 \\ \hline
    i_1 & a_1 & b_1 \\
    i_2 & b_2 & a_2 
\end{array}\quad , \quad
\begin{array}{c|ccc}
    & j_2 & j_1 \\ \hline
   i_2 & a_2 & b_2 \\
   i_1 & b_1 & a_1 
\end{array} 
\Rightarrow b_1 \text{ en } (i_1, j_2) \quad b_2 \text{ en } (i_2, j_1)
$$
$$
\begin{array}{c|ccc}
     & j_1 & j_2 \\ \hline
    i_1 & b_2 & a_1 \\
    i_2 & a_2 & b_1
\end{array}\quad , \quad b_1 
\begin{array}{c|ccc}
    & j_1 & j_2 \\ \hline
    i_2 & b_1 & a_2 \\
    i_1 & a_1 & b_2
\end{array}
\Rightarrow b_1 \text{ en } (i_2, j_1) \quad b_2 \text{ en } (i_1, j_2)
$$

\section*{Descifrado}
El código es basicamente el mismo con el cambio de restar 1 módulo 5 en vez de sumarlo y el cambio correspondiente al caso de fila y columna diferentes.
$$
\begin{array}{c|c}
    x & ((x+3) \: \textcolor{blue}{\text{Mod}} \: 5) + 1 \\ \hline
    1 & 5 \\
    2 & 1 \\
    3 & 2 \\
    4 & 3 \\
    5 & 4 \\
\end{array}
$$
La idea para esta fórmula ha venido de $-1\equiv_5 4 = 3 + 1$.
\end{document}