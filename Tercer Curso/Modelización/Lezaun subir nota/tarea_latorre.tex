\documentclass{article}
\usepackage[utf8]{inputenc}
\usepackage{graphicx}
\usepackage[spanish]{babel}
\usepackage{amssymb,amsmath,geometry,multicol,spalign,hyperref,changepage}
\usepackage[usenames,dvipsnames]{xcolor}
\usepackage{tikz,mathtools}
\usepackage{pgfplots}
\pgfplotsset{every axis/.append style={
                    axis x line=middle,    % put the x axis in the middle
                    axis y line=middle,    % put the y axis in the middle
                    axis line style={<->,color=blue}, % arrows on the axis
                    xlabel={$x$},          % default put x on x-axis
                    ylabel={$y$},          % default put y on y-axis
            }}
\usepackage{etoolbox} %titulo
\makeatletter %titulo
\patchcmd{\@maketitle}{\vskip 2em}{\vspace*{-3cm}}{}{} %titulo
\makeatother %titulo
\usepackage{vmargin}
\setpapersize{A4}
\setmargins{2.5cm}       % margen izquierdo
{1.5cm}                        % margen superior
{16.5cm}                      % anchura del texto
{23.42cm}                    % altura del texto
{10pt}                           % altura de los encabezados
{1cm}                           % espacio entre el texto y los encabezados
{0pt}                             % altura del pie de página
{2cm}                           % espacio entre el texto y el pie de página
\title{Tarea Modelización}
\author{Andoni Latorre Galarraga}
\date{}
\newcommand{\bb}[1]{\mathbb{#1}}
\newcommand{\R}{\bb{R}}
\newcommand{\nota}[3][2ex]{
    \underset{\mathclap{
        \begin{tikzpicture}
          \draw[->] (0, 0) to ++(0,#1);
          \node[below] at (0,0) {#3};
        \end{tikzpicture}}}{#2}
}
% Default fixed font does not support bold face
\DeclareFixedFont{\ttb}{T1}{txtt}{bx}{n}{12} % for bold
\DeclareFixedFont{\ttm}{T1}{txtt}{m}{n}{12}  % for normal

% Custom colors
\usepackage{color}
\definecolor{deepblue}{rgb}{0,0,0.5}
\definecolor{deepred}{rgb}{0.6,0,0}
\definecolor{deepgreen}{rgb}{0,0.5,0}

\usepackage{listings}

% Python style for highlighting
\newcommand\pythonstyle{\lstset{
language=Python,
basicstyle=\ttm,
morekeywords={self},              % Add keywords here
keywordstyle=\ttb\color{deepblue},
emph={MyClass,__init__},          % Custom highlighting
emphstyle=\ttb\color{deepred},    % Custom highlighting style
stringstyle=\color{deepgreen},
frame=tb,                         % Any extra options here
showstringspaces=false
}}


% Python environment
\lstnewenvironment{python}[1][]
{
\pythonstyle
\lstset{#1}
}
{}

% Python for external files
\newcommand\pythonexternal[2][]{{
\pythonstyle
\lstinputlisting[#1]{#2}}}

% Python for inline
\newcommand\pythoninline[1]{{\pythonstyle\lstinline!#1!}}
\begin{document}

\maketitle

\noindent\textbf{1.} Supongamos que tenemos identificado un producto por doce dígitos y queremos
añadir a su derecha un dígito de control. Procederemos como sigue.\\
\begin{adjustwidth}{1cm}{}
\textit{a)} Por un lado, las cifras que ocupan un lugar impar, empezando a contar por la
izquierda, se multiplican por tres y se retiene la última cifra. Hecho esto, se suman
los números obtenidos.
\\\\
\textit{b)} Por otro, las cifras que ocupa un lugar par, empezando a contar por la izquierda,
se multiplican por siete y se retiene la última cifra. Hecho esto, se suman los
números obtenidos.
\\\\
\textit{c)} Se suma el resultado del apartado \textit{a)} y el del apartado \textit{b)}. El dígito de control es
la última cifra de este número.\\
\end{adjustwidth}
Demostrar que este algoritmo detecta cualquier error de un único dígito.
\\\\
¿Este algoritmo detecta el intercambio de dos cifras consecutivas?
\\\\
En qué casos sí y en cuáles no.
\\\\\\
Sea $\textcolor{blue}{a_1a_2a_3a_4a_5a_6a_7a_8a_9a_{10}a_{11}a_{12}}$ el identificador del producto. Primero tomamos las cifras de posición impar $\textcolor{blue}{a_1}, \textcolor{blue}{a_3}, \textcolor{blue}{a_5}, \textcolor{blue}{a_7}, \textcolor{blue}{a_9}, \textcolor{blue}{a_{11}}$. Las multiplicamos por 3: $3\textcolor{blue}{a_1}, 3\textcolor{blue}{a_3}, 3\textcolor{blue}{a_5}, 3\textcolor{blue}{a_7}, 3\textcolor{blue}{a_9}, 3\textcolor{blue}{a_{11}}$ y retenemos la última cifra $\text{mód}(3\textcolor{blue}{a_1},10), \text{mód}(3\textcolor{blue}{a_3},10), \text{mód}(3\textcolor{blue}{a_5},10), \text{mód}(3\textcolor{blue}{a_7},10), \text{mód}(3\textcolor{blue}{a_9},10), \text{mód}(3\textcolor{blue}{a_{11}},10)$. Finálmente sumamos.
$$
\textcolor{red}{a} = \text{mód}(3\textcolor{blue}{a_1},10) + \text{mód}(3\textcolor{blue}{a_3},10) + \text{mód}(3\textcolor{blue}{a_5},10) + \text{mód}(3\textcolor{blue}{a_7},10) + \text{mód}(3\textcolor{blue}{a_9},10) + \text{mód}(3\textcolor{blue}{a_{11}},10)
$$
Procedemos con las cifras pares.
$$
\textcolor{red}{b} = \text{mód}(7\textcolor{blue}{a_2},10) + \text{mód}(7\textcolor{blue}{a_4},10) + \text{mód}(7\textcolor{blue}{a_6},10) + \text{mód}(7\textcolor{blue}{a_8},10) + \text{mód}(7\textcolor{blue}{a_{10}},10) + \text{mód}(7\textcolor{blue}{a_{12}},10)
$$
Sumamos $\textcolor{red}{a}$ y $\textcolor{red}{b}$ y nos quedamos con la última cifra para obtener el dígito de control $\textcolor{orange}{c}$.
$$
\textcolor{orange}{c} = \text{mód}(\textcolor{red}{a} + \textcolor{red}{b}, 10) =
$$
$$
= \text{mód}(\sum_{i=1}^6 \text{mód}(3\textcolor{blue}{a_{2i-1}},10) + \sum_{i=1}^6 \text{mód}(7\textcolor{blue}{a_{2i}},10), 10) =
$$
$$
= \text{mód}(\sum_{i=1}^6 3\textcolor{blue}{a_{2i-1}} + \sum_{i=1}^6 7\textcolor{blue}{a_{2i}}, 10) = \text{mód}(\sum_{i=1}^6 3\textcolor{blue}{a_{2i-1}} + 7\textcolor{blue}{a_{2i}}, 10)
$$
Supongamos ahora que se produce un error en el dígito $\textcolor{blue}{a_k}$ de tal manera que en vez de $\textcolor{blue}{a_k}$ tenemos $\textcolor{ForestGreen}{b_k} \ne \textcolor{blue}{a_k}$.\\
\begin{adjustwidth}{1cm}{}
    Caso 1: $k$ es par. Supongamos que el dígito de control no ha cambiado
    $$
    \text{mód}(\sum_{i=1}^6 3\textcolor{blue}{a_{2i-1}} + 7\textcolor{blue}{a_{2i}}, 10) = \text{mód}(7\textcolor{ForestGreen}{b_k}-7 \textcolor{blue}{a_k} + \sum_{i=1}^6 3\textcolor{blue}{a_{2i-1}} + 7\textcolor{blue}{a_{2i}}, 10)
    $$
    $$
    \Rightarrow
    \sum_{i=1}^6 3\textcolor{blue}{a_{2i-1}} + 7\textcolor{blue}{a_{2i}} \equiv_{10} 7\textcolor{ForestGreen}{b_k}-7 \textcolor{blue}{a_k} + \sum_{i=1}^6 3\textcolor{blue}{a_{2i-1}} + 7\textcolor{blue}{a_{2i}}
    $$
    $$ \Rightarrow
    0 \equiv_{10} 7\textcolor{ForestGreen}{b_k}-7 \textcolor{blue}{a_k}
    \rightarrow
    7\textcolor{ForestGreen}{b_k} \equiv_{10} 7 \textcolor{blue}{a_k}
    \Rightarrow
    3\cdot 7\textcolor{ForestGreen}{b_k} \equiv_{10} 3 \cdot 7 \textcolor{blue}{a_k}
    \Rightarrow
    21\textcolor{ForestGreen}{b_k} \equiv_{10} 21 \textcolor{blue}{a_k}
    \Rightarrow
    1\textcolor{ForestGreen}{b_k} \equiv_{10} 1 \textcolor{blue}{a_k}
    $$
    Pero $\textcolor{ForestGreen}{b_k} \ne \textcolor{blue}{a_k}$ y $0 \le \textcolor{ForestGreen}{b_k} ,\textcolor{blue}{a_k} \le 9$, lo cual es contradictorio con $\textcolor{ForestGreen}{b_k} \equiv_{10} \textcolor{blue}{a_k}$.
    \\\\
    Caso 2: $k$ es impar. Supongamos que el dígito de control no ha cambiado
    $$
    \text{mód}(\sum_{i=1}^6 3\textcolor{blue}{a_{2i-1}} + 7\textcolor{blue}{a_{2i}}, 10) = \text{mód}(3\textcolor{ForestGreen}{b_k}-3 \textcolor{blue}{a_k} + \sum_{i=1}^6 3\textcolor{blue}{a_{2i-1}} + 7\textcolor{blue}{a_{2i}}, 10)
    $$
    $$
    \Rightarrow
    \sum_{i=1}^6 3\textcolor{blue}{a_{2i-1}} + 7\textcolor{blue}{a_{2i}} \equiv_{10} 3\textcolor{ForestGreen}{b_k}-3 \textcolor{blue}{a_k} + \sum_{i=1}^6 3\textcolor{blue}{a_{2i-1}} + 7\textcolor{blue}{a_{2i}}
    $$
    $$ \Rightarrow
    0 \equiv_{10} 3\textcolor{ForestGreen}{b_k}-3 \textcolor{blue}{a_k}
    \rightarrow
    3\textcolor{ForestGreen}{b_k} \equiv_{10} 3 \textcolor{blue}{a_k}
    \Rightarrow
    7\cdot 3\textcolor{ForestGreen}{b_k} \equiv_{10} 7 \cdot 3 \textcolor{blue}{a_k}
    \Rightarrow
    21\textcolor{ForestGreen}{b_k} \equiv_{10} 21 \textcolor{blue}{a_k}
    \Rightarrow
    1\textcolor{ForestGreen}{b_k} \equiv_{10} 1 \textcolor{blue}{a_k}
    $$
    Pero $\textcolor{ForestGreen}{b_k} \ne \textcolor{blue}{a_k}$ y $0 \le \textcolor{ForestGreen}{b_k} ,\textcolor{blue}{a_k} \le 9$, lo cual es contradictorio con $\textcolor{ForestGreen}{b_k} \equiv_{10} \textcolor{blue}{a_k}$.\\
\end{adjustwidth}
Veamos que ocurre si intercambiamos dos cifras consecutivas $\textcolor{blue}{a_{2k}}$ y $\textcolor{blue}{a_{2k-1}}$ con $k \in \{1,2,3,4,5,6\}$. Veamos que tiene que ocurrir para que se mantenga $\textcolor{orange}{c}$.
$$
\text{mód}(\sum_{i=1}^6 3\textcolor{blue}{a_{2i-1}} + 7\textcolor{blue}{a_{2i}}, 10) = \text{mód}(3 \textcolor{blue}{a_{2k}} + 7 \textcolor{blue}{a_{2k-1}} - 7 \textcolor{blue}{a_{2k}} - 3 \textcolor{blue}{a_{2k-1}} +\sum_{i=1}^6 3\textcolor{blue}{a_{2i-1}} + 7\textcolor{blue}{a_{2i}}, 10)
$$
$$
\Leftrightarrow
0 \equiv_{10} 3 \textcolor{blue}{a_{2k}} + 7 \textcolor{blue}{a_{2k-1}} - 7 \textcolor{blue}{a_{2k}} - 3 \textcolor{blue}{a_{2k-1}}
\Leftrightarrow
7 \textcolor{blue}{a_{2k}} - 3 \textcolor{blue}{a_{2k}} \equiv_{10} 7 \textcolor{blue}{a_{2k-1}} - 3 \textcolor{blue}{a_{2k-1}}
\Leftrightarrow
4 \textcolor{blue}{a_{2k}} \equiv_{10} 4 \textcolor{blue}{a_{2k-1}}
$$
En este punto podemos concluir que en ocasiones no va a detectar el cambio de cifras consecutivas ya que 4 y 10 no son coprimos. Estudiemos la función $\text{mód}(4 x, 10)$ para saber en que casos detecta y en cuales no.
$$
\begin{array}{c|c}
    x & \text{mód}(4 x, 10) \\ \hline
    0 & 0 \\
    1 & 4 \\
    2 & 8 \\
    3 & 2 \\
    4 & 6 \\
    5 & 0 \\
    6 & 4 \\
    7 & 8 \\
    8 & 2 \\
    9 & 6 
\end{array}
$$
No detecta el intercambio de dos dígitos consecutivos cuando estos distan en 5 es decir:
\begin{center}
cambiar 0 y 5,\\
cambiar 1 y 6,\\
cambiar 2 y 7,\\
cambiar 3 y 8,\\
cambiar 4 y 9.
\end{center}
No es casualidad que sean aquellos que distan por 5.
$$
4x \equiv_{10} 4x-0 \equiv_{10} 4x-20 \equiv_{10} 4(x-5) 
$$
Veamos ahora que ocurre si cambiamos $\textcolor{orange}{c}$ y $\textcolor{blue}{a_{12}}$. Supongamos que no se detecta el cambio
$$
\text{mód}(\sum_{i=1}^6 3\textcolor{blue}{a_{2i-1}} + 7\textcolor{blue}{a_{2i}}, 10) = \text{mód}(7 \textcolor{orange}{c} - 7\textcolor{blue}{a_{12}} + \sum_{i=1}^6 3\textcolor{blue}{a_{2i-1}} + 7\textcolor{blue}{a_{2i}}, 10)
$$
$$
\Leftrightarrow
\sum_{i=1}^6 3\textcolor{blue}{a_{2i-1}} + 7\textcolor{blue}{a_{2i}} \equiv_{10} 7 \textcolor{orange}{c} - 7\textcolor{blue}{a_{12}} + \sum_{i=1}^6 3\textcolor{blue}{a_{2i-1}} + 7\textcolor{blue}{a_{2i}}
$$
$$
\Leftrightarrow
0 \equiv_{10} 7 \textcolor{orange}{c} - 7\textcolor{blue}{a_{12}}
\Leftrightarrow 
7\textcolor{blue}{a_{12}} \equiv_{10} 7 \textcolor{orange}{c}
\Leftrightarrow
3 \cdot 7\textcolor{blue}{a_{12}} \equiv_{10} 3 \cdot 7 \textcolor{orange}{c}
\Leftrightarrow
\textcolor{blue}{a_{12}} \equiv_{10} \textcolor{orange}{c}
\nota{\Leftrightarrow}{$\textcolor{blue}{a_{12}},\textcolor{orange}{c}\in \{0,1,2,3,4,5,6,7,8,9\}$}
\textcolor{blue}{a_{12}} = \textcolor{orange}{c}
$$
Es decir, la única manera de de que no se detecte el intercambio es que $\textcolor{blue}{a_{12}} = \textcolor{orange}{c}$, que no es realmente un intercambio. El algoritmo detecta cabiar la última cifra del identificador del producto con el dígito de control.
\\\\
\textit{Bonus}: Estudiamos la "simetría" del dígito de control.
$$
\begin{array}{ccccccccccccc}
    \textcolor{blue}{a_1} & \textcolor{blue}{a_2} & \textcolor{blue}{a_3} & \textcolor{blue}{a_4} & \textcolor{blue}{a_5} & \textcolor{blue}{a_6} & \textcolor{blue}{a_7} & \textcolor{blue}{a_8} & \textcolor{blue}{a_9} & \textcolor{blue}{a_{10}} & \textcolor{blue}{a_{11}} & \textcolor{blue}{a_{12}} & \textcolor{orange}{c} \\
    \downarrow & \downarrow & \downarrow & \downarrow & \downarrow & \downarrow & \downarrow & \downarrow & \downarrow & \downarrow & \downarrow & \downarrow & \downarrow \\
    \textcolor{orange}{\tilde{c}} & \textcolor{ForestGreen}{b_{12}} & \textcolor{ForestGreen}{b_{11}} & \textcolor{ForestGreen}{b_{10}} & \textcolor{ForestGreen}{b_9} & \textcolor{ForestGreen}{b_8} & \textcolor{ForestGreen}{b_7} & \textcolor{ForestGreen}{b_6} & \textcolor{ForestGreen}{b_5} & \textcolor{ForestGreen}{b_{4}} & \textcolor{ForestGreen}{b_{3}} & \textcolor{ForestGreen}{b_{2}} & \textcolor{ForestGreen}{b_1} \\
\end{array}
$$
$$
\textcolor{orange}{\tilde{c}} \equiv_{10} \sum_{i=1}^6 3\textcolor{ForestGreen}{b_{2i-1}} + 7\textcolor{ForestGreen}{b_{2i}} \equiv_{10}
$$
$$
\equiv_{10} 3 \textcolor{orange}{c} + 7\textcolor{blue}{a_{12}} + 3 \textcolor{blue}{a_{11}} + 7\textcolor{blue}{a_{10}} + 3 \textcolor{blue}{a_9} + 7\textcolor{blue}{a_8} + 3 \textcolor{blue}{a_7} + 7\textcolor{blue}{a_6} + 3 \textcolor{blue}{a_5} + 7\textcolor{blue}{a_4} + 3 \textcolor{blue}{a_3} + 7 \textcolor{blue}{a_2}
$$
$$
\equiv_{10} 3 \textcolor{orange}{c} + 7\textcolor{blue}{a_{12}} + 3 \textcolor{blue}{a_{11}} + 7\textcolor{blue}{a_{10}} + 3 \textcolor{blue}{a_9} + 7\textcolor{blue}{a_8} + 3 \textcolor{blue}{a_7} + 7\textcolor{blue}{a_6} + 3 \textcolor{blue}{a_5} + 7\textcolor{blue}{a_4} + 3 \textcolor{blue}{a_3} + 7 \textcolor{blue}{a_2} + 3 \textcolor{blue}{a_1} - 3 \textcolor{blue}{a_1} \equiv_{10}
$$
$$
\equiv_{10} 3 \textcolor{orange}{c} + \textcolor{orange}{c} - 3 \textcolor{blue}{a_1} \equiv_{10} 4 \textcolor{orange}{c} - 3 \textcolor{blue}{a_1} \equiv_{10}
$$
$$
\textcolor{orange}{\tilde{c}} \equiv_{10} 4 \textcolor{orange}{c} - 3 \textcolor{blue}{a_1}
\Rightarrow
\textcolor{blue}{a_1} \equiv_{10} 4 \textcolor{orange}{c} - 3 \textcolor{blue}{a_1}
\Rightarrow
4 \textcolor{blue}{a_1} \equiv_{10} 4 \textcolor{orange}{c}
$$
Que no es necesariamente cierto $000000000003\textcolor{orange}{1}$ pero $130000000000\textcolor{orange}{4}$. Los casos en los que el código es simétrico son similares a los del intercambio de digitos consecutivos, por ejemplo: $700000000003\textcolor{orange}{2}$ y $230000000000\textcolor{orange}{7}$ ya que $7-2=5$.
\begin{python}
>>> def f(s):
    	ss = 0
    	for i in range(0,6):
	    	ss += 7*int(s[2*i+1]) + 3*int(s[2*i])
	    return int(ss) % 10

>>> f('000000000003')
1
>>> f('130000000000')
4
>>> f('700000000003')
2
>>> f('230000000000')
7
\end{python}
\textbf{2.} Se solicita a cuatro pintores, Ander, Beñat, Carlos y David, que presenten a concurso un cuadro y el mejor lo comprará el ayuntamiento. Se acuerda que sean ellos mismos quienes valoren las obras. Entre ellos acuerdan que cada uno se valore así mismo y a otros dos, pero que todos se den así mismo un punto. Una vez hechas las puntuaciones, la valoración final de cada obra es proporcional a la suma de las puntuaciones obtenidas, cada una de ellas ponderada por la valoración que tendrá quien la haya otorgado.
\\\\
El resultado de las votaciones de cada uno a otros dos es el siguiente.
\begin{adjustwidth}{1cm}{}
Ander da 4 puntos a Beñat y 6 a David.\\
Beñat da 6 puntos a Ander y 8 a Carlos.\\
Carlos da 7 puntos a Beñat y 7 a David.\\
David da 5 a Ander y 3 a Carlos.\\
\end{adjustwidth}
¿Cuál es el resultado final?
\\\\
NOTA. Llegar al resultado de dos formas. Presentar las cuentas a mano, no se valorará si
solo presentan los resultados, tanto los finales como los intermedios, aunque sean
correctos.
\\\\
Sean $a,b,c,d$ las puntuaciones de Ander, Beñat, Carlos y David respectivamente. Tenemos el siguente sistema de ecuaciones:
$$
\left\{\begin{array}{ccccccccc}
    a & = & k(1a & + & 6b & + & 0c & + & 5d)\\
    b & = & k(4a & + & 1b & + & 7c & + & 0d)\\
    c & = & k(0a & + & 8b & + & 1c & + & 3d)\\
    d & = & k(6a & + & 0b & + & 7c & + & 1d)\\
\end{array}\right.
\Leftrightarrow
1/k
\left(\begin{array}{ccccccccc}
    a\\
    b\\
    c\\
    d\\
\end{array}\right)
=
\left(\begin{array}{ccccccccc}
    1 & 6 & 0& 5\\
    4 & 1 & 7& 0\\
    0 & 8 & 1& 3\\
    6 & 0 & 7& 1\\
\end{array}\right)
\left(\begin{array}{ccccccccc}
    a\\
    b\\
    c\\
    d\\
\end{array}\right)
$$
Es decir $1/k$ es un valor propio de la matriz cuadrada. Calculamos su polinomi característico
$$
\left|\begin{array}{ccccccccc}
    1-x & 6 & 0& 5\\
    4 & 1-x & 7& 0\\
    0 & 8 & 1-x& 3\\
    6 & 0 & 7& 1-x\\
\end{array}\right| =
$$
$$
= (1-x)
\left|\begin{array}{ccccccccc}
    1-x & 7& 0\\
    8 & 1-x& 3\\
    0 & 7& 1-x\\
\end{array}\right|
- 6
\left|\begin{array}{ccccccccc}
    4 & 7 & 0\\
    0 & 1-x & 3\\
    6 & 7 & 1-x\\
\end{array}\right|
- 5
\left|\begin{array}{ccccccccc}
    4 & 1-x & 7\\
    0 & 8 & 1-x\\
    6 & 0 & 7\\
\end{array}\right| =
$$
$$
= (1-x)\left(
(1-x)
\left|\begin{array}{ccccccccc}
    1-x& 3\\
    7& 1-x\\
\end{array}\right|
-
7
\left|\begin{array}{ccccccccc}
    8 & 3\\
    0 & 1-x\\
\end{array}\right|\right)
$$
$$
- 6 \left(
4
\left|\begin{array}{ccccccccc}
    1-x & 3\\
    7 & 1-x\\
\end{array}\right|
- 7
\left|\begin{array}{ccccccccc}
    0 & 3\\
    6 & 1-x\\
\end{array}\right|
\right)
$$
$$
- 5
\left(
4
\left|\begin{array}{ccccccccc}
    8 & 1-x\\
    0 & 7\\
\end{array}\right|
+ 6
\left|\begin{array}{ccccccccc}
    1-x & 7\\
    8 & 1-x\\
\end{array}\right|
\right) =
$$
$$
= (1-x)\left(
(1-x)
((1-x)^2 - 21)
-
7
(8(1-x))\right)
- 6 \left(
4
((1-x)^2-21)
- 7
(-18)
\right)
- 5
\left(
4
(56)
+ 6
((1-x)^2-56)
\right)
$$
Haciendo el cambio de variable $t=1-x$
$$
= t(
t
(t^2 - 21)
-
7
(8t))
- 6 (
4
(t^2-21)
- 7
(-18)
)
- 5
(
4
(56)
+ 6
(t^2-56)
)=
t^4 - 131 t^2 + 308
$$
$$
t^2 = \frac{131\pm\sqrt{15929}}{2}
\Rightarrow
x = 1 \pm \sqrt{\frac{131\pm\sqrt{15929}}{2}}
$$
Coomo
$$
\sqrt{\frac{131+\sqrt{15929}}{2}} = 11.34 \quad \sqrt{\frac{131+\sqrt{15929}}{2}} = 1.54
$$
Los valores propios positivos son
$$
\sqrt{\frac{131+\sqrt{15929}}{2}} = 12.34 = \frac{1}{k_1} \quad 1+\sqrt{\frac{131+\sqrt{15929}}{2}} = 2.54 = \frac{1}{k_2}
$$
Con $k_1$:
$$
\left\{\begin{array}{ccccccccc}
    12.34a & = & 1a & + & 6b & + & 0c & + & 5d\\
    12.34b & = & 4a & + & 1b & + & 7c & + & 0d\\
    12.34c & = & 0a & + & 8b & + & 1c & + & 3d\\
    12.34d & = & 6a & + & 0b & + & 7c & + & 1d\\
\end{array}\right.
$$
$$
\left\{\begin{array}{ccccccccc}
    0 & = & 11.34a & + & 6b     & + & 0c     & + & 5d    \\
    0 & = & 4a     & + & 11.34b & + & 7c     & + & 0d    \\
    0 & = & 0a     & + & 8b     & + & 11.34c & + & 3d    \\
    0 & = & 6a     & + & 0b     & + & 7c     & + & 11.34d\\
\end{array}\right.
$$
Si iteramos unas cuantas veces con el método de Jacobi
$$
v^{(k+1)} = 
\left(\begin{array}{cccc}
    0.881 & 0 & 0 & 0 \\
    0 & 0.881 & 0 & 0 \\
    0 & 0 & 0.881 & 0 \\
    0 & 0 & 0 & 0.881
\end{array}\right)\left(-
\left(\begin{array}{ccccccccc}
 0 & 6 & 0 & 5\\
 4 & 0 & 7 & 0\\
 0 & 8 & 0 & 3\\
 6 & 0 & 7 & 0\\
\end{array}\right) v^{(k)}\right)
$$
\begin{python}
>>> import numpy as np
>>> D = np.array([[-0.881, 0, 0, 0], [0, -0.881, 0, 0], [0, 0, -0.881, 0], [0, 0, 0, -0.881]])
>>> v = np.array([[1],[1],[1],[1]])
>>> R = np.array([[ 0 , 6 , 0 , 5], [ 4 , 0 , 7 , 0], [ 0 , 8 , 0 , 3], [ 6 , 0 , 7 , 0]])
>>> for k in range(15):
        v = np.dot(D, np.dot(-1*R,v))
        print(v.T)

        
[[ 9.691  9.691  9.691 11.453]]
[[101.677091  93.915481  98.572447 110.991023]]
[[ 985.35268888  966.20634933  955.26558388 1145.36138368]]
[[10152.68365766  9363.50573139  9837.01248715 11099.69716919]]
[[ 98389.65732641  96442.91321787  95330.48801299 114331.94182267]]
[[1013429.44299852  934628.27199437  981908.97459686 1107990.8482035 ]]
[[ 9821144.7320987   9626758.00346566  9515679.87281821 11412420.68202906]]
[[1.01158756e+08 9.32929118e+07 9.80124183e+07 1.10597769e+08]]
[[9.80329504e+08 9.60926039e+08 9.49838345e+08 1.13916777e+09]]
[[1.00974891e+10 9.31233425e+09 9.78342713e+09 1.10396748e+10]]
[[9.78547665e+10 9.59179466e+10 9.48111924e+10 1.13709722e+11]]
[[1.00791359e+12 9.29540820e+11 9.76564483e+11 1.10196092e+12]]
[[9.76769062e+12 9.57436067e+12 9.46388641e+12 1.13503044e+13]]
[[1.00608161e+14 9.27851292e+13 9.74789486e+13 1.09995800e+14]]
[[9.74993693e+14 9.55695837e+14 9.44668491e+14 1.13296742e+15]]
\end{python}
Como el vector propio tiene el mismo signo en todas los componentes continuamos. El ganador es David ya que es el mayor el valor absoluto.
\\\\
1. D\\
2. A\\
3. B\\
4. C
\end{document}