\documentclass{article}
\usepackage[utf8]{inputenc}
\usepackage{graphicx}
\usepackage[spanish]{babel}
\usepackage{amssymb,amsmath,geometry,multicol,spalign,hyperref}
\usepackage[usenames,dvipsnames]{xcolor}
\usepackage{tikz,mathtools}
\usepackage{pgfplots}
\pgfplotsset{every axis/.append style={
                    axis x line=middle,    % put the x axis in the middle
                    axis y line=middle,    % put the y axis in the middle
                    axis line style={<->,color=blue}, % arrows on the axis
                    xlabel={$x$},          % default put x on x-axis
                    ylabel={$y$},          % default put y on y-axis
            }}
\usepackage{etoolbox} %titulo
\makeatletter %titulo
\patchcmd{\@maketitle}{\vskip 2em}{\vspace*{-3cm}}{}{} %titulo
\makeatother %titulo
\usepackage{vmargin}
\setpapersize{A4}
\setmargins{2.5cm}       % margen izquierdo
{1.5cm}                        % margen superior
{16.5cm}                      % anchura del texto
{23.42cm}                    % altura del texto
{10pt}                           % altura de los encabezados
{1cm}                           % espacio entre el texto y los encabezados
{0pt}                             % altura del pie de página
{2cm}                           % espacio entre el texto y el pie de página
\title{Seminario}
\author{Andoni Latorre Galarraga}
\date{}
\newcommand{\bb}[1]{\mathbb{#1}}
\newcommand{\R}{\bb{R}}
\newcommand{\nota}[3][2ex]{
    \underset{\mathclap{
        \begin{tikzpicture}
          \draw[->] (0, 0) to ++(0,#1);
          \node[below] at (0,0) {#3};
        \end{tikzpicture}}}{#2}
}
\begin{document}

\maketitle
\begin{multicols}{2}
\noindent \textbf{1.}\\
Veamos que $\text{Gal}(\bb{Q}(\zeta)/\bb{Q})$ es cíclico de orden $p-1$. Sea $\sigma \in \text{Gal}(\bb{Q}(\zeta)/\bb{Q})$.
$$
\zeta^p -1 = 0 \Rightarrow \sigma(\zeta)^p -1 = 0
$$
Tenemos que $\sigma(\zeta)$ es raiz $p$-ésima de la unidad. Ahora, por ser $\zeta$ raiz primitiva.
$$
\sigma(\zeta) = \zeta^d
$$
para $d\in \{1, \cdots, p-1\}$. Veamos que $\sigma$ es automorfismo. Teniendo en cuenta que $\{\zeta, \zeta^2,\cdots, \zeta^{p-1}\}$ es base de $\bb{Q}(\zeta)/\bb{Q}$:
$$
\begin{array}{clcl}
\sigma : & \bb{Q}(\zeta) & \longrightarrow & \bb{Q}(\zeta) \\
& q\in\bb{Q} & \longmapsto     & q \\
& \zeta & \longmapsto     & \zeta^d \\
& \zeta^2 & \longmapsto     & \zeta^{2d}\\
& & \vdots & \\
& \zeta^{p-1} & \longmapsto     & \zeta^{(p-1)d}
\end{array}
$$
Supongamos que $\sigma(\zeta^{k_1}) = \sigma(\zeta^{k_2})$ con $k_1 > k_2$.
$$
\zeta^{dk_1} = \zeta^{dk_1} \Rightarrow 1 = \zeta^{d(k_1-k_2)} \Rightarrow p\mid d(k_1-k_2)
$$
pero $p$ es primo y $d, (k_1-k_2) \in \{1, \cdots, p-1\}$, que es contradictorio. Por lo tanto $\sigma$ es inyectivo y
$$
\sigma(\{\zeta, \zeta^2,\cdots, \zeta^{p-1}\}) = \{\zeta, \zeta^2,\cdots, \zeta^{p-1}\}
$$
es decir, $\sigma$ es automorfismo. Si llamamos $\sigma_d$ al automorfismo que satisface $\zeta \longmapsto \zeta^d$. tenemos que
$$
\begin{array}{crcl}
\varphi : & \text{Gal}(\bb{Q}(\zeta)/\bb{Q}) & \longrightarrow & ((\bb{Z}/p\bb{Z})^*, \cdot) \\
& \sigma_k & \longmapsto     & \overline{k}
\end{array}
$$
Es isomorfismo de grupos
$$
\sigma_a \circ \sigma_b : \zeta \longmapsto \zeta^{ab} \Rightarrow \varphi(\sigma_a \circ \sigma_b) = \overline{ab}
$$
$\text{Gal}(\bb{Q}(\zeta)/\bb{Q}) \simeq ((\bb{Z}/p\bb{Z})^*, \cdot) \simeq (\bb{Z}/(p-1)\bb{Z}, +)$, Probando que $\text{Gal}(\bb{Q}(\zeta)/\bb{Q})$ es cíclico. Ahora,
$$
|\text{Gal}(\bb{Q}(\zeta)/\bb{Q}) : \text{Gal}(\bb{Q}(\zeta)/E) | | \text{Gal}(\bb{Q}(\zeta)/E) | =
$$
$$
= | \text{Gal}(\bb{Q}(\zeta)/\bb{Q}) |
$$
$$
\Rightarrow
[E:\bb{Q}] | \text{Gal}(\bb{Q}(\zeta)/E) | = | \text{Gal}(\bb{Q}(\zeta)/\bb{Q}) |
$$
$$
\Rightarrow
2 | \text{Gal}(\bb{Q}(\zeta)/E) | = | \text{Gal}(\bb{Q}(\zeta)/\bb{Q}) |
$$
$$
\Rightarrow
| \text{Gal}(\bb{Q}(\zeta)/E) | = \frac{p-1}{2}
$$
Como $\frac{p-1}{2}$ es divisor de $p-1$ existe un único subgrupo de orden $\frac{p-1}{2}$ (2.30 en el libro de rojo). Resumiendo, si $[E:\bb{Q}] = 2$ entonces solo existe un posible $\text{Gal}(\bb{Q}(\zeta)/E)$ y por lo tanto $\exists ! E$ por la correspondencia de Galois ya que $\bb{Q}(\zeta)/\bb{Q}$ es de Galois $p-1 = [\bb{Q}(\zeta):\bb{Q}]$.\\
Supongamos que $E\subseteq \R \cap \bb{Q}(\zeta)$. Tenemos que $[\bb{Q}(\zeta): \bb{Q}(\zeta)\cap \R] = 2$ ya que $\bb{Q}(\zeta) = \left(\bb{Q}(\zeta)\cap \R\right)(i)$. Entonces,
\begin{center}
  \begin{tikzpicture}

  \node (F1) at (0,0) {$\mathbb{Q}$};
  \node (F2) at (0,1) {$E$};
  \node (F3) at (0,2) {$\bb{Q}(\zeta)\cap \R$};
  \node (F4) at (0,3) {$\bb{Q}(\zeta)$};

  \draw (F1)--(F2) node [pos=0.5, right,inner sep=0.25cm] {2};
  \draw (F2)--(F3);
  \draw (F3)--(F4) node [pos=0.5, right,inner sep=0.25cm] {2};

  \end{tikzpicture}
\end{center}
$\Rightarrow 4\mid [\bb{Q}(\zeta):\bb{Q}] = p-1 \Rightarrow p \equiv 1 (\text{mod }4)$\\
\noindent\textbf{2.}\\
\textit{i)}\\
Las raices de $(x^3-3)(x^2-3)$ son $\{\sqrt[3]{3}\zeta_3, \sqrt[3]{3}\zeta_3^2, \pm \sqrt{3}\}$.
$$
\zeta_3 = e^{i\frac{2\pi}{3}} = \cos(\frac{2\pi}{3}) + i \sin(\frac{3\pi}{3}) = -\frac{1}{2} + i \frac{\sqrt{3}}{2}
$$
$$
\zeta_3^2 = e^{i\frac{4\pi}{3}} = \cos(\frac{2\pi}{3}) + i \sin(\frac{4\pi}{3}) = -\frac{1}{2} - i \frac{\sqrt{3}}{2}
$$
Veamos que
$$
\left\{ -\frac{\sqrt[3]{3}}{2} \pm i \frac{\sqrt[6]{3}}{2}, \pm \sqrt{3} \right\} \subset \bb{Q}(\sqrt[6]{3}, i)
$$
Es suficiente con observar que $\sqrt[6]{3}^2 = \sqrt[3]{3}$ y $\sqrt[6]{3}^3 = \sqrt[2]{3}$. Veamos ahora que
$$
\{ \sqrt[6]{3}, i \} \subset \bb{Q}\left( -\frac{\sqrt[3]{3}}{2} \pm i \frac{\sqrt[6]{3}}{2}, \pm \sqrt{3} \right)
$$
$$
-\left( -\frac{\sqrt[3]{3}}{2} + i \frac{\sqrt[6]{3}}{2} -\frac{\sqrt[3]{3}}{2} - i \frac{\sqrt[6]{3}}{2} \right) = \sqrt[3]{3}
$$
$$
\frac{\sqrt{3}}{\sqrt[3]{3}} = \sqrt[6]{3}
$$
$$
\left( -\frac{\sqrt[3]{3}}{2} + i \frac{\sqrt[6]{3}}{2} \right) - \left( -\frac{\sqrt[3]{3}}{2} - i \frac{\sqrt[6]{3}}{2} \right) = i \sqrt[6]{3}
$$
$$
i \frac{\sqrt[6]{3}}{\sqrt[6]{3}} = i
$$
Tenemos que $\bb{Q}(\sqrt[6]{3}, i) = \bb{Q}\left( -\frac{\sqrt[3]{3}}{2} \pm i \frac{\sqrt[6]{3}}{2}, \pm \sqrt{3} \right)$. $\bb{Q}(\sqrt[6]{3}, i)$ es el cuerpo de escisión de $(x^3-3)(x^2-3)$ sobre $\bb{Q}$ y $F/\bb{Q}$ es de Galois.\\
Las raices de $x^3 + \sqrt{3}$ son $-\sqrt[6]{3}$ y $\frac{\sqrt[6]{3}}{2} \pm i\frac{\sqrt[3]{3}^2}{2}$. Por un lado,
$$
\left( \frac{\sqrt[6]{3}}{2} + i\frac{\sqrt[3]{3}^2}{2} \right) - \left( \frac{\sqrt[6]{3}}{2} - i\frac{\sqrt[3]{3}^2}{2} \right) = i \sqrt[3]{3}^2
$$
$$
i \frac{\sqrt[3]{3}^2}{\sqrt[6]{3}^4} = i
$$
$$
\Rightarrow F \subset \bb{Q}\left( -\sqrt[6]{3}, \frac{\sqrt[6]{3}}{2} \pm i\frac{\sqrt[3]{3}^2}{2} \right)
$$
Por otro lado,
$$
\sqrt[6]{3}^2 = \sqrt[3]{3} \Rightarrow \bb{Q}\left( -\sqrt[6]{3}, \frac{\sqrt[6]{3}}{2} \pm i\frac{\sqrt[3]{3}^2}{2} \right) \subset F
$$
es decir, $F$ es el cuerpo de escisión de $x^3 + \sqrt{3}$ sobre $\bb{Q}(\sqrt{3})$.\\
\textit{ii)}\\
Consideramos
$$
\sigma :
\left\{\begin{array}{ccc}
    \zeta & \longmapsto & \zeta^d\\
    \sqrt[6]{3} & \longmapsto & \sqrt[6]{3} \\
\end{array}\right.
$$
Si $d = 2$
$$
\sigma(\zeta^3) = \zeta^6 = 1 = \sigma(1)
$$
Si $d = 3$
$$
\sigma(\zeta^2) = \zeta^6 = 1 = \sigma(1)
$$
Si $d = 4$
$$
\sigma(\zeta^4) = \zeta^12 = 1 = \sigma(1)
$$
Ninguno es automorfismo por no ser inyectivo. Con $d=5$
$$
\sigma:\zeta\longmapsto \zeta^5 \longmapsto \zeta
$$
$$
\sigma:\zeta^2\longmapsto \zeta^4 \longmapsto \zeta^2
$$
$$
\sigma:\zeta^3\longmapsto \zeta^3
$$
Es automorfismo de orden 2. Ahora, consideramos
$$
\tau :
\left\{\begin{array}{ccc}
    \zeta & \longmapsto & \zeta\\
    \sqrt[6]{3} & \longmapsto & \zeta \sqrt[6]{3} \\
\end{array}\right.
$$
$$
\tau: \sqrt[6]{3} \longmapsto \zeta \sqrt[6]{3} \longmapsto \zeta^2 \sqrt[6]{3} \longmapsto \cdots \longmapsto \zeta^5 \sqrt[6]{3} \longmapsto \sqrt[6]{3}
$$
Es automorfismo de orden 6. Además podemos mandar $\sqrt[6]{3}$ a cualquier raiz de $x^6-3$ haciendo $\tau(\tau(\cdots(\tau(\sqrt[6]{3}))))$. Por lo tanto
$$
\text{Gal}(F/\bb{Q}) = \{\text{Id}, \sigma, \tau, \tau^2, \tau^3, \tau^4, \tau^5, \sigma\tau, \sigma\tau^2, \sigma\tau^3, \sigma\tau^4, \sigma\tau^5\}
$$
Que es isomorfo a $D_{12}$.\\
\textit{iii)}\\
$$
| \text{Gal}(F/\bb{Q}) : \text{Gal}(F/E) | | \text{Gal}(F/E) | = | \text{Gal}(F/\bb{Q}) |
$$
Como $6 = [E:\bb{Q}] = | \text{Gal}(F/\bb{Q}) : \text{Gal}(F/E) |$ y $| \text{Gal}(F/\bb{Q}) | = 12$.
$$
| \text{Gal}(F/E) | = 2
$$
Que nos deja como opciones para $\text{Gal}(F/E)$
$$
\langle \sigma \rangle, \langle \tau^3 \rangle, \langle \sigma \tau \rangle, \langle \sigma \tau^2 \rangle, \langle \sigma \tau^3 \rangle, \langle \sigma \tau^4 \rangle, \langle \sigma \tau^5 \rangle
$$
Y los subcuerpos intermedios serían
$$
\text{Fix}\langle \sigma \rangle, \text{Fix}\langle \tau^3 \rangle, \text{Fix}\langle \sigma \tau \rangle, \text{Fix}\langle \sigma \tau^2 \rangle
$$
$$
\text{Fix}\langle \sigma \tau^3 \rangle, \text{Fix}\langle \sigma \tau^4 \rangle, \text{Fix}\langle \sigma \tau^5 \rangle
$$
\textit{iv)}\\
No, $D_{12}$ no teine elementos de orden 4 y por lo tanto no tiene subgrupos cíclicos de orden 4.
\end{multicols}
\end{document}