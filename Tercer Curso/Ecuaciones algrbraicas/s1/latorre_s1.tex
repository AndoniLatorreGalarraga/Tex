\documentclass{article}
\usepackage[utf8]{inputenc}
\usepackage{graphicx}
\usepackage[spanish]{babel}
\usepackage{amssymb,amsmath,geometry,multicol,spalign,hyperref}
\usepackage[usenames,dvipsnames]{xcolor}
\usepackage{tikz,mathtools}
\usepackage{pgfplots}
\pgfplotsset{every axis/.append style={
                    axis x line=middle,    % put the x axis in the middle
                    axis y line=middle,    % put the y axis in the middle
                    axis line style={<->,color=blue}, % arrows on the axis
                    xlabel={$x$},          % default put x on x-axis
                    ylabel={$y$},          % default put y on y-axis
            }}
\usepackage{etoolbox} %titulo
\makeatletter %titulo
\patchcmd{\@maketitle}{\vskip 2em}{\vspace*{-3cm}}{}{} %titulo
\makeatother %titulo
\usepackage{vmargin}
\setpapersize{A4}
\setmargins{2.5cm}       % margen izquierdo
{1.5cm}                        % margen superior
{16.5cm}                      % anchura del texto
{23.42cm}                    % altura del texto
{10pt}                           % altura de los encabezados
{1cm}                           % espacio entre el texto y los encabezados
{0pt}                             % altura del pie de página
{2cm}                           % espacio entre el texto y el pie de página
\title{Seminario 1}
\author{Andoni Latorre Galarraga}
\date{}
\newcommand{\bb}[1]{\mathbb{#1}}
\newcommand{\R}{\bb{R}}
\newcommand{\nota}[3][2ex]{
    \underset{\mathclap{
        \begin{tikzpicture}
          \draw[->] (0, 0) to ++(0,#1);
          \node[below] at (0,0) {#3};
        \end{tikzpicture}}}{#2}
}
\begin{document}

\maketitle

$\boxed{a)\Rightarrow b)}$\\
En el caso $n=1$, como $f$ no puede tener más de una raiz y tiene al menos una, $f$ tiene exactamnete una raiz. Ahora, argumentamos por induccíon sobre el grado del polinomio. Sea $f(x)\in k[x]$ de grado $\delta(f)=n\ge 2$, como tiene al menos una raiz, supongamos que $a\in K$ es raiz de $f$. Tenemos que $(x-a)\mid f(x)$. Es decir, $f(x)=(x-a)g(x)$, y también $\delta(g)=n-1$. Por la hipótesis de inducción $g$ tiene $n-1$ raices y $f$ se anula en esas $n-1$ raices y en $a$ dando un total de $n$ raices cmo se quería probar.\\

$\boxed{b)\Rightarrow a)}$\\
Si $f$ tiene $n$ raices y $n\ge 1$ entonces, $f$ tiene al menos una raiz.\\

$\boxed{c)\Rightarrow d)}$\\
Como $K[x]$ es DFU todo polinomio se puede escribir como producto de irreducibles. Veamos que todo polinomio de grado 1 se puede escibir en la forma $c(x-a)$ para concluir $d)$.\\
$$
ux+v \nota{=}{$K$ cuerpo} u(x+u^{-1}v)
$$

$\boxed{d)\Rightarrow c)}$\\
Los polinomios de grado 1 siempre son irreducibles. Además, tenemos que un polinomio de grado $n\ge 2$ no es irreducible ya que se puede poner de la forma $c(x-a_1^{n_1})\cdots (x-a_t^{n_t})$ y los polinomios de grado 1 no son unidades.\\

$\boxed{d)\Rightarrow a)}$\\
Evidentemente $a_1$ es raiz de $f$.\\

$\boxed{a)\Rightarrow d)}$\\
Argumentando por inducción como en $\boxed{a)\Rightarrow b)}$ para obtener $f(x)=(x-a)g(x)$ con $\delta(g)=n-1$ y reescribiendo como en $\boxed{c)\Rightarrow d)}$ si fuera necesario se tiene el resultado deseado.

\end{document}