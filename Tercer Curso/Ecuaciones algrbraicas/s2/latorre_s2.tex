\documentclass{article}
\usepackage[utf8]{inputenc}
\usepackage{graphicx}
\usepackage[spanish]{babel}
\usepackage{amssymb,amsmath,geometry,multicol,spalign,hyperref}
\usepackage[usenames,dvipsnames]{xcolor}
\usepackage{tikz,mathtools}
\usepackage{pgfplots}
\pgfplotsset{every axis/.append style={
                    axis x line=middle,    % put the x axis in the middle
                    axis y line=middle,    % put the y axis in the middle
                    axis line style={<->,color=blue}, % arrows on the axis
                    xlabel={$x$},          % default put x on x-axis
                    ylabel={$y$},          % default put y on y-axis
            }}
\usepackage{etoolbox} %titulo
\makeatletter %titulo
\patchcmd{\@maketitle}{\vskip 2em}{\vspace*{-3cm}}{}{} %titulo
\makeatother %titulo
\usepackage{vmargin}
\setpapersize{A4}
\setmargins{2.5cm}       % margen izquierdo
{1.5cm}                        % margen superior
{16.5cm}                      % anchura del texto
{23.42cm}                    % altura del texto
{10pt}                           % altura de los encabezados
{1cm}                           % espacio entre el texto y los encabezados
{0pt}                             % altura del pie de página
{2cm}                           % espacio entre el texto y el pie de página
\title{Seminario 2}
\author{Andoni Latorre Galarraga}
\date{}
\newcommand{\bb}[1]{\mathbb{#1}}
\newcommand{\R}{\bb{R}}
\newcommand{\nota}[3][2ex]{
    \underset{\mathclap{
        \begin{tikzpicture}
          \draw[->] (0, 0) to ++(0,#1);
          \node[below] at (0,0) {#3};
        \end{tikzpicture}}}{#2}
}
\begin{document}

\maketitle

\noindent Tenemos que $b^2=(2a^2-3a+2)^2=4 a^4 - 12 a^3 + 17 a^2 - 12 a + 4$. Si sutituimos $a^3=a-1$.
$$
b^2 = 4 a(a-1) - 12 (a-1) + 17 a^2 - 12 a + 4 = 21 a^2 - 28 a + 16
$$
$$
2b^2-21b= 2 (21 a^2 - 28 a + 16) - 21 (2 a^2 - 3 a + 2) = 7 a - 10
$$
$$
2b^2 - 21b + 10 = 7 a
$$
Como $b = 2a^2 - 3a + 2$.
$$
49b = 2 (7a)^2 - 21 (7a) + 98 \Rightarrow
49b = 2 (2b^2 - 21b + 10)^2 - 21 (2b^2 - 21b + 10) + 98
$$
$$
\Rightarrow 0 = 8 b^4 - 168 b^3 + 920 b^2 - 448 b + 88
$$
Calculamos $\text{Irr} (b, \bb{Q})$.
$$
8 b^4 - 168 b^3 + 920 b^2 - 448 b + 88 = 8 (b^4 - 21 b^3 + 115 b^2 - 56 b + 11) = 8 (b - 11) (b^3 - 10 b^2 + 5 b - 1)
$$
Como $x^3 - 10 x^2 + 5 x - 1$ no tiene raices en $\bb{Q}$, es irreducible.
$$
\text{Irr} (b, \bb{Q}) = x^3 - 10 x^2 + 5 x - 1
$$

\end{document}