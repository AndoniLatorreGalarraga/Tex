\documentclass{article}
\usepackage[utf8]{inputenc}
\usepackage{graphicx}
\usepackage[spanish]{babel}
\usepackage{amssymb,amsmath,geometry,multicol,spalign,hyperref}
\usepackage[usenames,dvipsnames]{xcolor}
\usepackage{tikz,mathtools}
\usepackage{pgfplots}
\pgfplotsset{every axis/.append style={
                    axis x line=middle,    % put the x axis in the middle
                    axis y line=middle,    % put the y axis in the middle
                    axis line style={<->,color=blue}, % arrows on the axis
                    xlabel={$x$},          % default put x on x-axis
                    ylabel={$y$},          % default put y on y-axis
            }}
\usepackage{etoolbox} %titulo
\makeatletter %titulo
\patchcmd{\@maketitle}{\vskip 2em}{\vspace*{-3cm}}{}{} %titulo
\makeatother %titulo
\usepackage{vmargin}
\setpapersize{A4}
\setmargins{2.5cm}       % margen izquierdo
{1.5cm}                        % margen superior
{16.5cm}                      % anchura del texto
{23.42cm}                    % altura del texto
{10pt}                           % altura de los encabezados
{1cm}                           % espacio entre el texto y los encabezados
{0pt}                             % altura del pie de página
{2cm}                           % espacio entre el texto y el pie de página
\title{Seminario 3}
\author{Andoni Latorre Galarraga}
\date{}
\newcommand{\bb}[1]{\mathbb{#1}}
\newcommand{\R}{\bb{R}}
\newcommand{\nota}[3][2ex]{
    \underset{\mathclap{
        \begin{tikzpicture}
          \draw[->] (0, 0) to ++(0,#1);
          \node[below] at (0,0) {#3};
        \end{tikzpicture}}}{#2}
}
\begin{document}

\maketitle

a)\\
Veamos que ocurre en el caso $n=1$. Si tenemos en cuenta Einsestein con $p=a_1$,
$$
\text{Irr}(\sqrt{a_1})\bb{Q}) = x^2-a_1
$$
Y tenemos que $|\bb{Q}(\sqrt{a_1}):\bb{Q}| = 2$. Para aplicar inducción, por el teorema del grado, tenemos que
$$
|\bb{Q}(\sqrt{a_1},\cdots,\sqrt{a_n}):\bb{Q}|
=
|\bb{Q}(\sqrt{a_1},\cdots,\sqrt{a_{n-1}})(\sqrt{a_n}):\bb{Q}|
=
$$
$$
=
|\bb{Q}(\sqrt{a_1},\cdots,\sqrt{a_{n-1}}):\bb{Q}|
\cdot
|\bb{Q}(\sqrt{a_n}):\bb{Q}(\sqrt{a_1},\cdots,\sqrt{a_{n-1}})|
\nota{=}{Hipótesis de inducción}
$$
$$
=
2^{n-1}
\cdot
|\bb{Q}(\sqrt{a_n}):\bb{Q}(\sqrt{a_1},\cdots,\sqrt{a_{n-1}})|
$$
Veamos ahora que $|\bb{Q}(\sqrt{a_n}):\bb{Q}(\sqrt{a_1},\cdots,\sqrt{a_{n-1}})|=2$. Por una parte es menor o igual que 2 ya que $\sqrt{a_n}$ es raiz de $x^2-a_n$. Por otra parte, no es 1 ya que $\sqrt{a_n}\notin \bb{Q}(\sqrt{a_1},\cdots,\sqrt{a_{n-1}})$. Cocluimos que
$$
|\bb{Q}(\sqrt{a_1},\cdots,\sqrt{a_n}):\bb{Q}| = 2^{n-1} 2 = 2^n
$$

b)\\
Veamos que ocurre con $\bb{Q}(\sqrt{2},\sqrt{3},\sqrt{6})$.
$$
|\bb{Q}(\sqrt{2},\sqrt{3},\sqrt{6}):\bb{Q}|=|\bb{Q}(\sqrt{2},\sqrt{3})(\sqrt{6}):\bb{Q}| =
$$
$$
= |\bb{Q}(\sqrt{2},\sqrt{3}):\bb{Q}| \cdot |\bb{Q}(\sqrt{6}):\bb{Q}(\sqrt{2},\sqrt{3})| =
$$
$$
= 2^2 \cdot |\bb{Q}(\sqrt{6}):\bb{Q}(\sqrt{2},\sqrt{3})|
$$
Como $\sqrt{6}=\sqrt{2}\sqrt{6}$, deducimos $|\bb{Q}(\sqrt{6}):\bb{Q}(\sqrt{2},\sqrt{3})|=1$
$$
|\bb{Q}(\sqrt{2},\sqrt{3},\sqrt{6}):\bb{Q}|=2^2\cdot 1= 2^2 \ne 2^3
$$

c)\\
$$
|\bb{Q}(p,\sqrt{p}):\bb{Q}|
= |\underbrace{\bb{Q}(p)}_{\nota{=}{$p\in\bb{Q}$} \bb{Q}}(\sqrt{p}):\bb{Q}
= |\bb{Q}(\sqrt{p}):\bb{Q}|
\nota{=}{a)} 2 < \infty
$$

d)\\
Como $E\subseteq F$ y $|F:\bb{Q}| = 2$,
$$
|E:\bb{Q}|\le 2 \quad \Leftrightarrow \quad |E:\bb{Q}|\in \{1,2\} = \{2^0, 2^1\}
$$
\end{document}