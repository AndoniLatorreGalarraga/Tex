\documentclass{article}
\usepackage[utf8]{inputenc}
\usepackage{graphicx}
\usepackage[spanish]{babel}
\usepackage{amssymb,amsmath,geometry,xcolor,multicol}
\usepackage{etoolbox} %titulo
\makeatletter %titulo
\patchcmd{\@maketitle}{\vskip 2em}{\vspace*{-3cm}}{}{} %titulo
\makeatother %titulo
\usepackage{vmargin}
\setpapersize{A4}
\setmargins{2.5cm}       % margen izquierdo
{1.5cm}                        % margen superior
{16.5cm}                      % anchura del texto
{23.42cm}                    % altura del texto
{10pt}                           % altura de los encabezados
{1cm}                           % espacio entre el texto y los encabezados
{0pt}                             % altura del pie de página
{2cm}                           % espacio entre el texto y el pie de página
\title{Ejercicio 8}
\author{Andoni Latorre Galarraga}
\date{}
\newcommand{\bb}[1]{\mathbb{#1}}
\newcommand{\p}{\textbf{Proposición: }}
\newcommand{\dem}{\textit{Dem: }}
\newcommand{\R}{\mathbb{R}}
\begin{document}

\maketitle

\section*{a)}

Tenemos que $X^2 + 1 - (Y^2 + 1) = X^2 - Y^2 \in (X^2 + 1, Y^2 + 1)$. Pero $(X + Y)(X - Y) = X^2 - Y^2$ y
$$
\text{deg}_X(X + Y) = \text{deg}_X(X - Y) = \text{deg}_Y(X + Y) = \text{deg}_Y(X - Y) < \text{deg}_X(X^2 + 1) = \text{deg}_Y(Y^2 + 1)
$$
$$
\Rightarrow X + Y, X - Y \notin (X^2 + 1, Y^2 + 1)
$$

\section*{b)}

Veamos que $\frac{\R[X,Y]}{(X^2 + 1, X^2Y)}$ es DI. Plantemos el homorfismo
$$
\begin{array}{crcl}
\varphi : & \R[X,Y] & \longrightarrow &  \bb{C} \\
& X & \longmapsto & i \\
& Y & \longmapsto & 0 \\
& c \in \R & \mapsto & c
\end{array}
$$
Calculamos $\text{ker}\varphi$. Sea $\varphi(f) = 0$ si dividimos $f$ entre $Y$ tenmos que $f = g Y + h$ con $\text{deg}_Y(h) = 0$ entonces $\varphi(f) = 0 \Rightarrow \varphi(h) = 0$ pero $h \in \R[X]$ es tal que $i$ es una raiz y por tanto $-i$ también. Es decir $h \in (X^2 + 1)$ ya que $(X+i)(X-i) = X^2 + 1$. Concluimos que $\text{ker}\varphi = (X^2 + 1, Y)$. Ahora, veamos que $(X^2 + 1, Y) = (X^2 + 1, X^2Y)$. Sean $a,b\in\R[X,Y]$,
$$
a (X^2+1) + b X^2 Y = a (X^2+1) + b ((X^2+1)Y - Y) = (a + bY) (X^2+1) - bY \in (X^2 + 1) - bY
$$
$$
a (X^2+1) + b Y = a (X^2+1) + b ((X^2+1)Y - X^2Y) = (a + bY) (X^2+1) - bX^2Y \in (X^2 + 1) - bY
$$
Tenemos que $\frac{\R[X,Y]}{(X^2 + 1, X^2Y)} \simeq \varphi(\R[X,Y]) \subseteq \bb{C}$ que concluye la prueba.

\end{document}