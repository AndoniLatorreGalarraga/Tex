\documentclass{article}
\usepackage[utf8]{inputenc}
\usepackage{graphicx}
\usepackage[spanish]{babel}
\usepackage{amssymb,amsmath,geometry,xcolor,multicol,scalerel}
\usepackage{etoolbox} %titulo
\makeatletter %titulo
\patchcmd{\@maketitle}{\vskip 2em}{\vspace*{-3cm}}{}{} %titulo
\makeatother %titulo
\usepackage{vmargin}
\setpapersize{A4}
\setmargins{2.5cm}       % margen izquierdo
{1.5cm}                        % margen superior
{16.5cm}                      % anchura del texto
{23.42cm}                    % altura del texto
{10pt}                           % altura de los encabezados
{1cm}                           % espacio entre el texto y los encabezados
{0pt}                             % altura del pie de página
{2cm}                           % espacio entre el texto y el pie de página
\title{Seminario 5}
\author{Andoni Latorre Galarraga, Mariana Emilia Zaballa Bernabe}
\date{}
\newcommand{\bb}[1]{\mathbb{#1}}
\newcommand{\p}{\textbf{Proposición: }}
\newcommand{\dem}{\textit{Dem: }}
\newcommand{\R}{\mathbb{R}}
\begin{document}

\maketitle

\noindent \textbf{9.}\\
Consideramos el homomorfismo evaluación $\varphi_1$.
$$
\begin{array}{crcl}
\varphi_1 : & K[X,Y]=K[X][Y] & \longrightarrow & K[X] \\
& p(Y) & \longmapsto     & p(g(X))
\end{array}
$$
Junto con el homomorfismo $\varphi_2$.
$$
\begin{array}{crcl}
\varphi_2 : & K[X] & \longrightarrow & \frac{K[X]}{(f(X))}=D \\
& p & \longmapsto     & p+(f(X))
\end{array}
$$
Tenemos que $\varphi=\varphi_1 \circ \varphi_2$ es homomorfismo por ser composición de homomorfismos.
$$
\begin{array}{crcccl}
\varphi : & K[X,Y]=K[X][Y] & \longrightarrow & K[X] & \longrightarrow & \frac{K[X]}{(f(X))}=D \\
& p(Y) & \longmapsto & p(g(X)) & \longmapsto & p(g(X))+(f(X))
\end{array}
$$
$$
\begin{array}{crcl}
\varphi : & K[X,Y] & \longrightarrow & D \\
& X & \longmapsto & X +(f(X)) \\
& Y & \longmapsto & g(x) +(f(X)) \\
& k\in K & \longmapsto & k +(f(X)) \\
\end{array}
$$
Veamos que $ker\varphi = (f(X), Y-g(X))$.\\
$\boxed{ker\varphi \supseteq (f(X), Y-g(X))}$
$$
\varphi(f(X)) = f(X) +(f(X)) = 0_D
$$
$$
\varphi(Y-g(X)) = \varphi(Y) - \varphi(g(X)) = (g(X)+(f(X))) - (g(x)+(f(X))) = 0_D
$$
$\boxed{ker\varphi \subseteq (f(X), Y-g(X))}$\\
Sea $p\in ker\varphi$. Dividimos entre $Y-g(X)$.
$$
p = q (Y-g(X)) + r \text{ con } q,r\in K[X,Y]  \text{ y } 1>deg_Y(r)=0 \Rightarrow r\in K[X]
$$
Aplicando $\varphi$:
$$
0_D = \varphi(q)0_D + \varphi(r) = \varphi(r) = r + (f(X)) \Rightarrow r \in (f(X))\subseteq (f(X), Y-g(X))
$$
Tenemos que $p\in(f(X), Y-g(X))$.\\
Por el primer teorema de isomorfía y por ser $\varphi$ sobreyectivo.
$$
\frac{K[X,Y]}{(f(X),Y-g(X))} \sim im(\varphi) = D
$$
$$
f \text{ irreducible } \Leftrightarrow
\frac{K[X]}{(f(X))} \text{ DI } \Leftrightarrow
\frac{K[X,Y]}{(f(X), Y-g(X))} \text{ DI } \Leftrightarrow
(f(X), Y-g(X)) \text{ primo}
$$
\\
\textbf{1.}\\
\textit{a)}\\
Por inducción en $n$.\\
$\boxed{n=1}$
$$
X^\alpha \ge X^0 \text{ porque } \alpha \ge 0
$$
$\boxed{n-1\Rightarrow n}$
$$
(X^{\alpha_1}_1\cdots X^{\alpha_{n-1}}_{n-1})X^{\alpha_n}_n > (X^0_1\cdots X^0_{n-1})X^0_n \text{ y } \alpha_n > 0
$$
\textit{b)}\\
$\boxed{n=1}$
$$
X^\alpha \ge X^\beta \Leftrightarrow \alpha \ge \beta
$$
$\boxed{n-1\Rightarrow n}$
$$
(X^{\alpha_1}_1\cdots X^{\alpha_{n-1}}_{n-1})X^{\alpha_n}_n \ge (X^{\beta_1}_1\cdots X^{\beta_{n-1}}_{n-1})X^{\beta_n}_n
$$
\textit{c)}\\
Tomando $(\bar{X}^{\bar{\alpha}_1},\cdots,\bar{X}^{\bar{\alpha}_n},\cdots)\subseteq Mon(X_1,\cdots,X_n) \Rightarrow \exists \bar{X}^{\bar{\alpha}_i} \in (\bar{X}^{\bar{\alpha}_1},\cdots,\bar{X}^{\bar{\alpha}_n},\cdots) \text{ t.q. } \bar{X}^{\bar{\alpha}_i} < \bar{X}^{\bar{\alpha}_j} \:\: \forall j\ne i$ (es decir, todo subconjnto de $Mon(X_1,\cdots,X_n)$ tiene un elemento mínimo y máximo) $\Rightarrow \forall j>i$ t.q $\bar{X}^{\bar{\alpha}_i} \ge \bar{X}^{\bar{\alpha}_l}$, $\bar{X}^{\bar{\alpha}_i} = \bar{X}^{\bar{\alpha}_j}$.\\
\textbf{2.}\\
\textit{a)}\\
$$
\begin{array}{rrllllllllll}
 6X^2Y &        & -X & +4Y^3 & -1 & & |2XY +Y^3 \\
-6X^2Y & -3XY^3 &    &       &    & & \overline{3X-\frac{3}{2}Y^2} \\
\cline{1-5}
 /\:\:\:     & -3XY^3 & -X & +4Y^3 & -1 & \\
       &  3XY^3 &    &      &     & +\frac{3}{2}Y^5 \\
\cline{2-6}
       & /\:\:\:& -X & +4Y^3 & -1 & +\frac{3}{2}Y^5
\end{array}
$$
$ 6X^2Y -X + 4Y^3 -1 = (2XY+Y^3)(3X-\frac{3}{2}Y^2) -X + 4Y^3 -1 +\frac{3}{2}Y^5$\\
\textit{b)}\\
$$
\begin{array}{rrlllllllllllllllll}
 4Y^3 & +6X^2Y & -X & -1 & & |Y^3+2XY \\
-4Y^3 & -8XY   &    &    & & \overline{4\quad\quad\quad\quad} \\
\cline{1-5}
/\:\:\:     & 6X^2Y & -8XY & -X & -1 \\
\end{array}
$$
$ 6X^2Y -X + 4Y^3 -1 = 4(2XY+Y^3) + 6X^2Y-8XY-X-1$\\
\end{document}