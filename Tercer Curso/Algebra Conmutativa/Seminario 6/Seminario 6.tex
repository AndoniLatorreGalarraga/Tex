\documentclass{article}
\usepackage[utf8]{inputenc}
\usepackage{graphicx}
\usepackage[spanish]{babel}
\usepackage{amssymb,amsmath,geometry,xcolor,multicol}
\usepackage{etoolbox} %titulo
\makeatletter %titulo
\patchcmd{\@maketitle}{\vskip 2em}{\vspace*{-3cm}}{}{} %titulo
\makeatother %titulo
\usepackage{vmargin}
\setpapersize{A4}
\setmargins{2.5cm}       % margen izquierdo
{1.5cm}                        % margen superior
{16.5cm}                      % anchura del texto
{23.42cm}                    % altura del texto
{10pt}                           % altura de los encabezados
{1cm}                           % espacio entre el texto y los encabezados
{0pt}                             % altura del pie de página
{2cm}                           % espacio entre el texto y el pie de página
\title{Seminario 6}
\author{Andoni Latorre Galarraga}
\date{}
\newcommand{\bb}[1]{\mathbb{#1}}
\newcommand{\p}{\textbf{Proposición: }}
\newcommand{\dem}{\textit{Dem: }}
\newcommand{\R}{\mathbb{R}}
\begin{document}

\maketitle
\begin{multicols}{2}
\stepcounter{section}
\section{}
\noindent\textit{c)}\\
Sean $x,y\in \varphi(N_1)$, $a,b\in A$. Entonces,
$$
\exists u,v\in N_1 : \varphi(u)=x, \varphi(v)=y
$$
Ahora, por $N_1 \le M_1$,
$$
\begin{array}{c}
au+bv\in N_1 \\
\Rightarrow \varphi(au+bv)\in\varphi(N_1) \\
\Rightarrow a\varphi(u)+b\varphi(v)\in\varphi(N_1) \\
\Rightarrow ax+by\in\varphi(N_1)
\end{array}
$$
Como se cumple $\forall x,y\in\varphi(N_1)$, $\forall a,b\in A$, tenemos que $\varphi(N_1)\le M_2$.\\\\
Sean $x,y\in\varphi^{-1}(N_2)$, $a,b\in A$. Como $N_2\le M_2$ y $\varphi(x),\varphi(y)\in N_2$.
$$
\begin{array}{c}
\varphi(ax+by)=a\varphi(x)+b\varphi(y)\in N_2 \\
\Rightarrow ax+by\in \varphi^{-1}(N_2)
\end{array}
$$
Como se cumple $\forall x,y\in\varphi^{-1}(N_2)$, $\forall a,b\in A$, tenemos que $\varphi^{-1}(N_2)\le M_1$.\\\\
\textit{d)}
$$
M_1\le M_1 \Rightarrow \text{im}\varphi=\varphi(M_1)\le M_2
$$
$$
\{0_{M_2}\}\le M_2 \Rightarrow \text{ker}\varphi=\varphi^{-1}(\{0_{M_2}\})\le M_1
$$
\section{}
\noindent Veamos que la siguiente aplicación es un isomorfismo.
$$
\begin{array}{crcl}
\overline{\varphi} : & \frac{M_1}{\text{ker}\varphi} & \longrightarrow & \text{im}\varphi \\
& m+\text{ker}\varphi & \longmapsto     & \varphi(m)
\end{array}
$$
Veamos que está bien definido.
$$
\begin{array}{c}
m+\text{ker}\varphi=m'+\text{ker}\varphi \\
\Rightarrow m-m'\in\text{ker}\varphi \\
\Rightarrow \varphi(m-m')=0 \\
\Rightarrow \varphi(m)-\varphi(m')=0 \\
\Rightarrow \varphi(m)=\varphi(m')
\end{array}
$$
Veamos que es homomorfismo.
$$
\begin{array}{c}
    \overline{\varphi}(\overline{x}+\overline{y}) = \overline{\varphi}(\overline{x+y}) = \varphi(x+y) \\
    = \varphi(x)+\varphi(y) = \overline{\varphi}(\overline{x})+\overline{\varphi}(\overline{y})
\end{array}
$$
$$
\overline{\varphi}(a\overline{x}) = \overline{\varphi}(\overline{ax}) = \varphi(ax) = a\varphi(x) = a\overline{\varphi}(\overline{x})
$$
Veamos que es inyectivo.
$$
\begin{array}{c}
    \overline{\varphi}(\overline{x}) = \overline{\varphi}(\overline{y}) \\
    \Rightarrow \overline{\varphi}(\overline{x-y}) = 0 \\
    \Rightarrow \varphi(x-y) = 0 \\
    \Rightarrow x-y\in\text{ker}\varphi \\
    \Rightarrow \overline{x} = \overline{y}
\end{array}
$$
Veamos que es suprayectivo.
$$
\begin{array}{c}
    m\in \text{im}\varphi \\
    \Rightarrow \exists x\in M_1 : \varphi(x)=m \\
    \Rightarrow \overline{\varphi}(\overline{x}) = \varphi(x)=m \text{ con } \overline{x}\in\frac{M_1}{\text{ker}\varphi}
\end{array}
$$
\section{}
\noindent Consideramos la siguiente aplicación.
$$
\begin{array}{crcl}
\varphi : & \frac{M}{N} & \longrightarrow & \frac{M}{L} \\
& m+N & \longmapsto     & m+L
\end{array}
$$
Veamos que está bien definido.
$$
\begin{array}{c}
    m+N = m'+N \Rightarrow m-m'\in N \subseteq L \\
    \Rightarrow m+L = m'+L
\end{array}
$$
Veamos que es homomorfismo.\\
$$
\begin{array}{c}
    \varphi((a+b)+N) = (a+b)+L = (a+L)+(b+L) \\
    = \varphi(a+N) + \varphi(b+N)
\end{array}
$$
Veamos que es suprayectivo. Dado $m+L \in M/L$, $\varphi(m+N) = m+L$.\\
Calculamos el núcleo.
$$
\left(\varphi(m+N)=m+L=0+L \Leftrightarrow m\in L\right) \Rightarrow \text{ker}\varphi = \frac{L}{N}
$$
Por el ejercicio 3, concluimos
$$
\frac{M/N}{L/N} \simeq \frac{M}{L}
$$
\stepcounter{section}
\stepcounter{section}
\stepcounter{section}
\stepcounter{section}
\stepcounter{section}
\stepcounter{section}
\stepcounter{section}
\section{}
\noindent Si $x,y\in A[X^2]$, $a,b\in A$. Evidentemente, $ax+by \in A[X^2]$ y concuimos $A[X^2]\le A[X]$.\\\\
Dos elementos $\overline{p},\overline{q}\in\frac{A[X]}{A[X^2]}$ son iguales si y solo si $p-q\in A[X^2]$. Esto ocurre si y solo si los coeficientes $\left( \sum_{i=0}^N k_i X^i\right)$ con $i$ impar son iguales en $p$ y $q$. Es decir, no importa el valor de de los $k_i$ con $i$ par. Entonces,
$$
\begin{array}{crcl}
\varphi : & A[X]/A[X^2] & \longrightarrow & A[X] \\
& \displaystyle\overline{\sum_{i=0}^N k_i X^i} & \longmapsto     & \displaystyle\sum_{\substack{i=0 \\ i \text{ impar}}}^N k_iX^{\frac{i-1}{2}}
\end{array}
$$
Es inyectivo y está bien definido. Finalmente, como esta claro que es suprayectivo, quedaría ver que es homomorfismo.
$$
\begin{array}{c}
    \displaystyle\varphi\left(a\overline{\sum_{i=0}^N k_i X^i}\right) = \displaystyle\varphi\left(\overline{\sum_{i=0}^N ak_i X^i}\right) \\
    = \displaystyle\sum_{\substack{i=0 \\ i \text{ impar}}}^N ak_iX^{\frac{i-1}{2}} = \displaystyle a \sum_{\substack{i=0 \\ i \text{ impar}}}^N k_iX^{\frac{i-1}{2}} = \displaystyle a \varphi\left(\overline{\sum_{i=0}^N k_i X^i}\right)
\end{array}
$$
$$
\begin{array}{c}
    \displaystyle\varphi\left(\overline{\sum_{i=0}^N k_i X^i} + \overline{\sum_{i=0}^N l_i X^i}\right) = \varphi\left(\overline{\sum_{i=0}^N (k_i+l_i) X^i}\right) \\
    \displaystyle = \sum_{\substack{i=0 \\ i \text{ impar}}}^N (k_i+l_i)X^{\frac{i-1}{2}} = \sum_{\substack{i=0 \\ i \text{ impar}}}^N k_iX^{\frac{i-1}{2}} + \sum_{\substack{i=0 \\ i \text{ impar}}}^N l_iX^{\frac{i-1}{2}} \\
    \displaystyle = \varphi\left(\overline{\sum_{i=0}^N k_i X^i}\right) + \varphi\left(\overline{\sum_{i=0}^N l_i X^i}\right)
\end{array}
$$
Conluimos $\displaystyle\frac{A[X]}{A[X^2]} \simeq A[X]$.
\end{multicols}
\end{document}