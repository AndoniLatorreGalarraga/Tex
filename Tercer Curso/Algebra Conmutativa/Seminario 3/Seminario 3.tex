\documentclass{article}
\usepackage[utf8]{inputenc}
\usepackage{graphicx}
\usepackage[spanish]{babel}
\usepackage{amssymb,amsmath,geometry,xcolor,multicol}
\usepackage{etoolbox} %titulo
\makeatletter %titulo
\patchcmd{\@maketitle}{\vskip 2em}{\vspace*{-3cm}}{}{} %titulo
\makeatother %titulo
\usepackage{vmargin}
\setpapersize{A4}
\setmargins{2.5cm}       % margen izquierdo
{1.5cm}                        % margen superior
{16.5cm}                      % anchura del texto
{23.42cm}                    % altura del texto
{10pt}                           % altura de los encabezados
{1cm}                           % espacio entre el texto y los encabezados
{0pt}                             % altura del pie de página
{2cm}                           % espacio entre el texto y el pie de página
\title{Seminario 3}
\author{Andoni Latorre Galarraga}
\date{}
\newcommand{\bb}[1]{\mathbb{#1}}
\newcommand{\p}{\textbf{Proposición: }}
\newcommand{\dem}{\textit{Dem: }}
\newcommand{\R}{\mathbb{R}}
\begin{document}

\maketitle
\begin{multicols}{2}
\section{} % 1
\textit{a)} Sea $c = \text{mcm}(a, b)$.\\
\indent \indent $\boxed{(a)\cap(b)\supseteq (c)}$
$$
\left.\begin{array}{rcl}
    a\mid c & \Rightarrow & (c) \subseteq (a) \\
    b\mid c & \Rightarrow & (c) \subseteq (b)
\end{array}\right\} \Rightarrow (c) \subseteq (a) \cap (b)
$$
\indent \indent $\boxed{(a)\cap(b)\subseteq (c)}$
$$
x\in (a) \cap (b) \Rightarrow
\left\{\begin{array}{lcr}
    (x) \subseteq (a) & \Rightarrow & a\mid x \\
    (x) \subseteq (b) & \Rightarrow & b\mid x
\end{array}\right.
$$
$$
\Rightarrow
c \mid x \Rightarrow x\in (c) \Rightarrow (a) \cap (b) \subseteq (c)
$$
\indent \textit{c)} En el DFU $\bb{C}[X, Y]$, $(X)+(Y)\ne \indent (\text{MCM}(X,Y))$. Si $C = \text{MCM}(X,Y)$ tenemos que \indent $C = 1$, es decir, $(C) = \bb{C}[X,Y]$ pero $1\notin (X) + (Y)$

\section{}
Sabemos que $r(a)= \{b\in A : \exists n \in \bb{N} : b^n \in(a)\}$. \\
\indent Sea $b \in a$ y su producto en irreducibles $b = \indent q_1^{e_1} \cdots q_r^{e_s}$. Ahora, $b^n = q_1^{n e_1} \cdots q_r^{n e_s}$. Para que \indent $b^n \in (a)$ se necesita que $a\mid b^n$, es decir,
$$
\forall i \in \{1, \cdots, m\} \exists j \in \{1, \cdots, r\} :
$$
$$
\exists n_i\in\bb{N}_{>0} \: p_i^{n_i} \mid q_j^{ne_j} \Leftrightarrow p_i \sim q_j
$$
\indent Entonces $b$ es de la forma $cp_1\cdots p_m$ con $c\in A$.
$$
r(a) = (p_1\cdots p_m)
$$

\stepcounter{section}
\stepcounter{section}
\stepcounter{section}
\stepcounter{section}
\section{}
\textit{a)}\\
\indent $\boxed{\text{ker }\varphi \supseteq (X^2 -Y^3)}$
$$
\varphi(X^2 - Y^2) = \varphi(X)^2 + \varphi(Y)^3 = (T^3)^2 - (T^2)^3
$$
$$
= 0_{K[X,Y]} \Rightarrow \varphi(f(X^2 -Y^3)
$$
$$
= \varphi(f) 0_{K[X,Y]} \forall f\in K[X,Y]
$$
$$
\Rightarrow \text{ker }\varphi \supseteq (X^2 -Y^3)
$$
\indent $\boxed{\text{ker }\varphi \subseteq (X^2 -Y^3)}$\\
\indent Sea $f(X,Y)\in \text{ker}\varphi$. Si dividimos $f(X,Y)$ entre \indent $X^2-Y^3$ respecto de $X$.
$$
f(X,Y) = q(X,Y) (X^2 - Y^3) + r(X,Y)
$$
$$
\text{con } 2>\deg_X(r)
$$
\indent Podemos escribir $r(X,Y) = a(Y)X + b(Y)$.
$$
\varphi(f)=\varphi(q)\varphi(X^2-Y^3)+\varphi(r)
$$
$$
0 = \varphi(a(Y))\varphi(X)+\varphi(b(Y))
$$
$$
0 = \varphi(\sum_{i\ge 0}a_iY^i)T^3+\varphi(\sum_{i\ge 0}b_iY^i)
$$
$$
0 = \sum_{i\ge 0}\varphi(a_i)T^{2i+3}+\sum_{i\ge 0}\varphi(b_i)T^{2i}
$$
$$
\Rightarrow \varphi(a_i)=\varphi(b_i)=0 \quad \forall i \ge 0
$$
\indent Como $K$ es cuerpo y $1\ne 0$ en $K[T]$.
$$
a_i = b_i = 0 \quad \forall i \ge 0
$$
$$
\Rightarrow r(X,Y) = 0 \Rightarrow f(X,Y)\in (X^2 - Y^3)
$$
\\
\indent \textit{b)}\\
\indent $\boxed{K[T^2,T^3] \supseteq \{\sum_{i\ge 0} c_i T^i \in K[T]\mid c_1 = 0\}}$\\
$$
\sum_{i\ge 0} c_i T^i = \underbrace{c_0}_{\in K[T^2,T^3]} + \underbrace{c_2T^2}_{\in K[T^2,T^3]} + \underbrace{c_3 T^3}_{\in K[T^2,T^3]}
$$
$$
+ \underbrace{c_4 T^4}_{= c_4 (T^2)^2\in K[T^2,T^3]} + \underbrace{c_5 T^5}_{= c_5 T^2T^3\in K[T^2,T^3]} \cdots \in K[T^2,T^3]
$$
\indent En general si $i$ es par $c_i T^i =c_i(T^2)^{i/2} \in \indent K[T^2,T^3]$, cuando $i\ge 3$ es impar $c_iT^i = \indent c_i T^3 (T^2)^{\frac{i-3}{2}} \in K[T^2,T^3]$.\\
\indent $\boxed{K[T^2,T^3] \subseteq \{\sum_{i\ge 0} c_i T^i \in K[T]\mid c_1 = 0\}}$\\
\indent Como $\{\sum_{i\ge 0} c_i T^i \in K[T]\mid c_1 = 0\}$ es subanillo \indent de $K[T]$ y $T^2, T^3 \in \{\sum_{i\ge 0} c_i T^i \in K[T]\mid c_1 = 0\}$. \indent Se tiene que,
$$
K[T^2,T^3] \subseteq \{\sum_{i\ge 0} c_i T^i \in K[T]\mid c_1 = 0\}
$$
\end{multicols}
\end{document}