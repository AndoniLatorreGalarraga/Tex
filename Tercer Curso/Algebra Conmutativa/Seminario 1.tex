\documentclass{article}
\usepackage[utf8]{inputenc}
\usepackage{graphicx}
\usepackage[spanish]{babel}
\usepackage{amssymb,amsmath,geometry,xcolor,multicol}
\usepackage{etoolbox} %titulo
\makeatletter %titulo
\patchcmd{\@maketitle}{\vskip 2em}{\vspace*{-3cm}}{}{} %titulo
\makeatother %titulo
\usepackage{vmargin}
\setpapersize{A4}
\setmargins{2.5cm}       % margen izquierdo
{1.5cm}                        % margen superior
{16.5cm}                      % anchura del texto
{23.42cm}                    % altura del texto
{10pt}                           % altura de los encabezados
{1cm}                           % espacio entre el texto y los encabezados
{0pt}                             % altura del pie de página
{2cm}                           % espacio entre el texto y el pie de página
\title{Seminario 1}
\author{Andoni Latorre Galarraga}
\date{}
\newcommand{\bb}[1]{\mathbb{#1}}
\newcommand{\p}{\textbf{Proposición: }}
\newcommand{\dem}{\textit{Dem: }}
\newcommand{\R}{\mathbb{R}}
\begin{document}

\maketitle

\begin{multicols}{2}

% EJERCICIO 1

\noindent
\textbf{1.}\\

% APARTADO B

\textit{(b)} Probaremos que $A$ es subanillo de $\R$.\\\\
\indent \textit{(b.i)} Veamos que $(A,+)$ es subgrupo de $\R,+$.\\
\indent \indent \textit{(b.i.i)} $0\in A$.
$$
0\in \mathbb{Z} \Rightarrow 0 = 0 + 0 \sqrt{2} \in A
$$
\indent \indent \textit{(b.i.ii)} $x,y\in A \Rightarrow x-y \in A$.\\\\
\indent \indent \indent Sean $x = a_1 + b_1 \sqrt{2}$, $y = a_2 + b_2 \sqrt{2}$.
$$
x-y = a_1 + b_1 \sqrt{2} - a_2 - b_2 \sqrt{2}
$$
$$
= (a_1-a_2) + (b_1-b_2) \sqrt{2} \in A
$$
$$
\because a_1-a_2 , b_1 - b_2 \in \mathbb{Z}
$$\\
\indent \textit{(b.ii)} $x,y\in A  \Rightarrow xy \in A$.\\
\indent \indent \indent Sean $x = a_1 + b_1 \sqrt{2}$, $y = a_2 + b_2 \sqrt{2}$.
$$
xy= (a_1 + b_1 \sqrt{2})(a_2 + b_2 \sqrt{2})
$$
$$
= a_1a_2+2b_1b_2 + (a_2b_1+a_1b_2) \sqrt{2} \in A
$$
$$
\because a_1a_2+2b_1b_2, a_2b_1+a_1b_2 \in \mathbb{Z}
$$\\
\indent \textit{(b.iii)} $1 \in A$
$$
1,0 \in \mathbb{Z} \Rightarrow 1=1+0\sqrt{2} \in A
$$\\
Tenemos que $A\supseteq \mathbb{Z}[\sqrt{2}]$ ya que $A$ es subanillo de $\R$ y $\{\sqrt{2}\} = \{0+1\sqrt{2}\} \subseteq A$. También tenemos que $A\subseteq \mathbb{Z}[\sqrt{2}]$ ya que dado $a+b\sqrt{2}\in A$,
$$
a+b\sqrt{2} = \underbrace{\underbrace{a\cdot 1}_{\in \mathbb{Z}[\sqrt{2}]} + \underbrace{b \sqrt{2}}_{\in \mathbb{Z}[\sqrt{2}]}}_{\in \mathbb{Z}[\sqrt{2}]}
$$
Tambien sabemos que $S = \{\sqrt{2}\}$ es mínimo ya que $\sqrt{2}\notin \mathbb{Z}[\emptyset]$.

% APARTADO D

\textit{(d)} Probaremos que $A$ es subanillo de $\R$.\\\\
\indent \textit{(d.i)} Veamos que $(A,+)$ es subgrupo de $(\R,+)$.\\
\indent \indent \textit{(d.i.i)} $0 \in A$
$$
0 \in \mathbb{Z} \Rightarrow 0 =\frac{0}{2^n3^m} \in A
$$
\indent \indent \textit{(d.i.ii)} $x,y\in A  \Rightarrow x-y \in A$.\\
\indent \indent \indent Sean $x=\frac{a_1}{2^{n_1}3^{m_1}}$, $y=\frac{a_2}{2^{n_2}3^{m_2}}$.
$$
\frac{a_1}{2^{n_1}3^{m_1}} - \frac{a_2}{2^{n_2}3^{m_2}}
$$
$$
= \frac{a_1 2^{n_2} 3^{m_2} - a_2 2^{n_1} 3^{m_1}}{2^{n_1+n_2}3^{m_1+m_2}} \in A
$$
$$
\because
\left.\begin{array}{l}
    a_1 2^{n_2} 3^{m_2} - a_2 2^{n_1} 3^{m_1} \in \mathbb{Z}\\
    n_1+n_2, m_1+m_2 \in \mathbb{Z}^+
\end{array}\right.
$$
\indent \textit{(d.ii)} $x,y\in A \Rightarrow xy\in A$.\\
\indent \indent \indent Sean $x=\frac{a_1}{2^{n_1}3^{m_1}}$, $y=\frac{a_2}{2^{n_2}3^{m_2}}$.
$$
xy= \frac{a_1}{2^{n_1}3^{m_1}} \frac{a_2}{2^{n_2}3^{m_2}}
$$
$$
= \frac{a_1 a_2}{2^{n_1+n_2}3^{m_1+m_2}} \in A
$$
$$
\because
\left.\begin{array}{l}
    a_1 a_2 \in \mathbb{Z}\\
    n_1+n_2, m_1+m_2 \in \mathbb{Z}^+
\end{array}\right.
$$
\indent\textit{(d.iii)} $1\in A$
$$
1 \in \mathbb{Z} , 0 \in \mathbb{Z}^+ \Rightarrow 1= \frac{1}{2^0 3^0} \in A
$$
\indent Tenemos que $A \supseteq \mathbb{Z}[\frac{1}{6}]$ ya que $A$ es subanillo de \indent $\R$ y $\{\frac{1}{6}\}= \{\frac{1}{2^1 3^1}\} \subseteq A$. Tambien tenemos que \indent $A \subseteq \mathbb{Z}[\frac{1}{6}]$ ya que dado $\frac{a}{2^n3^m}\in A$,
$$
\frac{a}{2^n 3^m} = 
\left\{\begin{array}{ll}
    a3^{n-m} \cdot \left( \frac{1}{6} \right) ^n & n\ge m\\
    a2^{m-n} \cdot \left( \frac{1}{6} \right) ^m & n\le m
\end{array}\right.
$$
\indent También sabemos que $S = \{\frac{1}{6}\}$ es mínimo ya que \indent $\frac{1}{6}\notin \mathbb{Z}[\emptyset]$.\\\\\\

%EJERCICIO 2

\noindent
\textbf{2.}\\\\
\indent Tenemos que
$$
\frac{a}{2^n3^m} \in \mathcal{U} (A)
$$
$$
\Leftrightarrow \exists a'\in \mathbb{Z} n',m'\in\mathbb{Z}^+ \mid \frac{2^n 3^m}{a} = \frac{a'}{2^{n'} 3^{m'}}
$$
\indent Ahora, $2^n 3^m2^{n'} 3^{m'}=aa'\in \mathbb{Z} \Rightarrow a = 2^s3^t \mid s,t\in \indent \mathbb{Z}$, es decir, $\mathfrak{U}=\{2^s3^t \mid s,t\in \mathbb{Z}\} \supseteq \mathcal{U}(A)$. Veamos \indent que $\mathfrak{U} \subseteq \mathcal{U}(A)$. Dado $2^s3^t\in \mathfrak{U}$, $\exists \frac{2^{s-|s|}2^{t-|t|}}{2^{s+|s|}2^{t+|t|}}\in A$ \indent ya que $2^{s-|s|}2^{t-|t|}\in \mathbb{Z}$ y $2^{s+|s|},2^{t+|t|}\in \mathbb{Z}^+$.\\\\\\

%EJERCICIO 5

\noindent
\textbf{5.}\\\\
\textit{(b)} $B$ no es anillo ya que $X\in B$ pero $X \cdot X = X^2 \notin B$.\\\\
\textit{(c)} $B$ Es anillo. Veamos que es subanillo de $A[X]$.
\indent \textit{(c.i)} Veamos que $(B,+)$ es subgrupo de $(A[x], +)$.\\
\indent \indent \textit{(c.i.i)} $0\in B$
$$
0 = 0 X^0
$$
\indent \indent \textit{(c.i.ii)} $p,q \in B \Rightarrow p-q \in B$.\\
\indent \indent \indent Sean $p = \sum_{i\ge0} a_{i,p}X^i$, $q = \sum_{i\ge0} a_{i,q}X^i$
$$
p-q = \sum_{i\ge0} (a_{i,p}-a_{i,q}) X^i
$$
\indent \indent \indent Cuando $i$ es impar
$$
a_{i,p},a_{i,q}=0 \Rightarrow a_{i,p}-a_{i,q}=0 \Rightarrow p-q \in B
$$
\indent \textit{(c.ii)} $p,q\in B \Rightarrow pq \in B$.\\
\indent \indent Sean $p = \sum_{j\ge0} a_{j,p}X^i$, $q = \sum_{k\ge0} a_{k,q}X^i$.\\
\indent \indent Si $pq= \sum_{i\ge0} a_iX^i$. Tenemos que
$$
a_i = \sum_{j,k \mid j+k = 1} a_{j,p} a_{k,q} \Rightarrow pq \in B
$$
$$
\because i \text{ impar} \Rightarrow
\left\{\begin{array}{l}
    k \text{ impar}\\
    \text{ó}\\
    j \text{ impar}
\end{array}\right.
\Rightarrow a_{j,p} a_{k,q} = 0
$$
\indent \textit{(c.iii)} $1\in B$.
$$
1 = 1 X^0
$$

%EJERCICIO 10

\noindent
\textbf{10.}

\indent \textit{(a)} Veamos que $\mathcal{U}(A\times B) \subseteq \mathcal{U}(A)\times\mathcal{U}(B)$. Sea\\
\indent $(u_a,u_b)\in \mathcal{U}(A\times B)$, entoneces $\exists (v_a,v_b)\in A\times B$\\
\indent tal que $(u_a,u_b)(v_a,v_b)=1_{A\times B}$.
$$
(u_a,u_b)(v_a,v_b)=(1_A,1_B) \Rightarrow
\left\{\begin{array}{l}
    u_av_a=1_A\\
    u_b,v_b=1_B
\end{array}\right.
$$
$$
\Rightarrow (u_a,u_b)\in \mathcal{U}(A)\times\mathcal{U}(B)
$$

\indent Veamos que $\mathcal{U}(A\times B) \supseteq \mathcal{U}(A)\times\mathcal{U}(B)$. Sean\\
\indent $(u_a,u_b)\in\mathcal{U}(A)\times\mathcal{U}(B)$, $u_av_a=1_A$, $u_bv_b=1_B$.
$$
(u_a,u_b)\cdot(v_a,v_b)= (1_A, 1_B) = 1_{A\times B}
$$
$$
\Rightarrow (u_a,u_b)\in\mathcal{U}(a\times B)
$$

\indent \textit{(b)} Sean $0\ne x=(x_a,x_b)$, $0\ne y=(y_a,y_b)$\\
\indent distintos de $0_{A\times B}$ tales que $xy=0$.
$$
xy=(x_a,x_b)(y_a,y_b)=(x_ay_a,x_by_b)=(0_A,0_B)
$$
$$
\Leftrightarrow x_ay_a=0_A, a_by_b=0_B
$$
\indent Es decir, los divisoresde cero en $A\times B$ son de la \indent forma $(a,b)$ con $a$ y $b$ divisores de cero en $A$ y $B$ \indent respectivamente.\\

\indent \textit{(c)} Es suficiente con encontar un elemento no \indent nulo de $A\times B$ que no sea inversible.
$$
\underbrace{(1_A,0_B)}_{\ne 0_{A\times B}}\underbrace{(a,b)}_{\in A\times B} = (a,0_B) \ne 1_{A\times B}
$$\\

\noindent
\textbf{17.}\\
\indent Queremos encontrar el mínimo $n\in\mathbb{N}$ tal que \indent $1_{A\times B}^n = 0_{A\times B}$.
$$
\left.\begin{array}{l}
    1_{A\times B}^n = (1_A^n, 1_B^n)\\
    0_{A\times B} = (0_A, 0_B)
\end{array}\right\} \Rightarrow (1_A^n, 1_B^n) = (0_A, 0_B)
$$
$$
\Rightarrow
\left\{\begin{array}{l}
    \text{char} (A)\mid n\\
    \text{char} (B)\mid n
\end{array}\right. \Rightarrow
\text{mcm}(\text{char} (A), \text{char} (B))\mid n
$$
\indent Además,
$$
1_{A\times B}^{\text{mcm}(\text{char} (A), \text{char} (B))}
$$
$$
= (1_A^{\text{mcm}(\text{char} (A), \text{char} (B)))}, 1_B^{\text{mcm}(\text{char} (A), \text{char} (B))})
$$
$$
= (0_A, 0_B) = 0_{A\times B}
$$
$$
\Rightarrow n \mid \text{mcm}(\text{char} (A), \text{char} (B))
$$
\indent Se tiene que $n = \text{mcm}(\text{char} (A), \text{char} (B))$
\end{multicols}

\end{document}