\documentclass{article}
\usepackage[utf8]{inputenc}
\usepackage{graphicx}
\usepackage[spanish]{babel}
\usepackage{amssymb,amsmath,geometry,multicol,spalign,hyperref}
\usepackage[usenames,dvipsnames]{xcolor}
\usepackage{tikz,mathtools}
\usepackage{pgfplots}
\pgfplotsset{every axis/.append style={
                    axis x line=middle,    % put the x axis in the middle
                    axis y line=middle,    % put the y axis in the middle
                    axis line style={<->,color=blue}, % arrows on the axis
                    xlabel={$x$},          % default put x on x-axis
                    ylabel={$y$},          % default put y on y-axis
            }}
\usepackage{etoolbox} %titulo
\makeatletter %titulo
\patchcmd{\@maketitle}{\vskip 2em}{\vspace*{-3cm}}{}{} %titulo
\makeatother %titulo
\usepackage{vmargin}
\setpapersize{A4}
\setmargins{2.5cm}       % margen izquierdo
{1.5cm}                        % margen superior
{16.5cm}                      % anchura del texto
{23.42cm}                    % altura del texto
{10pt}                           % altura de los encabezados
{1cm}                           % espacio entre el texto y los encabezados
{0pt}                             % altura del pie de página
{2cm}                           % espacio entre el texto y el pie de página
\title{Seminario Un Paso}
\author{Andoni Latorre y Mariana Zaballa}
\date{}
\newcommand{\bb}[1]{\mathbb{#1}}
\newcommand{\R}{\bb{R}}
\newcommand{\nota}[3][2ex]{
    \underset{\mathclap{
        \begin{tikzpicture}
          \draw[->] (0, 0) to ++(0,#1);
          \node[below] at (0,0) {#3};
        \end{tikzpicture}}}{#2}
}
\usepackage{matlab-prettifier}
\begin{document}

\maketitle
Calculemos $\frac{y_{n+1}}{y_n}$ para luego ver cuando está acotado.
$$
y_{n+1} - y_n = \frac{h}{2}(f_n+f_{n+1}) + \frac{h^2}{12}(f_n^{(1)}-f_{n+1}^{(1)})
$$
Sustituimos $f^{(k-1)}_n = \lambda^k y_n $
$$
y_{n+1}-y_n = \frac{h}{2}(\lambda y_n+\lambda y_{n+1}) + \frac{h^2}{12}(\lambda^2y_n-\lambda^2 y_{n+1})
$$
$$
y_{n+1}(1 - \frac{\lambda h}{2} + \frac{\lambda^2 h^2}{12}) = y_n (1 + \frac{\lambda h}{2} + \frac{\lambda^2 h^2}{12})
$$
$$
\frac{y_{n+1}}{y_n} = \frac{1 + \frac{\lambda h}{2} + \frac{\lambda^2 h^2}{12}}{1 - \frac{\lambda h}{2} + \frac{\lambda^2 h^2}{12}}
$$
Sustituimos $\lambda h = \bar{h}$
$$
\frac{y_{n+1}}{y_n} = \frac{1 + \frac{\bar{h}}{2} + \frac{\bar{h}^2}{12}}{1 - \frac{\bar{h}}{2} + \frac{\bar{h}^2}{12}}
$$
\end{document}