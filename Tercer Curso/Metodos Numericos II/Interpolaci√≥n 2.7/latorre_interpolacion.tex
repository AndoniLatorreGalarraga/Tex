\documentclass{article}
\usepackage[utf8]{inputenc}
\usepackage{graphicx}
\usepackage[spanish]{babel}
\usepackage{amssymb,amsmath,geometry,multicol,spalign,hyperref}
\usepackage[usenames,dvipsnames]{xcolor}
\usepackage{tikz,mathtools}
\usepackage{pgfplots}
\pgfplotsset{every axis/.append style={
                    axis x line=middle,    % put the x axis in the middle
                    axis y line=middle,    % put the y axis in the middle
                    axis line style={<->,color=blue}, % arrows on the axis
                    xlabel={$x$},          % default put x on x-axis
                    ylabel={$y$},          % default put y on y-axis
            }}
\usepackage{etoolbox} %titulo
\makeatletter %titulo
\patchcmd{\@maketitle}{\vskip 2em}{\vspace*{-3cm}}{}{} %titulo
\makeatother %titulo
\usepackage{vmargin}
\setpapersize{A4}
\setmargins{2.5cm}       % margen izquierdo
{1.5cm}                        % margen superior
{16.5cm}                      % anchura del texto
{23.42cm}                    % altura del texto
{10pt}                           % altura de los encabezados
{1cm}                           % espacio entre el texto y los encabezados
{0pt}                             % altura del pie de página
{2cm}                           % espacio entre el texto y el pie de página
\title{Ejemplo 2.7}
\author{Andoni Latorre Galarraga}
\date{}
\newcommand{\bb}[1]{\mathbb{#1}}
\newcommand{\R}{\bb{R}}
\newcommand{\nota}[3][2ex]{
    \underset{\mathclap{
        \begin{tikzpicture}
          \draw[->] (0, 0) to ++(0,#1);
          \node[below] at (0,0) {#3};
        \end{tikzpicture}}}{#2}
}
\usepackage{matlab-prettifier}
\begin{document}

\maketitle

\section{Método de interpolación}

Construimos un polinomio de interpolación $\textcolor{blue}{p(x)= a_5 x^5 + a_4 x^4 + a_3 x^3 + a_2 x^2 + a_1 x + a_0}$ que verifique las condiciones $\textcolor{blue}{p(-1)=f(-1), p(0)=f(0), p(1)=f(1), p''(-1)=f''(-1), p''(0)=f''(0), p''(1)=f''(1)}$. Planteamos el siguiente sistema lineal para calcular los coeficientes $\textcolor{blue}{a_i}$:
$$
\left(\begin{array}{cccccc}
     -1 &  1 & -1 & 1 & -1 & 1 \\
      0 &  0 &  0 & 0 &  0 & 1 \\
      1 &  1 &  1 & 1 &  1 & 1 \\
    -20 & 12 & -6 & 2 &  0 & 0 \\
      0 &  0 &  0 & 2 &  0 & 0 \\
     20 & 12 &  6 & 2 &  0 & 0
\end{array}\right)
\left(\begin{array}{c}
    a_5 \\
    a_4 \\
    a_3 \\
    a_2 \\
    a_1 \\
    a_0
\end{array}\right)
=
\left(\begin{array}{c}
    f(-1) \\
    f(0) \\
    f(1) \\
    f''(-1) \\
    f''(0) \\
    f''(1)
\end{array}\right) 
$$
Tras resolver el sistema, se tiene la fórmula de cuadratura
$$
\textcolor{red}{\mathcal{I}_5}= \int_{-1}^1 p(x) dx = \sum_{k=0}^5 \frac{a_k}{k+1} (1^{k+1}-(-1)^{k+1}) = \textcolor{red}{2 a_0 + \frac{2}{3} a_2 + \frac{2}{5} a_4}
$$

\section{Método directo}

\noindent Plantemos el sistema,
$$
\begin{array}{r|l}
    f(1) & \alpha_1 \\
    f(0) & \alpha_0 \\
    f(-1) & \alpha_{-1}\\
    f''(1) & \beta_1 \\
    f''(0) & \beta_0 \\
    f''(-1) & \beta_{-1}
\end{array}
\quad  \quad
\spalignsys{
\alpha_1 + \alpha_0 + \alpha_{-1} \: \: \: \: \: \: = 2;
\alpha_1 \: \: - \alpha_{-1} \: \: \: \: \: \: = 0;
\alpha_1 \: \: + \alpha_{-1} + 2\beta_1 + 2\beta_0 + 2\beta_{-1} = \frac{2}{3};
\alpha_1 \: \: - \alpha_{-1} + 6\beta_1 \: \: - 6\beta_{-1} = 0;
\alpha_1 \: \: + \alpha_{-1} + 12\beta_1 \: \: + 12\beta_{-1} = \frac{2}{5};
\alpha_1 \: \: - \alpha_{-1} + 20\beta_1 \: \: - 20\beta_{-1} = 0
}
$$
Como este sistema no tiene solución única, podemos añadir una condición extra
$$
\begin{array}{r|l}
    f(1) & \alpha_1 \\
    f(0) & \alpha_0 \\
    f(-1) & \alpha_{-1}\\
    f''(1) & \beta_1 \\
    f''(0) & \beta_0 \\
    f''(-1) & \beta_{-1}
\end{array}
\quad  \quad
\spalignsys{
\alpha_1 + \alpha_0 + \alpha_{-1} \: \: \: \: \: \: = 2;
\alpha_1 \: \: - \alpha_{-1} \: \: \: \: \: \: = 0;
\alpha_1 \: \: + \alpha_{-1} + 2\beta_1 + 2\beta_0 + 2\beta_{-1} = \frac{2}{3};
\alpha_1 \: \: - \alpha_{-1} + 6\beta_1 \: \: - 6\beta_{-1} = 0;
\alpha_1 \: \: + \alpha_{-1} + 12\beta_1 \: \: + 12\beta_{-1} = \frac{2}{5};
\alpha_1 \: \: - \alpha_{-1} + 20\beta_1 \: \: - 20\beta_{-1} = 0;
\alpha_1 \: \: + \alpha_{-1} + 30\beta_1 \: \: + 30\beta_{-1} = \frac{2}{7}
}
$$
El nuevo sistema tiene solución única
\begin{multicols}{2}
\begin{lstlisting}[style=Matlab-editor]
a = [1 1 1 0 0 0;
    1 0 -1 0 0 0;
    1 0 1 2 2 2;
    1 0 -1 6 0 -6;
    1 0 1 12 0 12;
    1 0 -1 20 0 -20;
    1 0 1 30 0 30]
b = [2; 0; 2/3; 0; 2/5; 0; 2/7]
a\b

ans =

0.2381
1.5238
0.2381
-0.0032
0.1016
-0.0032
    \end{lstlisting}
\end{multicols}
\noindent Tenemos la fórmula de cuadratura
$$
\textcolor{red}{\mathcal{I}_6 = 0.2381 f(1) + 1.5238 f(0) + 0.2381 f(-1) - 0.0032 f''(1) + 0.1016 f''(0) - 0.0032 f''(-1) }
$$
\end{document}