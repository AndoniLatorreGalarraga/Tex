\documentclass{article}
\usepackage[utf8]{inputenc}
\usepackage{graphicx}
\usepackage[spanish]{babel}
\usepackage{amssymb,amsmath,geometry,multicol,spalign,hyperref}
\usepackage[usenames,dvipsnames]{xcolor}
\usepackage{tikz,mathtools}
\usepackage{pgfplots}
\pgfplotsset{every axis/.append style={
                    axis x line=middle,    % put the x axis in the middle
                    axis y line=middle,    % put the y axis in the middle
                    axis line style={<->,color=blue}, % arrows on the axis
                    xlabel={$x$},          % default put x on x-axis
                    ylabel={$y$},          % default put y on y-axis
            }}
\usepackage{etoolbox} %titulo
\makeatletter %titulo
\patchcmd{\@maketitle}{\vskip 2em}{\vspace*{-3cm}}{}{} %titulo
\makeatother %titulo
\usepackage{vmargin}
\setpapersize{A4}
\setmargins{2.5cm}       % margen izquierdo
{1.5cm}                        % margen superior
{16.5cm}                      % anchura del texto
{23.42cm}                    % altura del texto
{10pt}                           % altura de los encabezados
{1cm}                           % espacio entre el texto y los encabezados
{0pt}                             % altura del pie de página
{2cm}                           % espacio entre el texto y el pie de página
\title{Interpolación trigonométrica}
\author{\textcolor{WildStrawberry}{Yeray Alvarez, Andoni Latorre}}
\date{}
\newcommand{\bb}[1]{\mathbb{#1}}
\newcommand{\R}{\bb{R}}
\newcommand{\nota}[3][2ex]{
    \underset{\mathclap{
        \begin{tikzpicture}
          \draw[->] (0, 0) to ++(0,#1);
          \node[below] at (0,0) {#3};
        \end{tikzpicture}}}{#2}
}
\begin{document}

\maketitle

\noindent En primer lugar, vamos a encontrar una fórmula cerrada para la suma de senos y cosenos. Para ello, vamos a considerar la siguiente suma:

$$
\textcolor{red}{\sum_{j=0}^n \cos(jx)} + i \textcolor{blue}{\sum_{j=0}^n \sen(jx)} = \sum_{j=0}^n \cos(jx) + i \sen(jx) \nota{=}{\text{De Moivre}}
$$
\begin{equation}
=\sum_{j=0}^n (\cos(x) + i \sen(x))^j \nota{=}{\text{Euler}} \sum_{j=0}^n (e^{ix})^j = \frac{1-e^{i(n+1)x}}{1-e^{ix}}
\end{equation}

\noindent Ahora bien, podemos escribir la siguiente igualdad:

$$
\sen(x) = \frac{e^{ix}-e^{-ix}}{2i} = \frac{e^{2ix}-1}{2ie^{ix}} \Rightarrow  1-e^{2ix} = -2ie^{ix}\sen(x)
$$

\noindent De donde se deduce, sustituyendo $x$ por $\frac{x}{2}$ y $x$ por $\frac{n+1}{2}x$, las igualdades:

$$
\begin{array}{rl}
i) & \displaystyle{1-e^{ix} = -2ie^{i\frac{x}{2}}\sen(\frac{x}{2})} \\
ii) & \displaystyle{1-e^{i(n+1)x} = -2ie^{i\frac{n+1}{2}x}\sen(\frac{n+1}{2}x)}
\end{array}
$$

\noindent Ahora llevamos estos resultados a $(1)$ y obtenemos:

$$
= \frac{1-e^{i(n+1)x}}{1-e^{ix}} = (1) 
= \frac{-2ie^{i\frac{n+1}{2}x}\sen(\frac{n+1}{2}x)}{-2ie^{i\frac{x}{2}}\sen(\frac{x}{2})}
= e^{i\frac{n}{2}x} \frac{\sen(\frac{n+1}{2}x)}{\sen(\frac{x}{2})}
$$
$$
= \textcolor{red}{\cos(\frac{n}{2} x ) \frac{\sen(\frac{n+1}{2}x)}{\sen(\frac{x}{2})}} + i \textcolor{blue}{\sen(\frac{n}{2} x ) \frac{\sen(\frac{n+1}{2}x)}{\sen(\frac{x}{2})}}
$$

\noindent Con este resultado en mente, podemos probar la ortogonalidad del sistema trigonométrico.
Sustituyendo en la igualdad anterior $n$ por $2N-1$ y $x$ por $(p \pm q) \frac{\pi}{N}$ y obtenemos la siguiente expresión:

$$
\sum_{j=0}^{2N-1} \cos px_j \cos qx_j = 
\frac{1}{2}\sum_{j=0}^{2N-1} \cos{(p-q)x_j} + \frac{1}{2}\sum_{j=0}^{2N-1} \cos{(p+q)x_j} =
$$
$$
=\frac{1}{2}\sum_{j=0}^{2N-1} \cos \left((p-q)\frac{\pi}{N}j\right) +\frac{1}{2} \sum_{j=0}^{2N-1} \cos\left((p+q)\frac{\pi}{N}j\right) =
$$
$$
=\frac{1}{2} \cos\left(\frac{2N-1}{2N}(p-q)\pi\right) \frac{\textcolor{WildStrawberry}{\sen\left((p-q)\pi\right)}}{\sen((p-q)\frac{\pi}{2N})} + \frac{1}{2}
\cos\left(\frac{2N-1}{2N}(p+q)\pi\right) \frac{\textcolor{WildStrawberry}{\sen\left((p+q)\pi\right)}}{\sen((p+q)\frac{\pi}{2N})}
\nota{=}{$p\ne q$} 0
$$
\noindent Cuando $p=q$,
$$
\sum_{j=0}^{2N-1} \cos\left(p\frac{\pi}{N}j\right) \cos\left(p\frac{\pi}{N}j\right) = \sum_{j=0}^{2N-1} \cos^2\left(p\frac{\pi}{N}j\right) =
\frac{1}{2} \sum_{j=0}^{2N-1} 1 + \cos\left(\frac{p}{n}2\pi pj\right) =
$$
$$
=\frac{1}{2} \sum_{j=0}^{2N-1} 1 + \frac{1}{2} \sum_{j=0}^{2N-1} \cos \left( \frac{p}{N} 2\pi j \right)
\nota{=}{$p \ne 0,N$}
N + \frac{1}{2}\left( \cos\left( \frac{2N-1}{2N} 2\pi p \right) \frac{\sen \left(2p \pi \right)}{\sen\left(\frac{p}{N}\pi \right)} \right)
= N
$$

\noindent Cuando $p=q=0,N$,


$$
=\frac{1}{2} \sum_{j=0}^{2N-1} 1 + \frac{1}{2} \sum_{j=0}^{2N-1} \cos \left( \frac{p}{N} 2\pi j \right)
\nota{=}{$p = 0,N$}
N + N = 2N
$$

\noindent La suma finita de productos de senos es bastante similar a la de los cosenos:

$$
\sum_{j=0}^{2N-1} \sen px_j \sen qx_j = 
\frac{1}{2}\sum_{j=0}^{2N-1} \cos{\left((p-q)x_j\right)} - \frac{1}{2}\sum_{j=0}^{2N-1} \cos{\left((p+q)x_j\right)} =
$$
$$
=\frac{1}{2}\sum_{j=0}^{2N-1} \cos \left((p-q)\frac{\pi}{N}j\right) - \frac{1}{2} \sum_{j=0}^{2N-1} \cos\left((p+q)\frac{\pi}{N}j\right) =
$$
$$
=\frac{1}{2} \cos\left(\frac{2N-1}{2N}(p-q)\pi\right) \frac{\textcolor{WildStrawberry}{\sen\left((p-q)\pi\right)}}{\sen((p-q)\frac{\pi}{2N})} + \frac{1}{2}
\cos\left(\frac{2N-1}{2N}(p+q)\pi\right) \frac{\textcolor{red}{\sen\left((p+q)\pi\right)}}{\sen((p+q)\frac{\pi}{2N})}
\nota{=}{$p\ne q$} 0
$$

\noindent Cuando $p=q$,
$$
\sum_{j=0}^{2N-1} \sen\left(p\frac{\pi}{N}j\right) \sen\left(p\frac{\pi}{N}j\right) = \sum_{j=0}^{2N-1} \sen^2\left(p\frac{\pi}{N}j\right) =
\frac{1}{2} \sum_{j=0}^{2N-1} 1 - \cos\left(2p\frac{\pi}{N}j\right) =
$$
$$
=\frac{1}{2} \sum_{j=0}^{2N-1} 1 - \frac{1}{2} \sum_{j=0}^{2N-1} \cos \left( \frac{p}{N} 2\pi j \right)
\nota{=}{$p \ne 0,N$}
N - \frac{1}{2}\left( \cos\left( \frac{2N-1}{2N} 2\pi p \right) \frac{\sen \left(2p \pi \right)}{\sen\left(\frac{p}{N}\pi \right)} \right)
= N
$$

\noindent Finalmente, veremos que la suma de productos de senos y cosenos es nula:

$$
\sum_{j=0}^{2N-1} \cos px_j \sen qx_j = 
\frac{1}{2}\sum_{j=0}^{2N-1} \sen{(p-q)x_j} + \frac{1}{2}\sum_{j=0}^{2N-1} \sen{(p+q)x_j} =
$$

$$
=\frac{1}{2}\sum_{j=0}^{2N-1} \sen \left((p-q)\frac{\pi}{N}j\right) +\frac{1}{2} \sum_{j=0}^{2N-1} \sen\left((p+q)\frac{\pi}{N}j\right) =
$$
$$
=\frac{1}{2} \sen\left(\frac{2N-1}{2N}(p-q)\pi\right) \frac{\textcolor{WildStrawberry}{\sen\left((p-q)\pi\right)}}{\sen((p-q)\frac{\pi}{2N})} + \frac{1}{2} \sen\left(\frac{2N-1}{2N}(p+q)\pi\right) \frac{\textcolor{WildStrawberry}{\sen\left((p+q)\pi\right)}}{\sen((p+q)\frac{\pi}{2N})} = 0
$$

\noindent Utilizando estas relaciones de ortogonalidad, demuestre que existe un único polinomio trigonométrico que interpola la función f por los nodos (1) con los coeficientes:

$$
\textcolor{red}{A_k = \frac{1}{N} \sum_{j=0}^{2N-1} f(x_j) \cos k x_j \quad \quad k = 0,\hdots,N}
$$
$$
\textcolor{blue}{B_k = \frac{1}{N} \sum_{j=0}^{2N-1} f(x_j) \sen k x_j \quad \quad k = 1,\hdots,N-1}
$$

$$
f(x) = \frac{A_0}{2} + \sum_{k=1}^{N-1} (A_k \cos k x + B_k \sen k x) + \frac{A_N}{2} \cos N x
$$

\noindent Veamos primero el valor del coeficiente $A_0$. Evaluamos la función interpoladora en $x_j$.
$$
\frac{A_0}{2} + \sum_{k=1}^{N-1} (A_k \cos k x_j + B_k \sen k x_j) + \frac{A_N}{2} \cos N x_j = f(x_j)
$$
Ahora, sumamos las expresiones para cada valor de $j$.
$$
\sum_{j=0}^{2N-1} \frac{A_0}{2} + \sum_{k=1}^{N-1} \sum_{j=0}^{2N-1} (A_k \cos k x_j + B_k \sen k x_j) + \sum_{j=0}^{2N-1} \frac{A_N}{2} \cos N x_j = \sum_{j=0}^{2N-1} f(x_j)
$$
$$
2N \frac{A_0}{2} + \sum_{k=1}^{N-1} \left ( A_k \sum_{j=0}^{2N-1} \cos k x_j + B_k \sum_{j=0}^{2N-1} \sen k x_j \right ) + \frac{A_N}{2} \sum_{j=0}^{2N-1} \cos N x_j = \sum_{j=0}^{2N-1} f(x_j)
$$
Aplicando las relaciones de ortogonalidad vemos que las sumas marcadas valen $\textcolor{WildStrawberry}{0}$.
$$
2N \frac{A_0}{2} + \sum_{k=1}^{N-1} \left ( A_k \textcolor{WildStrawberry}{\sum_{j=0}^{2N-1} \cos k x_j \cos 0 x_j} + B_k \textcolor{WildStrawberry}{\sum_{j=0}^{2N-1} \sen k x_j \cos 0 x_j} \right ) + \frac{A_N}{2} \textcolor{WildStrawberry}{\sum_{j=0}^{2N-1} \cos N x_j \cos 0 x_j} = \sum_{j=0}^{2N-1} f(x_j)
$$
De donde finalmente se deduce:
$$
N A_0 = \sum_{j=0}^{2N-1} f(x_j) \Rightarrow \textcolor{blue}{A_0 = \frac{1}{N} \sum_{j=0}^{2N-1} f(x_j)}
$$

\noindent Para hallar el valor de $A_N$, partimos de la igualdad clásica multiplicada por $\cos N x_j$ y sumamos en j:
$$
\frac{A_0}{2} + \sum_{k=1}^{N-1} (A_k \cos k x_j + B_k \sin k x_j) + \frac{A_N}{2}\cos N x_j = f(x_j)
$$
$$
\frac{A_0}{2}\cos N x_j + \sum_{k=1}^{N-1} (A_k \cos k x_j \cos N x_j + B_k \sin k x_j \cos N x_j) + \frac{A_N}{2}\cos N x_j \cos N x_j = f(x_j)\cos N x_j
$$
$$
\frac{A_0}{2} \textcolor{WildStrawberry}{\sum_{j=0}^{2N-1} \cos N x_j \cos 0 x_j} + \sum_{k=1}^{N-1} \textcolor{WildStrawberry}{\sum_{j=0}^{2N-1}(A_k \cos k x_j \cos N x_j + B_k \sin k x_j \cos N x_j)} + \frac{A_N}{2} \textcolor{WildStrawberry}{\sum_{j=0}^{2N-1} \cos N x_j \cos N x_j}
$$
$$
=\sum_{j=0}^{2N-1} f(x_j)\cos N x_j
$$
De donde finalmente deducimos:
$$
2N \frac{A_N}{2}  = \sum_{j=0}^{2N-1} f(x_j)\cos N x_j \Rightarrow \textcolor{red}{A_N = \frac{1}{N} \sum_{j=0}^{2N-1} f(x_j)\cos N x_j}
$$

\noindent Para hallar los coeficientes $A_m$ con $m\in \{1,\hdots,N-1\}$ tomamos la igualdad:
$$
\frac{A_0}{2} + \sum_{k=1}^{N-1} (A_k \cos k x_j + B_k \sen k x_j) + \frac{A_N}{2} \cos N x_j = f(x_j)
$$
La multiplicamos por $\cos(m x_j)$:
$$
\frac{A_0}{2}\cos m x_j + \sum_{k=1}^{N-1} (A_k \cos k x_j \cos m x_j + B_k \sin k x_j \cos m x_j) + \frac{A_N}{2}\cos N x_j \cos m x_j = f(x_j)\cos m x_j
$$
Y sumamos en j:
$$
\sum_{j=0}^{2N-1} \frac{A_0}{2}\cos m x_j + \sum_{k=1}^{N-1} \sum_{j=0}^{2N-1}(A_k \cos k x_j \cos m x_j + B_k \sin k x_j \cos m x_j) + \sum_{j=0}^{2N-1} \frac{A_N}{2}\cos N x_j \cos m x_j
$$
$$
= \sum_{j=0}^{2N-1} f(x_j)\cos m x_j\
$$
Aplicando las relaciones de ortogonalidad obtenemos los valores de las sumas marcadas:
$$
\frac{A_0}{2} \textcolor{WildStrawberry}{ \sum_{j=0}^{2N-1} \cos m x_j \cos 0 x_j } + \sum_{k=1}^{N-1} \textcolor{WildStrawberry}{ \sum_{j=0}^{2N-1}(A_k \cos k x_j \cos m x_j + B_k \sin k x_j \cos m x_j ) } + \frac{A_N}{2} \textcolor{WildStrawberry}{ \sum_{j=0}^{2N-1}\cos N x_j \cos m x_j }
$$
$$
= \sum_{j=0}^{2N-1} f(x_j)\cos m x_j
$$
Del sumatorio en j del producto de $\textcolor{WildStrawberry}{A_k \cos k x_j \cos m x_j}$ sobrevive el término que corresponde a $j=m$:
$$
N A_m  = \sum_{j=0}^{2N-1} f(x_j)\cos m x_j \Rightarrow \textcolor{blue}{A_m = \frac{1}{N} \sum_{j=0}^{2N-1} f(x_j)\cos m x_j}
$$

\noindent Mutatis mutandis, para hallar los coeficientes $B_m$ con $m\in \{1,\hdots,N-1\}$ seguiremos la misma serie de pasos multiplicando esta vez las respectivas igualdades por $\sen m x_j$ y luego sumando en j:

$$
\frac{A_0}{2} + \sum_{k=1}^{N-1} (A_k \cos k x_j + B_k \sen k x_j) + \frac{A_N}{2} \cos N x_j = f(x_j)
$$
$$
\frac{A_0}{2}\cos m x_j + \sum_{k=1}^{N-1} (A_k \cos k x_j \sen m x_j + B_k \sin k x_j \sen m x_j) + \frac{A_N}{2}\cos N x_j \sen m x_j = f(x_j)\sen m x_j
$$
$$
\sum_{j=0}^{2N-1} \frac{A_0}{2}\sen m x_j + \sum_{k=1}^{N-1} \sum_{j=0}^{2N-1}(A_k \cos k x_j \sen m x_j + B_k \sen k x_j \sen m x_j) + \sum_{j=0}^{2N-1} \frac{A_N}{2}\cos N x_j \sen m x_j
$$
$$
= \sum_{j=0}^{2N-1} f(x_j)\sen m x_j\
$$
$$
\frac{A_0}{2} \textcolor{WildStrawberry}{ \sum_{j=0}^{2N-1} \sen m x_j \cos 0 x_j } + \sum_{k=1}^{N-1} \textcolor{WildStrawberry}{ \sum_{j=0}^{2N-1}(A_k \cos k x_j \sen m x_j + B_k \sin k x_j \sen m x_j ) } + \frac{A_N}{2} \textcolor{WildStrawberry}{ \sum_{j=0}^{2N-1}\cos N x_j \sen m x_j }
$$
$$
= \sum_{j=0}^{2N-1} f(x_j)\sen m x_j
$$
Análogamente, del sumatorio en j del producto de $\textcolor{WildStrawberry}{B_k \sen k x_j \sen m x_j}$ sobrevive el término que corresponde a $j=m$:
$$
N B_m  = \sum_{j=0}^{2N-1} f(x_j)\sen m x_j \Rightarrow \textcolor{red}{B_m = \frac{1}{N} \sum_{j=0}^{2N-1} f(x_j)\sen m x_j}
$$



\noindent (iii) Suponga ahora que la función f tiene un desarrollo de Fourier
$$
f(x) = \frac{a_0}{2} + \sum_{k=1}^{\infty} (a_k \cos kx + b_k \sen kx)
$$
que converge puntualmente en $[0, 2\pi]$. ¿Están relacionados los coeficientes $ak, Ak, bk, Bk$ ?

\noindent Es bien conocido que los coeficientes de Fourier para una función de clase C1 a trozos son de la siguiente forma:

$$
a_k =  \frac{1}{\pi} \int_0^{2\pi}f(x)\cos kxdx \quad \sim \quad A_k = \frac{1}{N} \sum_{j=0}^{2N-1} f(x_j) \cos kx_j
$$
$$
b_k =  \frac{1}{\pi} \int_0^{2\pi}f(x)\sen kxdx \quad \sim \quad B_k = \frac{1}{N} \sum_{j=0}^{2N-1} f(x_j) \sen kx_j
$$

\noindent Si escribimos 

$$
A_k = \frac{1}{N} \sum_{j=0}^{2N-1} f(x_j) \cos kx_j = \frac{1}{\pi} \frac{2\pi}{2N} \sum_{j=0}^{2N-1} f(x_j) \cos kx_j = \frac{1}{\pi} \sigma(f(x)\cos kx, [0,2\pi], \pi/N)
$$
\noindent Que se trata de una suma de Riemann (1826-1866) en el intervalo $[0,2\pi]$ con diámetro de partición $\frac{\pi}{N}$ para la función $g(x)=f(x)\cos(kx)$, y tomando limites cuando $N\to\infty$ queda
$$
\frac{1}{\pi}\int_0^{2\pi} f(x) \cos kx dx
$$

\noindent Siguiendo un razonamiento similar podemos probar que $B_k$ es una suma de Riemann (1826-1866) en el intervalo $[0,2\pi]$ con diámetro de partición $\frac{\pi}{N}$ para la función $g(x)=f(x)\sen(kx)$, y tomando limites una vez más obtenemos el resultado deseado:
$$
B_k = \frac{1}{N} \sum_{j=0}^{2N-1} f(x_j) \sen kx_j = \frac{1}{\pi} \frac{2\pi}{2N} \sum_{j=0}^{2N-1} f(x_j) \sen kx_j = \frac{1}{\pi}\int_0^{2\pi} f(x) \sen kx dx
$$

\end{document}