\documentclass{article}
\usepackage[utf8]{inputenc}
\usepackage{graphicx}
\usepackage[spanish]{babel}
\usepackage{amssymb,amsmath,geometry,xcolor,multicol}
\usepackage{etoolbox} %titulo
\makeatletter %titulo
\patchcmd{\@maketitle}{\vskip 2em}{\vspace*{-3cm}}{}{} %titulo
\makeatother %titulo
\usepackage{vmargin}
\setpapersize{A4}
\setmargins{2.5cm}       % margen izquierdo
{1.5cm}                        % margen superior
{16.5cm}                      % anchura del texto
{23.42cm}                    % altura del texto
{10pt}                           % altura de los encabezados
{1cm}                           % espacio entre el texto y los encabezados
{0pt}                             % altura del pie de página
{2cm}                           % espacio entre el texto y el pie de página
\title{GEOMETRIA GLOBAL}
\author{Andoni Latorre Galarraga}
\date{18 de Febrero}
\newcommand{\bb}[1]{\mathbb{#1}}
\newcommand{\p}{\textbf{Proposición: }}
\newcommand{\dem}{\textit{Dem: }}
\newcommand{\R}{\mathbb{R}}
\begin{document}

\maketitle
Def: $\mathcal{S}\subset \bb{R}^3$, decimos que es superficie regular si $\forall p \in \mathcal{S} \exists V$ abierto (con la topología relativa) y $\exists \mathcal{U} \text{(abierto y conexo)} \subset \bb{R}^2$ y $\varphi : \mathcal{U} \longrightarrow V\in \bb{R}^3 $ tal que.\\
i) $\varphi$ homeomorfismo, $\exists (\varphi)^{-1}$ continua \\
ii) $\varphi$ diferenciable \\
iii) $D\varphi$ inyectiva (no confundir diferencial con matriz jacobiana)\\
(por el teorema de la función inversa $\exists \tilde{V}\subset V \text{t.q} \varphi^{-1} : \tilde{V}\longmapsto \tilde{\mathcal{U}}$ diferenciable)
\\
\\
$\begin{array}{crcl}
\alpha : & \bb{R} & \longrightarrow & \bb{R}^2 \\
& t & \longmapsto     & (x(t),y(t))
\end{array}$\\
Cilindro:\\
$\begin{array}{crcl}
\varphi : & \bb{R}^2 & \longrightarrow & \bb{R}^3 \\
& (t,z) & \longmapsto     & (x(t),y(t),z)
\end{array}$\\\\
Def: Sea $\mathcal{S}\subset \bb{R}^3$ superficie $\mathcal{A}= \{(u_i, \varphi_i) \mid i\in I\}$ es un atlas local.\\
i) $\forall p \in \mathcal{S} \exists V \in \mathcal{N}_\text{abierto}, \exists i \in I$ t.q $\begin{array}{crcl}
\varphi_i : & U_i\text{(conexo)} & \longrightarrow & V
\end{array}$ cumple la definición anterior.\\
ii) $\mathcal{S}_i = \bigcup_{i\in I} \varphi(u_i) $\\
iii) Compatibilidad entre lass cartas.\\\\


\end{document}