\documentclass{article}
\usepackage[utf8]{inputenc}
\usepackage{graphicx}
\usepackage[spanish]{babel}
\usepackage{amssymb,amsmath,geometry,multicol,spalign,hyperref}
\usepackage[usenames,dvipsnames]{xcolor}
\usepackage{tikz,mathtools}
\usepackage{pgfplots}
\pgfplotsset{every axis/.append style={
                    axis x line=middle,    % put the x axis in the middle
                    axis y line=middle,    % put the y axis in the middle
                    axis line style={<->,color=blue}, % arrows on the axis
                    xlabel={$x$},          % default put x on x-axis
                    ylabel={$y$},          % default put y on y-axis
            }}
\usepackage{etoolbox} %titulo
\makeatletter %titulo
\patchcmd{\@maketitle}{\vskip 2em}{\vspace*{-3cm}}{}{} %titulo
\makeatother %titulo
\usepackage{vmargin}
\setpapersize{A4}
\setmargins{2.5cm}       % margen izquierdo
{1.5cm}                        % margen superior
{16.5cm}                      % anchura del texto
{23.42cm}                    % altura del texto
{10pt}                           % altura de los encabezados
{1cm}                           % espacio entre el texto y los encabezados
{0pt}                             % altura del pie de página
{2cm}                           % espacio entre el texto y el pie de página
\title{Hamming}
\author{Andoni Latorre Galarraga y Mariana Zaballa Bernabé}
\date{}
\newcommand{\bb}[1]{\mathbb{#1}}
\newcommand{\R}{\bb{R}}
\newcommand{\nota}[3][2ex]{
    \underset{\mathclap{
        \begin{tikzpicture}
          \draw[->] (0, 0) to ++(0,#1);
          \node[below] at (0,0) {#3};
        \end{tikzpicture}}}{#2}
}
\begin{document}

\maketitle

\noindent Para calcular $p_1,p_2,p_3$ hemos utilizado la función $\textcolor{blue}{\text{RESIDUO}}$
$$
\begin{array}{c}
p_1 = \textcolor{blue}{\text{RESIDUO}}(a_1 + a_2 + a_4; 2) \\
p_2 = \textcolor{blue}{\text{RESIDUO}}(a_1 + a_3 + a_4; 2) \\
p_3 = \textcolor{blue}{\text{RESIDUO}}(a_2 + a_3 + a_4; 2)
\end{array}
$$

\noindent Hemos calculado $s_1,s_2,s_3$ de la misma manera que $p_1,p_2,p_3$. Para calcular $r_1,r_2,r_3$ también hemos utilizado la función $\textcolor{blue}{\text{RESIDUO}}$.
$$
r_j = \textcolor{blue}{\text{RESIDUO}}(s_j + q_j ; 2)
$$

\noindent La razón por la que funciona el código es que si tenmos dos números diferentes, necesariamente uno va a ser un 1 y el otro un 0, es decir $s_j+q_j=1\equiv_2 1$. Si los dos son iguales tenemos que $s_j+q_j= 2s_j=2q_j\equiv_2 0$. Para escribir $r_3r_2r_1$ hemos ''pensado'' en base 10 en la celda $\textcolor{orange}{\text{C21}}$ y en base 2 en la celda $\textcolor{orange}{\text{D21}}$.
$$
\begin{array}{l}
    $\textcolor{orange}{\text{C21}}$ = 100r_3+10r_2+r_1 \\
    $\textcolor{orange}{\text{D21}}$ = 4r_3+2r_2+r_1
\end{array}
$$
\noindent Para corregir el código hemos utilizado
$$
\begin{array}{rl}
\textcolor{Orchid}{\text{B18}} & = \textcolor{blue}{\text{RESIDUO}}(\textcolor{orange}{\text{B11}} + \textcolor{blue}{\text{SI}}(\textcolor{orange}{\text{C21}
}-\text{B4} = 0;1;0);2) \\
\textcolor{Orchid}{\text{C18}} & = \textcolor{blue}{\text{RESIDUO}}(\textcolor{orange}{\text{C11}} + \textcolor{blue}{\text{SI}}(\textcolor{orange}{\text{C21}
}-\text{C4} = 0;1;0);2) \\
\textcolor{Orchid}{\text{D18}} & = \textcolor{blue}{\text{RESIDUO}}(\textcolor{orange}{\text{D11}} + \textcolor{blue}{\text{SI}}(\textcolor{orange}{\text{C21}
}-\text{D4} = 0;1;0);2) \\
\textcolor{Orchid}{\text{E18}} & = \textcolor{blue}{\text{RESIDUO}}(\textcolor{orange}{\text{E11}} + \textcolor{blue}{\text{SI}}(\textcolor{orange}{\text{C21}
}-\text{E4} = 0;1;0);2) \\
\textcolor{Orchid}{\text{F18}} & = \textcolor{blue}{\text{RESIDUO}}(\textcolor{orange}{\text{F11}} + \textcolor{blue}{\text{SI}}(\textcolor{orange}{\text{C21}
}-\text{F4} = 0;1;0);2) \\
\textcolor{Orchid}{\text{G18}} & = \textcolor{blue}{\text{RESIDUO}}(\textcolor{orange}{\text{G11}} + \textcolor{blue}{\text{SI}}(\textcolor{orange}{\text{C21}
}-\text{G4} = 0;1;0);2) \\
\textcolor{Orchid}{\text{H18}} & = \textcolor{blue}{\text{RESIDUO}}(\textcolor{orange}{\text{H11}} + \textcolor{blue}{\text{SI}}(\textcolor{orange}{\text{C21}
}-\text{H4} = 0;1;0);2) \\
\end{array}
$$
Ya que sumar 1 módulo 2 es cambiar el bit.
$$
\begin{array}{l}
    \textcolor{blue}{0} + 1 \equiv_2 \textcolor{blue}{1} \\
    \textcolor{blue}{1} + 1 \equiv_2 \textcolor{blue}{0}
\end{array}
$$
Para la codificación del segundo código $(7,4)$, el proceso es análogo al primero.
\end{document}