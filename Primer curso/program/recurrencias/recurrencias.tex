\documentclass{article}
\usepackage[utf8]{inputenc}
\usepackage{amsmath}%esta linea no estaba en el original
\title{Recurrencias}
\author{Andoni Latorre Galarraga}
\date{}

\begin{document}

\maketitle

\section{}
$$
t(n) = \log_2(n) + t(\frac{n}{2}) = \log_2(n) + \log_2(\frac{n}{2}) + t(\frac{n}{4})
$$
$$
= 2 \log_2(n) - 1 + t(\frac{n}{4}) = 2 \log_2(n) - 1 + \log_2(\frac{n}{4}) + t(\frac{n}{8})
$$
$$
= 3 \log_2(n) - 1 - 2 + t(\frac{n}{8}) = 4 \log_2(n) - 1 - 2 -3 + t(\frac{n}{16})
$$
$$
= k \log_2 n - \frac{k(k-1)}{2} + t(\frac{n}{2^k})
$$
$$
1 = \frac{n}{2^k} \quad \Leftrightarrow \quad k = \log_2(n)
$$
$$
t(n) = \log_2(n) \log_2 n - \frac{\log_2(n)(\log_2(n)-1)}{2} + t(1) = \log_2^2(n) - \frac{\log_2^2(n) - \log_2(n)}{2} + 1
$$
$$
t(n) = \log_2^2(n) - \frac{\log_2(n)}{2} + 1
$$
\section{}
$$
t(n) = n^2 + t(n-1) = n^2 + (n-1)^2 + t(n-2) = \cdots = n^2 + (n+1)^2 + \cdots + 0^2 + t(0)
$$
$$
t(n) = \frac{n(n+1)(2n+1)}{6}+1
$$
\section{}
\textit{Recursion.py}\\
Es evidente que el metodo de expansión no lleva a ninguna parte. Intentemos encontrar una cota superior $t_{\text{sup}}(n)$. Y una cota inferior $t_{\text{inf}}(n) = 1$
$$
t_{\text{sup}}(n)=1+2 \cdot t(n-1)= 1 + 2 + 4 \cdot t(n-2)
$$
$$
\cdots = \frac{k^2+1}{2} + 2^k \cdot t(n-k)
$$
$$
t_{\text{sup}}(n) = \frac{n^2+1}{2} + 2^n \cdot t(0) = \frac{n^2+1}{2} + 2^n
$$
$$
t(n,K) \in O(2^n) \land t(n,k) \in \Omega(1)
$$
\end{document}